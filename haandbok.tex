% Options for packages loaded elsewhere
\PassOptionsToPackage{unicode}{hyperref}
\PassOptionsToPackage{hyphens}{url}
%
\documentclass[
  12pt,
  a4paper, 12pt]{article}
\title{\includegraphics[width=1in,height=\textheight]{figs/norce_logo.png}\\
\bigskip  
Hvordan gjennomføre borgerpanel}
\author{Sveinung Arnesen (red.)\\
Henrik Litlere Bentsen\\
Pål Bjørseth\\
Anne Lise Fimreite (red.)\\
Arild Ohren\\
Jon Kåre Skiple\\
Jacob Aars}
\date{}

\usepackage{amsmath,amssymb}
\usepackage{lmodern}
\usepackage{iftex}
\ifPDFTeX
  \usepackage[T1]{fontenc}
  \usepackage[utf8]{inputenc}
  \usepackage{textcomp} % provide euro and other symbols
\else % if luatex or xetex
  \usepackage{unicode-math}
  \defaultfontfeatures{Scale=MatchLowercase}
  \defaultfontfeatures[\rmfamily]{Ligatures=TeX,Scale=1}
\fi
% Use upquote if available, for straight quotes in verbatim environments
\IfFileExists{upquote.sty}{\usepackage{upquote}}{}
\IfFileExists{microtype.sty}{% use microtype if available
  \usepackage[]{microtype}
  \UseMicrotypeSet[protrusion]{basicmath} % disable protrusion for tt fonts
}{}
\makeatletter
\@ifundefined{KOMAClassName}{% if non-KOMA class
  \IfFileExists{parskip.sty}{%
    \usepackage{parskip}
  }{% else
    \setlength{\parindent}{0pt}
    \setlength{\parskip}{6pt plus 2pt minus 1pt}}
}{% if KOMA class
  \KOMAoptions{parskip=half}}
\makeatother
\usepackage{xcolor}
\IfFileExists{xurl.sty}{\usepackage{xurl}}{} % add URL line breaks if available
\IfFileExists{bookmark.sty}{\usepackage{bookmark}}{\usepackage{hyperref}}
\hypersetup{
  pdftitle={  Hvordan gjennomføre borgerpanel},
  hidelinks,
  pdfcreator={LaTeX via pandoc}}
\urlstyle{same} % disable monospaced font for URLs
\usepackage{longtable,booktabs,array}
\usepackage{calc} % for calculating minipage widths
% Correct order of tables after \paragraph or \subparagraph
\usepackage{etoolbox}
\makeatletter
\patchcmd\longtable{\par}{\if@noskipsec\mbox{}\fi\par}{}{}
\makeatother
% Allow footnotes in longtable head/foot
\IfFileExists{footnotehyper.sty}{\usepackage{footnotehyper}}{\usepackage{footnote}}
\makesavenoteenv{longtable}
\usepackage{graphicx}
\makeatletter
\def\maxwidth{\ifdim\Gin@nat@width>\linewidth\linewidth\else\Gin@nat@width\fi}
\def\maxheight{\ifdim\Gin@nat@height>\textheight\textheight\else\Gin@nat@height\fi}
\makeatother
% Scale images if necessary, so that they will not overflow the page
% margins by default, and it is still possible to overwrite the defaults
% using explicit options in \includegraphics[width, height, ...]{}
\setkeys{Gin}{width=\maxwidth,height=\maxheight,keepaspectratio}
% Set default figure placement to htbp
\makeatletter
\def\fps@figure{htbp}
\makeatother
\setlength{\emergencystretch}{3em} % prevent overfull lines
\providecommand{\tightlist}{%
  \setlength{\itemsep}{0pt}\setlength{\parskip}{0pt}}
\setcounter{secnumdepth}{5}
\newlength{\cslhangindent}
\setlength{\cslhangindent}{1.5em}
\newlength{\csllabelwidth}
\setlength{\csllabelwidth}{3em}
\newlength{\cslentryspacingunit} % times entry-spacing
\setlength{\cslentryspacingunit}{\parskip}
\newenvironment{CSLReferences}[2] % #1 hanging-ident, #2 entry spacing
 {% don't indent paragraphs
  \setlength{\parindent}{0pt}
  % turn on hanging indent if param 1 is 1
  \ifodd #1
  \let\oldpar\par
  \def\par{\hangindent=\cslhangindent\oldpar}
  \fi
  % set entry spacing
  \setlength{\parskip}{#2\cslentryspacingunit}
 }%
 {}
\usepackage{calc}
\newcommand{\CSLBlock}[1]{#1\hfill\break}
\newcommand{\CSLLeftMargin}[1]{\parbox[t]{\csllabelwidth}{#1}}
\newcommand{\CSLRightInline}[1]{\parbox[t]{\linewidth - \csllabelwidth}{#1}\break}
\newcommand{\CSLIndent}[1]{\hspace{\cslhangindent}#1}
\usepackage[T1]{fontenc}
\usepackage[utf8]{inputenc}
\usepackage[normalem]{ulem}
\usepackage[font=singlespacing]{caption}
\usepackage{
  sectsty, lmodern, setspace, graphicx, booktabs, geometry, hyperref,
  tabularx, color, subcaption, bookmark, ragged2e, bookmark, arydshln, 
  minitoc, fancyhdr, xcolor
}

%% Font-related
\usepackage{libertine, libertinust1math}
\usepackage[scaled=0.9]{inconsolata}

%% Setup
\geometry{
  hmargin = 2.5cm,
  vmargin = 2.5cm
}


\renewcommand{\contentsname}{Innholdsfortegnelse}
\renewcommand{\listfigurename}{Figurliste}
\renewcommand{\figurename}{Figur}
\renewcommand{\listtablename}{Tabelliste}
\renewcommand{\tablename}{Tabell}
\renewcommand{\abstractname}{\vspace{-\baselineskip}}
\renewcommand{\refname}{Referanser}

\usepackage{fancyhdr}
\pagestyle{fancy}


% page number on the left of even pages and right of odd pages
%\fancyfoot[LE,RO]{\thepage}
\usepackage{booktabs}
\usepackage{longtable}
\usepackage{array}
\usepackage{multirow}
\usepackage{wrapfig}
\usepackage{float}
\usepackage{colortbl}
\usepackage{pdflscape}
\usepackage{tabu}
\usepackage{threeparttable}
\usepackage{threeparttablex}
\usepackage[normalem]{ulem}
\usepackage{makecell}
\usepackage{xcolor}
\ifLuaTeX
  \usepackage{selnolig}  % disable illegal ligatures
\fi

\begin{document}
\maketitle
\begin{abstract}
\includegraphics{figs/2x/Forside.png}
\end{abstract}

\newpage
\tableofcontents

\newpage
\listoftables
\listoffigures

\newpage

\hypertarget{om-rapporten}{%
\section{Om rapporten}\label{om-rapporten}}

Prosjekttittel: Demokratisk innovasjon i praksis -- Forskning på medvirkning og legitimitet i kommunale beslutningsprosesser\\
Institusjon: NORCE Helse og samfunn\\
Oppdragsgiver(e): Norges forskningsråd (prosjektnummer 295892)\\
Gradering: Åpen\\
Rapportnr.: 38-2022\\
ISBN: 978-82-8408-254-7\\
Antall sider: 85\\
Publiseringsmnd.: November, 2022\\
Denne versjonen: 02. november, 2022\\
Url: \url{https://hdl.handle.net/11250/3028610}

\begin{figure}
\centering
\includegraphics[width=0.2\textwidth,height=\textheight]{"figs/qrcode.png"}
\caption{QR-kode til rapport i vitenarkivet}
\end{figure}

\newpage

\hypertarget{forord}{%
\section{Forord}\label{forord}}

\begin{quote}
Sveinung Arnesen og Anne Lise Fimreite
\end{quote}

I en tid hvor demokratiet settes på prøve også i etablerte vestlige demokratier, er viktigheten av å utvikle demokratiske prosesser kanskje viktigere enn noensinne (\protect\hyperlink{ref-dahlberg2015democratic}{Dahlberg, Linde, and Holmberg 2015}). Internasjonalt ser vi lav, og ofte synkende, tillit til grunnleggende demokratiske institusjoner som politiske partier og regjeringer (\protect\hyperlink{ref-dalton2004democratic}{Dalton 2004}). Samtidig eksisterer det store forskjeller mellom innbyggerne med tanke på hvem som involverer seg i politikk. Denne politiske ulikheten sammenfaller ofte med andre typer ulikhet i samfunnet, som økonomi, helse og utdanning. Den politiske ulikheten gir seg utslag i blant annet lavere valgdeltakelse og engasjement i politiske partier -- både nasjonalt og lokalt.

Norge er generelt et land med høy tillit og høy politisk deltakelse, men også i vårt land finnes det tydelige tegn på lignende mønstre. En lokaldemokratirapport fra Bergen som analyserte valgdeltakelsen til alle kommunens innbyggere med stemmerett i 2015-valget, fant store forskjeller basert på sosiale bakgrunnsvariabler som alder, utdanning og kjønn (\protect\hyperlink{ref-fimreitebyen}{Fimreite 2018}): Valgdeltakelsen blant unge menn uten høyere utdannelse var ned mot 30 prosent, som er halvparten så lav som den samlede valgdeltakelsen dette året. Også geografisk var det store forskjeller. I Vadmyra var valgdeltakelsen på 51 prosent, mens på Bønes var den på 72 prosent. De samme forskjellene så vi også i 2019-valget, men da var valgdeltakelsen generelt noe høyere. Valgkretsen med høyest deltakelse var den samme i begge valgene: Bønes med 77 prosent. Ved 2019-valget var det imidlertid valgkretsen Ny-Kronborg som hadde lavest deltakelse, med 57 prosent.

Det er ofte ressurssterke grupper i samfunnet som benytter seg av mulighetene til å påvirke politiske prosesser. Dette er et gode. Utfordringen ligger i å øke engasjementet også blant innbyggerne som vanligvis ikke er politisk aktive. Det er viktig å sikre politisk deltakelse blant de mindre engasjerte for å sikre at alle stemmer blir hørt. Hvis undergrupper i samfunnet føler seg avskåret fra det politiske systemet, utgjør det ikke bare et problem for hver enkelt, men også for samfunnet som helhet. Hvis politisk fremmedgjorte innbyggere opplever at systemet ikke er laget for dem, kan de ende opp med å søke alternative, ikke-demokratiske løsninger for å få gjennomslag for sine interesser. Derfor er det viktig å øke deltakelsen blant alle innbyggere, hvor vanskelig det enn er. Dette er ikke noe som kan gjøres i en håndvending, men noe som krever kontinuerlig oppmerksomhet.

Når borgerpaneler brukes riktig, har de potensial til å lede til bedre beslutninger. Videre kan de styrke innbyggernes oppfatning om at den demokratiske prosessen er rettferdig og tar hensyn til innbyggernes synspunkter, som i sin tur regnes som en avgjørende bestanddel av et robust og legitimt demokrati (\protect\hyperlink{ref-marien2019fair}{Marien and Werner 2019}).

Med denne håndboken ønsker vi å øke oppmerksomheten om borgerpaneler og være til inspirasjon for lesere som vurderer å gjennomføre et borgerpanel selv. Den er også sluttproduktet til forskningsprosjektet «Demokratisk innovasjon i praksis: Forskning på medvirkning og legitimitet i kommunale beslutningsprosesser (DEMOVATE)». Prosjektet har blitt ledet av Bergen kommune, mens forskningsinstituttet NORCE har hatt den faglige ledelsen. Håndboken baserer seg på erfaringer gjort i DEMOVATE og i forskningsaktiviteter knyttet til demokratisk innovasjon ved Universitetet i Bergen, Stanford University og NORCE. Håndbokens målgruppe er alle som er interessert i det norske demokratiet. Vi håper den kan komme til nytte for politikere, byråkrater, personer som engasjerer seg i sivilsamfunnet, og alle andre som er nysgjerrige på borgerpaneler, og som kanskje går med tanker om å sette i stand et selv.

Boken er strukturert i tre deler. Den første delen tar et overordnet blikk på borgerpaneler, hva som definerer dem, og hvordan de blir brukt inn i den demokratiske beslutningsprosessen. Den andre delen løfter frem praktiske problemstillinger man må ta hensyn til under planlegging av borgerpaneler. Rådene baserer seg på tre borgerpaneler som artikkelforfatterne selv har gjennomført. Den tredje og siste delen bringer inn flere erfaringer fra andre kommuner. Vi får også innblikk i hvordan borgerpaneler og medvirkning generelt oppfattes av en ansatt i kommuneadministrasjonen.

\newpage

\hypertarget{del-i-borgerpaneler-hva-og-hvorfor}{%
\section{DEL I: BORGERPANELER -- HVA OG HVORFOR?}\label{del-i-borgerpaneler-hva-og-hvorfor}}

\begin{quote}
Arild Ohren og Jacob Aars
\end{quote}

\includegraphics{figs/1x/Del1.png}
Det norske demokratiet, inkludert lokaldemokratiet, er i hovedsak et indirekte demokrati, der innbyggerne øver innflytelse over politikken gjennom valgte representanter. De folkevalgte utgjør en elite i dobbel forstand -- ikke bare er de en politisk elite siden de alltid utgjør et mindretall av befolkningen, de er også en sosial elite fordi de folkevalgte ikke utgjør et tverrsnitt av befolkningen med hensyn til sosiale og demografiske bakgrunnstrekk. Høyt utdannede middelaldrende menn har tradisjonelt vært overrepresentert blant de folkevalgte sammenlignet med den øvrige befolkningen. De valgte er ikke et speilbilde av velgerne, og demokratiet har derfor et representativitetsproblem. I tillegg vil de deltakelseskanalene som eksisterer, for eksempel valgkanalen, ofte gi begrensede muligheter for meningsutveksling og diskusjon blant deltakerne. Det å avgi stemme ved valg er utvilsomt en form for meningsytring, men selve ytringen er begrenset, og det stilles heller ikke noe krav til velgerne om å begrunne det valget de tar. Vi kan derfor si at demokratiet også har et deliberasjonsproblem i den betydningen at mange deltakelseskanaler gir lite rom for overveielse og meningsutveksling.

Både kommunale og statlige myndigheter har forsøkt å gjennomføre tiltak for å bøte på disse svakhetene ved demokratiet. I tillegg til bred og representativ deltakelse er det et formål med mange av disse tiltakene at deltakerne skal ha anledning til å fordype seg i en sak ved at de kan samtale og tenke seg om før de tar endelig stilling. Målet er med andre ord å forene et kvantitativt og et kvalitativt krav til demokratisk deltakelse. Samlebetegnelser for slike myndighetsinitierte tiltak er demokratipolitikk, demokratisk fornyelse eller demokratisk innovasjon. I denne håndboken retter vi oppmerksomheten mot én kategori av slike tiltak, det vi har kalt borgerpaneler.

Offentlige myndigheters tiltak for å trekke innbyggerne med i politiske prosesser utenom valg bidrar ikke alltid til å jevne ut den skjeve sosiale fordelingen i deltakelsen. Et eksempel på dette er folkemøter som kommunene inviterer til, der deltakerne som regel ikke gjenspeiler befolkningen når det gjelder utdanning eller alder. Utvelgelse av deltakere til slike møter foregår i form av selvutvelgelse, det vil si at folk bestemmer selv om de vil komme eller ikke. Det er gjerne «gjengangere» som møter opp. Et folkemøte representerer en ekstra deltakelseskanal for dem som allerede er aktive. En utfordring for initiativtakerne er derfor å gi en stemme til de innbyggerne som sjelden eller aldri deltar i politikken. I de tilfellene der deltakelsen er tilnærmelsesvis representativ for befolkningen, slik som i en brukerundersøkelse, får deltakerne lite eller ingen plass til å tenke grundig gjennom de svarene de gir, eller å drøfte disse med andre.

I dette kapitlet skal vi først gjøre rede for begrepet «demokratisk innovasjon» og hvilke kategorier av ordninger dette omfatter. Dernest skal vi gå nærmere inn på ordninger som sikter mot å kombinere de to hensynene vi har nevnt innledningsvis, representativitet og deliberasjon.

\hypertarget{demokratiske-innovasjoner}{%
\subsubsection{Demokratiske innovasjoner}\label{demokratiske-innovasjoner}}

Når vi snakker om deliberative meningsmålinger, borgerpaneler og så videre, så refererer vi gjerne til demokratiske innovasjoner. Alt dette kan være veldig forvirrende. Så la oss bare kort og godt rydde litt opp i disse begrepene.

\begin{figure}

{\centering \includegraphics[width=0.8\linewidth]{figs/familier} 

}

\caption{Familier av demokratiske innovasjoner}\label{fig:unnamed-chunk-1}
\end{figure}

Det første begrepet vi har her, er «demokratiske innovasjoner». Dette er da et vanlig begrep som ikke bare brukes innenfor fagfeltet, men også i praksis. Begrepet kommer fra Graham Smiths bok fra 2009, kalt Democratic Innovations. For å gjøre det klart da hva vi snakker om, bruker man Graham Smiths definisjon på hva en demokratisk innovasjon faktisk er. Han skriver at demokratiske innovasjoner er institusjoner som er spesielt designet for å øke og/eller forbedre deltakelse i politiske prosesser (\protect\hyperlink{ref-smith_democratic_2009}{Smith 2009}). Dette har blitt utdypet av Elstub og Escobar. De skriver at demokratiske innovasjoner er prosesser eller institusjoner som er utviklet for å omforme og utdype innbyggernes rolle i styringsprosesser, ved å øke mulighetene for deltakelse, deliberasjon og innflytelse (\protect\hyperlink{ref-escobar_defining_2019}{Escobar and Elstub 2019}). I bunn og grunn er demokratisk innovasjon en paraplybetegnelse for mange forskjellige metoder og prosesser, som alle har det til felles at de prøver å lage nye kanaler der folk kan påvirke beslutningsprosesser. I all hovedsak vil disse teknikkene være myndighetsinitierte. Som man ser i figur 1, har man da mange forskjellige familier av demokratiske innovasjoner. Dette kan for eksempel være deltakende budsjettering, folkeavstemninger og innbyggerinitiativ, og så videre. Disse familiene har forskjellige fokus og forskjellige teoretiske utgangspunkter, og de er også designet for å løse forskjellige typer problemer. En av disse familiene er det vi kaller deliberative borgerpaneler. Heretter kaller vi dette borgerpaneler.

\hypertarget{nuxe6rmere-om-borgerpaneler}{%
\subsubsection{Nærmere om borgerpaneler}\label{nuxe6rmere-om-borgerpaneler}}

De forskjellige formene for borgerpaneler bygger alle på en idé om at hensynet til representativitet bør kombineres med hensynet til deliberasjon. Sagt på en annen måte søker borgerpaneler på ulikt vis å forene hensynet til bredde og dybde i deltakelsen. Deltakelsesordningene sikter mot å inkludere et bredt og representativt utvalg av befolkningen, dermed også innbyggere som vanligvis ikke er politisk aktive. Men det er lett å tenke seg hvordan de to hensynene, representativitet og deliberasjon, vil kunne komme i konflikt med hverandre. Øker man antallet deltakere, vil det bli færre muligheter for den enkelte til å delta aktivt i debatt og rådslagning. De to hensynene kan utgjøre avveininger ved at mer av det ene hensynet lett vil gå på bekostning av det andre. Ulike kombinasjoner av disse to dimensjonene ved deltakelse er vist i skjemaet nedenfor, med eksempler på deltakelsesordninger/-kanaler i hver av kategoriene.

\begin{table}[!h]

\caption{\label{tab:tbl-borgerpaneler}Klassifikasjon av deltakelsestiltak/-kanaler}
\centering
\begin{tabular}[t]{lll}
\toprule
\textbf{Samtale} & \textbf{Representativitet} & \textbf{Deltakelsestiltak}\\
\midrule
\cellcolor{gray!6}{} & \cellcolor{gray!6}{} & \cellcolor{gray!6}{Folkemøte/høring uten mulighet for diskusjon}\\
\cmidrule{3-3}
 
 & \multirow[t]{-2}{*}{\raggedright\arraybackslash Nei} & Undersøkelse med skjevt utvalg\\
\cmidrule{2-3}
 
\cellcolor{gray!6}{} & \cellcolor{gray!6}{} & \cellcolor{gray!6}{Brukerundersøkelse}\\
\cmidrule{3-3}
 
\multirow[t]{-4}{*}{\raggedright\arraybackslash Nei} & \multirow[t]{-2}{*}{\raggedright\arraybackslash Ja} & Valg\\
\cmidrule{1-3}
 
\cellcolor{gray!6}{} & \cellcolor{gray!6}{} & \cellcolor{gray!6}{Folkemøte}\\
\cmidrule{3-3}
 
 &  & Høring\\
\cmidrule{3-3}
 
\cellcolor{gray!6}{} & \cellcolor{gray!6}{} & \cellcolor{gray!6}{Kommunestyre}\\
\cmidrule{3-3}
 
 &  & Charrette\\
\cmidrule{3-3}
 
\cellcolor{gray!6}{} & \cellcolor{gray!6}{\multirow[t]{-5}{*}{\raggedright\arraybackslash Nei}} & \cellcolor{gray!6}{Deltakende budsjettering}\\
\cmidrule{2-3}
 
 &  & Borgerpanel\\
\cmidrule{3-3}
 
\cellcolor{gray!6}{\multirow[t]{-7}{*}{\raggedright\arraybackslash Ja}} & \cellcolor{gray!6}{\multirow[t]{-2}{*}{\raggedright\arraybackslash Ja}} & \cellcolor{gray!6}{Deliberativ meningsmåling}\\
\bottomrule
\end{tabular}
\end{table}

Skjemaet viser at forholdet mellom det kvantitative og det kvalitative kravet til deltakelse kan være motsetningsfylt. Deltakelseskanaler som er åpne for et bredt tilfang av deltakere, for eksempel valg, vil normalt ikke gi mye rom for refleksjon eller meningsbrytning. Motsatt vil deltakelseskanaler der deltakerne får god anledning til å snakke sammen og tenke seg om, for eksempel høringer eller folkemøter, vanligvis ikke omfatte et bredt antall deltakere.

Det første utfallet i tabellen viser en situasjon der det ikke gis anledning til deliberasjon, samtidig som deltakerne ikke er representative for befolkningen. Eksempler her er folkemøter uten mulighet for utspørring, samtale eller diskusjon og undersøkelser der deltakerne ikke er representative for populasjonen. Den andre kategorien dekker et utfall der deltakerne utgjør et tverrsnitt av befolkningen, men har liten anledning til å drøfte de aktuelle sakene med de andre deltakerne. Bruker- eller opinionsundersøkelser er det klareste eksemplet. Under en viss tvil har vi plassert valgdeltakelse innenfor denne kategorien, men det kan argumenteres for at valg ofte innebærer en hel del debatt om de sakene som er på dagsorden, gjennom valgkampen. Samtidig kan det innvendes at de som deltar i valg, heller ikke utgjør et speilbilde av befolkningen. Blant hjemmesitterne er det betydelig flere unge og lavt utdannede enn blant de som avlegger stemme. Likevel er valgdeltakelsen jevnere sosialt fordelt enn andre, mer krevende deltakelsesformer. I den tredje kategorien finner vi deltakelsesordninger som gir rom for samtale og overveielse, men der deltakerne ikke er representative for innbyggerne. Debatt og meningsutveksling er en sentral ingrediens i et kommunestyre eller andre folkevalgte forsamlinger, selv om man kan stille spørsmål ved hvor ofte enkelte representanter skifter mening som følge av et innlegg i et kommunestyre eller i stortingssalen. Når det gjelder sammensetningen av de folkevalgte, kan vi si at kjønnsprofilen over tid har blitt likere kjønnssammensetningen i befolkningen, men de folkevalgte utgjør stadig en utdanningselite sammenlignet med velgerne. I siste kategori finner vi de deltakelsesordningene som i størst grad kombinerer bredde og dybde, det vil si representativitet og samtale. Som tidligere nevnt bruker vi samlebetegnelsen «borgerpanel» om denne typen demokratiordning. Terminologien er imidlertid noe springende i litteraturen. Her refererer begrepet borgerpanel ofte til en mindre gruppe av deltakere som sitter samlet i et antall dager, og som blir bedt om å levere noen anbefalinger om et gitt politisk spørsmål. Anbefalingene gis gjerne i form av en rapport. Et annet begrep som ofte har vært brukt om denne typen ordning, er borgerjury, der man kan se for seg at deltakerne avsier en dom i et utvalgt policyspørsmål. I denne boken omfatter borgerpanelbegrepet både ordninger med færre (men tilfeldig valgte) deltakere og ordninger der man tar sikte på at utvalget av deltakere skal være statistisk representativt for populasjonen.

Knytter vi spørsmålet om deltakelsesordninger til kommunene og lokaldemokratiet, er det en sentral idé at lokalbefolkningen representerer en ressurs i lokalpolitiske beslutningsprosesser. Innbyggerne er forvaltere av kunnskap som kommunen kan dra nytte av. De kjenner de lokale forholdene og virkningene av lokal politikk og tjenesteyting. I tillegg har innbyggerne både idéer og synspunkter som det kan være viktig for kommunale beslutningstakere å kjenne til. I siste instans er innbyggerne, i kraft av å være velgere eller politiske deltakere på andre måter, kilden til legitimitet for den politikken som lokalpolitikerne er med på å utforme. Det å gjennomføre tiltak med representative befolkningsutvalg er i dette perspektivet å ta innbyggernes ressurser i bruk. Lekmannskompetanse kan være et stikkord for denne typen demokratiinitiativ. Det er ikke nødvendig med en formell utdannelse for å få denne kompetansen. En slik kompetanse utvikles gjennom erfaringer, og slike erfaringer kan være spesifikke for ulike sosiale grupper. Når antallet folkevalgte blir færre og avstanden mellom velgere og valgte blir større, blir det desto viktigere, om enn muligens vanskeligere, å utnytte lekmannskompetansen.

En mulig tolkning av idealet om lekmannskompetanse kan være at det ikke holder å avdekke overfladiske holdninger hos befolkningen. Lekmannsskjønnet utvikles nettopp i nærkontakt med andre innbyggere og i tilknytning til saker der innbyggerne gjerne har en interesse eller er berørt.

Resultatet av representative deltakelsesprosesser kan betraktes som en kvalifisert meningsytring. Deltakernes stillingtagen er basert på argumenter og begrunnelser som andre kunne ha kommet frem til dersom de hadde hatt samme mulighet til å lære om og diskutere et saksfelt. Dermed vil en uttalelse fra en representativt sammensatt forsamling ha større legitimitet enn holdninger som kommer frem under for eksempel et ordinært folkemøte.

I tillegg til målet om å trekke et mer representativt utvalg av innbyggere inn i politiske beslutningsprosesser kan et formål med slike teknikker være å gi et bedre og bredere grunnlag for å stille de folkevalgte til ansvar for politikken de fører. Politikere må argumentere foran et utvalg av innbyggere. De stilles til ansvar der og da. Videre vil deltakelse på en innbyggerhøring kunne øke medborgernes kompetanse som demokratiske kontrollører også ved senere anledninger. I de tilfellene der deliberative prosesser formidles videre til den bredere offentligheten, er det mulig at deltakelsen blant de utvalgte kan skape et engasjement hos innbyggere som ikke deltar på samlingen selv. Når kommunen arrangerer et borgerpanel, kan vi videre tenke oss at oppmerksomheten rundt lokale politiske saker øker. Ved å bidra til offentlig debatt lokalt kan representative deltakelsesordninger rette oppmerksomheten mot lokale stridsspørsmål og kanskje trenge de nasjonale sakene i bakgrunnen for en periode. Dette er en viktig forutsetning for at kommunepolitikere skal kunne holdes til ansvar for lokal politikk, og for at kommunevalg ikke blir rene avspeilinger av nasjonale valg.

\hypertarget{borgerpanelenes-opprinnelse-og-utvikling}{%
\subsubsection{Borgerpanelenes opprinnelse og utvikling}\label{borgerpanelenes-opprinnelse-og-utvikling}}

Den moderne historien om forsøk med borgerpaneler i vesten startet på 1970-tallet i USA og Tyskland. I USA utviklet Ned Crosby det han kalte «Citizens' Jury»-modellen (borgerjury), og i Tyskland utviklet Peter Dienel det han kalte «Planning Cell»-modellen (planleggingscelle/-nettverk). Fra disse tidlige eksperimentene kan vi spore noen andre viktige utviklinger: I 1985 utviklet det danske teknologirådet «Consensus Conference»-modellen (lekfolkskonferanse), på 1990-tallet utviklet James Fishkins «Deliberative Poll»-modellen (deliberativ høring/meningsmåling), i 2003--2004 ble Citizens Assembly (borgersamling) i British Columbia etablert, i 2008 ble Oregon Citizens Initiative etablert, og kanskje aller mest kjent: i 2016 ble borgersamlingen i Irland etablert. Disse kan ses på tidslinjen nedenfor.

\begin{figure}

{\centering \includegraphics[width=0.8\linewidth]{figs/tidslinje} 

}

\caption{Tidslinje over viktige borgerpanel}\label{fig:unnamed-chunk-2}
\end{figure}

I den senere tid har bruken av borgerpaneler økt betraktelig. OECDs rapport «den deliberative bølgen» slår fast at bruken i OECD-landene spesielt har økt etter 2010 (\protect\hyperlink{ref-oecd_innovative_2020}{OECD 2020}). Dette kan ses i sammenheng med den økende interessen rundt demokratiske innovasjoner generelt, som ofte blir sett på som et mulig demokratisk hjelpemiddel for å løse problemene som vi vil stå ovenfor. Kort sagt, det vi ser er at flere demokratier er under press, og at de nåværende demokratiske systemene har noen mangler som gjør at de kanskje ikke er så godt egnet til å løse disse utfordringene. Demokratiske innovasjoner og borgerpaneler har derfor blitt sett på som et av flere demokratiske hjelpemidler som man kan ha i verktøykassen for å styrke våre demokratiske systemer.

Historisk sett har det vært delte meninger om hvilke modeller som kan passe inn i familien som vi kaller borgerpanel. Selv om det tidligere var noe uenighet rundt definisjonen, har det nå blitt en sterk enighet både i fagfeltet og i praksis om hva et borgerpanel er. Feltet har landet på en definisjon som inkluderer typer som Citizens' Juries (borgerpanel/borgerjury), Planning Cell (planleggingscelle), Consensus Conference (lekfolkskonferanse), Citizens' Assembly (borgerforsamling) samt Deliberative Polls (deliberative meningsmålinger). Dette er den mest aksepterte definisjonen på et borgerpanel (\protect\hyperlink{ref-curato_deliberative_2021}{Curato et al. 2021}; \protect\hyperlink{ref-escobar_forms_2017}{Escobar and Elstub 2017}; \protect\hyperlink{ref-farrell_deliberative_2019}{David M. Farrell et al. 2019}; \protect\hyperlink{ref-lacelle-webster_citizens_2021}{Lacelle-Webster and Warren 2021}). Basert på dette har vi lagd tabell \ref{tab:tbl-opprinnelse} med de forskjellige modellene som faller inn under denne definisjonen.

\newgeometry{margin=1cm}
\begin{landscape}\begin{table}[!h]

\caption{\label{tab:tbl-opprinnelse}Typer av borgerpanel}
\centering
\resizebox{\linewidth}{!}{
\begin{tabular}[t]{llllll}
\toprule
\textbf{Type} & \textbf{Originalnavn} & \textbf{Opprinnelse} & \textbf{Deltakerantall} & \textbf{Tid} & \textbf{Resultat}\\
\midrule
\cellcolor{gray!6}{Borgerjury} & \cellcolor{gray!6}{Citizen Jury/ Reference Panels/ Citizen panels} & \cellcolor{gray!6}{Ned Crosby, USA, 1971} & \cellcolor{gray!6}{12 - 36(50)} & \cellcolor{gray!6}{2 - 5 dager} & \cellcolor{gray!6}{Anbefalinger i form av en borgerrapport}\\
 
Plannleggingscelle/Planleggingsnettverk & Planning Cell & Peter Dienel, Tyskland, 1970-tallet & 25 i hver celle, men flere celler. Totalt 100 - 500 & 2 -7 dager & Spørreundersøkelser og en kollektiv rapport fra alle cellene\\
 
\cellcolor{gray!6}{Lekfolkskonferanser} & \cellcolor{gray!6}{Consensus Conference} & \cellcolor{gray!6}{Teknologirådet - The Danish Board of Technology, Danmark, 1987} & \cellcolor{gray!6}{10 - 25} & \cellcolor{gray!6}{3 - 8 dager} & \cellcolor{gray!6}{Anbefalinger i form av en borgerrapport}\\
 
Borgerforsamling/
Borgersamling & Citizen Assembly & Gordon Gibson, Canada 2002 & (50) 99 - 150 & Over flere helger & Detaljerte anbefalinger\\
 
\cellcolor{gray!6}{Borgerhøring/Deliberativ meningsmåling} & \cellcolor{gray!6}{Deliberative Poll} & \cellcolor{gray!6}{James Fishkin, USA, 1994} & \cellcolor{gray!6}{100 (130) - 500} & \cellcolor{gray!6}{En helg} & \cellcolor{gray!6}{En spørreundersøkelse etter prosessen}\\
 
G1000 & G1000 & 27 underskrivere, Belgia, 2011 & 1000 & 1 dag & Avstemming på diverse forslag\\
\bottomrule
\multicolumn{6}{l}{\rule{0pt}{1em}\textsuperscript{1} Basert på Escobar og Elstub (2017), Farrell og Stone (2019), Smith og Setälä (2018).}\\
\end{tabular}}
\end{table}
\end{landscape}
\restoregeometry

Som man kan se ut fra tabellen, varierer typene som faller inn under definisjonen av borgerpanel, i størrelse, tid og hva som blir resultatet. De har derimot følgende klare fellestrekk:

\begin{quote}
et borgerpanel består av uavhengige og fasiliterte gruppediskusjoner mellom en (nær) tilfeldig trukket ut gruppe mennesker som tar imot informasjon fra eksperter og interessegrupper» (\protect\hyperlink{ref-smith_mini-publics_2018}{Smith and Setälä 2018}).
\end{quote}

Ut fra dette kan man si at et borgerpanel har to kjerneelementer (jf. tabell \ref{tab:tbl-kjerneelementer}):

\begin{itemize}
\tightlist
\item
  tilfeldig utvelgelse av deltakere gjennom loddtrekning\\
\item
  at disse deltakerne gjennomgår en deliberativ prosess
\end{itemize}

Disse sentrale elementene er det som kjennetegner prosesser som er en del av familien borgerpanel. Derfor er det viktig å gå litt nærmere inn på disse elementene.

\hypertarget{deliberativ-prosess}{%
\subsubsection{Deliberativ prosess}\label{deliberativ-prosess}}

Det første elementet for et borgerpanel som vi tar for oss, er den deliberative prosessen. Som nevnt ovenfor, er det slik at de som da blir tilfeldig utvalgt, kommer sammen og gjennomgår en såkalt deliberativ prosess. Deliberasjon er derfor helt sentralt i borgerpaneler, og det har også en sentral plass innenfor demokratiteori generelt. Derfor er det viktig å presisere her hva som ofte menes med deliberasjon.
La oss ta en forklaring fra John Gastil. Han skriver at når folk delibererer, så undersøker de et problem nøye og kommer frem til en velbegrunnet løsning etter en periode med inkluderende og respektfull vurdering av ulike synspunkter (\protect\hyperlink{ref-gastil_political_2008}{Gastil 2008, 8}). Her er det en del ting som er pakket inn, så la oss gå gjennom denne prosessen. For John Gastil involverer dette ni viktige punkter i deliberasjon, som kan deles opp i en analytisk del og en sosial del. Disse er oppsummert i tabell \ref{tab:tbl-kjerneelementer}.

\begin{table}[!h]

\caption{\label{tab:tbl-kjerneelementer}Kjerneelementer i deliberasjon}
\centering
\resizebox{\linewidth}{!}{
\begin{tabular}[t]{l}
\toprule
Element\\
\midrule
\addlinespace[0.3em]
\multicolumn{1}{l}{\textbf{Analytisk}}\\
\hspace{1em}\cellcolor{gray!6}{Lag et solid kunnskaps- og læringsgrunnlag}\\
 
\hspace{1em}Diskutér personlig og følelsesmessig erfaringer, samt kjente fakta\\
 
\hspace{1em}\cellcolor{gray!6}{Prioritér verdiene som står på spill}\\
 
\hspace{1em}Reflektér over dine egne verdier samt også verdiene til de andre\\
 
\hspace{1em}\cellcolor{gray!6}{Identifisér et bredt spekter av løsninger}\\
 
\hspace{1em}Brainstorm på en rekke måter å løse problemet\\
 
\hspace{1em}\cellcolor{gray!6}{Vei fordeler, ulemper og avveininger}\\
 
\hspace{1em}Se begrensningene i løsningene som du foretrekker, og fordelene i andres løsninger\\
 
\hspace{1em}\cellcolor{gray!6}{Ta den best mulige avgjørelsen}\\
 
\hspace{1em}Oppdatér din egen mening i lys av det du har lært. Det er ikke nødvendig å ta en felles beslutning\\
 
\addlinespace[0.3em]
\multicolumn{1}{l}{\textbf{Sosialt}}\\
\hspace{1em}\cellcolor{gray!6}{Sørg for at alle får snakke}\\
 
\hspace{1em}Bytt på i samtalen eller ta andre handlinger for å sikre en balansert diskusjon\\
 
\hspace{1em}\cellcolor{gray!6}{Sørg for at alle forstår}\\
 
\hspace{1em}Snakk tydelig til hverandre og be om avklaring når du er forvirret\\
 
\hspace{1em}\cellcolor{gray!6}{Vurder andres ideer og erfaringer}\\
 
\hspace{1em}Lytt nøye til hva andre sier, spesielt når du er uenig\\
 
\hspace{1em}\cellcolor{gray!6}{Respektér andre deltakere}\\
 
\hspace{1em}Anta at andre deltakere er ærlige og at de mener vel. Anerkjenn deres unike livserfaringer og perspektiver\\
\bottomrule
\multicolumn{1}{l}{\rule{0pt}{1em}\textsuperscript{1} Basert på Gastil (2008).}\\
\end{tabular}}
\end{table}

Den analytiske delen skjer ikke nødvendigvis stegvis, men man starter ofte med det første punktet, å lage et solid kunnskaps- og læringsgrunnlag, og man ender alltid opp på det siste punktet, å ta en avgjørelse. Den sosiale delen er ekstremt viktig, og disse punktene er gjeldende gjennom hele prosessen. Dette er altså grunnleggende punkter som må være til stede for at den analytiske delen skal fungere godt.

For å si det kort, så er deliberasjon gjensidig kommunikasjon som innebærer å veie og reflektere over preferanser, verdier og interesser angående saker av felles interesse (\protect\hyperlink{ref-bachtiger_deliberative_2018}{Bächtiger et al. 2018}). Her må man derfor utforske alle sider av saken, som fakta, argumentasjon og erfaringer, og sette disse opp mot hverandre i en åpen, grundig og kritisk diskusjon om problemstillingen. Målet er å komme frem til det som kan sies å være den beste løsningen på problemet. Dette gjøres ved at dine egne meninger settes på prøve og modereres gjennom den deliberative prosessen.

Et eksempel på en deliberativ prosess er juryordningen i rettssystemet. Her får juryen informasjon og fakta gjennom de bevisene som blir lagt frem. De får også høre argumentasjon og motargumentasjon om saken. På denne bakgrunn diskuterer juryen seg imellom alle sider av saken. Etter å ha veid alle elementene opp mot hverandre, vil de komme med sin beslutning om hva de tenker er det riktige.

Dette er et ideal som man skal strekke seg etter, og det kan nok variere hvor mye deliberasjon som finner sted. I demokratiteori stilte man seg spørsmålet om hvordan man kan designe metoder og prosesser som legger til rette for at innbyggere kan deliberere om politiske spørsmål og problemstillinger. Svaret som kom ut av dette, var blant annet borgerpanelene, som er spesielt designet for deliberasjon mellom innbyggere.

Selv om borgerpanelene og prosessen er noe ulikt designet, kan man grovt si at de er designet rundt de samme fasene. Disse fasene vises i tabell \ref{tab:tbl-kjerneelementer}. Den første fasen blir ofte kalt kunnskaps- og læringsfasen. Her kombineres ofte ekspertise og forskning med personlige erfaringer og interesser for at deltakerne bedre skal forstå problemet og konsekvensene. Den første fasen kan innebære at læringsmateriell skal leses på egen hånd, men vanligvis legges det opp til at det gis diverse presentasjoner til borgerpanelet når det er samlet i plenum. Her legger man også opp til at deltakerne kan stille spørsmål til personene som presenterer for borgerpanelet, og også utfordrer dem med de argumentasjonene de kommer med. Det legges også opp til at dersom det er noe man ikke forstår, skal dette påpekes, slik at det blir presentert eller diskutert nærmere.

Den andre fasen kalles diskusjons- og refleksjonsfasen. Her skal deltakerne ta inn over seg det som er fremlagt, og se på mulige løsninger som de skal veie opp mot hverandre. Ofte er det da slik at borgerpanelet deles opp i mindre grupper. Hver gruppe ledes av en moderator, som ikke skal komme med sine egne synspunkter, men sørge for at gruppen ikke «går seg bort», og sørge for at alle får uttale seg og bidra. Dersom deltakerne ikke selv vil, kan moderatorene også presentere hva de har funnet ut, til resten av borgerpanelet i plenum.
Den siste fasen kalles beslutningsfasen, der sluttproduktet blir produsert, for eksempel i form av en spørreundersøkelse eller en rapport.

\hypertarget{tilfeldig-uttrekk}{%
\paragraph{Tilfeldig uttrekk}\label{tilfeldig-uttrekk}}

Det andre elementet i et borgerpanel er at de som deltar i den deliberative prosessen, er tilfeldig plukket ut gjennom loddtrekning. Dette kan kanskje høres revolusjonerende og oppsiktsvekkende ut i dag, men vi må huske på at før fremveksten av politiske massepartier på slutten av det nittende og begynnelsen av det tjuende århundre var det å bruke loddtrekning anerkjent som en mer demokratisk mekanisme for valg av representanter enn valg slik vi kjenner det i dag. Som Aristoteles skrev,

\begin{quote}
når det gjelder ansettelse av politiske ledere, for eksempel, regnes det for demokratisk å benytte loddtrekning, mens valg er oligarkisk (\protect\hyperlink{ref-aristoteles_politikk_2007}{Aristoteles 2007, bk. IV}, Kap 9, 1294b).
\end{quote}

Det å plukke ut representanter gjennom loddtrekning har dermed en lang og tung demokratisk tradisjon. I Athen ble det trukket ut representanter fra befolkningen gjennom loddtrekning, og de som ble plukket ut, hadde stor makt og innflytelse og fikk en sentral posisjon i det politiske systemet. I det femte århundre f.Kr. ble det i Athen laget en lotterimaskin for akkurat denne funksjonen, kalt «kleroterion».

I de senere år har interessen for loddtrekning som en demokratisk utvelgelsesmetode økt igjen, blant annet på grunn av arbeidet innenfor det deliberative demokratiet og eksperimenteringen med borgerpaneler. Man kan si at man igjen har begynt å bruke en demokratisk utvelgelsesmekanisme som tidligere inngikk i den demokratiske verktøykassen vår.

Hvorfor bruker man loddtrekning for borgerpaneler? Her kan man peke på tre elementer: likhet, upartiskhet og mangfold.

Det første aspektet er at det å gjennomføre en loddtrekning er knyttet opp til en intuitiv form for rettferdighet. I en loddtrekning vil man for eksempel ikke belønne de som er de beste til å snakke for seg. I en loddtrekning anses alle borgere som like kompetente, og alle har -\/-- i teorien --- like stor sjanse for å bli plukket ut.

Her kommer også upartiskhet inn i bildet. I mange medvirkningsprosesser har man åpne møter. De som dukker opp på disse møtene, kan være de som roper høyest, eller de som har sterke meninger om eller interesser i saken. Dette er ikke nødvendigvis et problem i seg selv, men i enkelte medvirkningsprosesser er det kanskje ikke ønskelig å at disse personene er sterkt representert. I en deliberativ prosess skal man være åpen for andres meninger, og målet er å komme frem til det som er best for samfunnet som helhet.

En loddtrekning øker sjansen for at de som blir plukket ut til deliberasjon, ikke har en skjult agenda, og at de heller vil forsøke å finne ut hva som er best for samfunnet (\protect\hyperlink{ref-courant_sortition_2019}{Courant 2019}). Dette kan beskytte prosessen mot sterke interesseorganisasjoner og personer med sterke interesser, som kanskje ikke representerer det som er best for samfunnet som helhet. I og med at interesseorganisasjoner kan bli invitert til å snakke sin sak til et borgerpanel, vil de selvsagt få sagt sitt, men sjansen for at de «kupper» prosessen, vil bli betydelig mindre med en loddtrekning.

Et av de viktigste argumentene som blir brukt for loddtrekning, er at man da får en mer representativ gruppe deltakere enn man ville ha fått med andre metoder. Borgerpaneler med loddtrekning blir derfor ofte nevnt som en prosess som generer deskriptiv representasjon. Det vil si at de som deltar, har de samme sosiale eller demografiske bakgrunnene som dem de representerer. Idéen her er at folk kan se for seg at en person som seg selv er representert i borgerpanelet. Dette er spesielt viktig for borgerpaneler. Som nevnt ovenfor er et av hovedelementene i et borgerpanel deliberasjon. Et av de viktigste elementene for å få til en god deliberativ prosess er mangfold -- av perspektiver, meninger, livserfaringer og så videre. Det man har sett, er at det kanskje er nettopp gjennom loddtrekning, og da en mer deskriptiv representasjon, at man sikrer størst mangfold. Med loddtrekning blir personer fra alle sosiale og demografiske bakgrunner representert i borgerpanelene. Fra et av borgerpanelene i Norge observerte en moderator for eksempel at:

\begin{quote}
på mitt bord sitter det en eldre dame med innvandrerbakgrunn, en profesjonell pokerspiller og en bilmekaniker. De kunne ha bodd i samme by i 3000 år og aldri møttes. Men her er de, på samme bord, og diskuterer bærekraft sammen.
\end{quote}

Loddtrekning kan derfor være den metoden der man sikrer størst mangfold, og der denne viktige ressursen brukes i deliberasjonen for å løse komplekse problemstillinger.

I definisjonen kan man merke seg at det står «(nær) tilfeldig trukket ut gruppe mennesker». Ikke alle borgerpaneler gjennomfører et helt tilfeldig utvalg, men bruker heller det som kalles stratifisert utvalg. Med et stratifisert utvalg mener man at befolkningen inndeles i grupper (strata), og at det plukkes ut deltakere fra hver gruppe. Dette kalles en tostegs loddtrekning. Det som da skjer, er at man først sender ut et visst antall invitasjoner til helt tilfeldig personer. De som da får invitasjonen, svarer så om de vil delta eller ikke. Blant de som takker ja, gjør man så en ny loddtrekning etter å ha inndelt dem i grupper (kjønn, alder, geografi, utdanning osv.) for å sikre at borgerpanelet, så godt det lar seg gjøre, utgjør en miniatyr av befolkningen.

En begrunnelse for å foreta et såkalt stratifisert utvalg er antallet personer som deltar i noen av prosessene. Noen borgerpaneler har så få som 16 deltakere, og hadde man valgt ut 16 personer helt tilfeldig gjennom en ren loddtrekning, kunne man ha endt opp med et panel bestående av 15 menn og 1 kvinne. Derfor brukes disse kategoriene i loddtrekningen for å sørge for at panelet, så godt det lar seg gjøre, vil være en miniatyr av samfunnet det representerer.

Det er imidlertid ikke bare de små borgerpanelene som bruker stratifisert utvalg. Også de største typene av borgerpaneler, som deliberative meningsmålinger, tar noen gang i bruk stratifisert utvalg (\protect\hyperlink{ref-fishkin_deliberative_2018}{James S. Fishkin et al. 2018}). Grunnen til dette er at den politiske deltakelsen er skeiv og ofte følger utdanningsnivå og sosial-økonomisk klasse (\protect\hyperlink{ref-farrell_sortition_2019}{David M. Farrell and Stone 2019}). Det kan altså være mange forhold som forhindrer folk i å delta, spesielt med tanke på tid, ressurser, interesse osv. Dersom man ikke er obs på dette, kan man ende opp med at større deltakelse kan øke overrepresentasjonen av dem som allerede er godt representert, og dette «kan skape et paradoks om at det å øke innbyggernes muligheter for deltakelse kan øke den politiske ulikheten» (\protect\hyperlink{ref-warren_citizen_2008}{Warren 2008, 56}). I en loddtrekning kan man nå ut til dem som kanskje ikke deltar i særlig grad ellers, og siden de blir personlig invitert, vil de kunne føle seg beæret over å ha blitt plukket ut, noe som kan øke motivasjonen for å delta. Siden man vet på forhånd hvem som er plukket ut, og har mulighet til å prate med dem på forhånd, har man større mulighet for å kunne tilrettelegge for at de kan delta, med tanke på barnepass, reise til og fra osv. Ved loddtrekning kan man dermed, i hvert fall i teorien, minimere barrierene mot deltakelse. Det er imidlertid viktig å påpeke at barrierene ikke forsvinner med loddtrekning. Selv om man blir plukket ut til å delta i et borgerpanel, er det stadig frivillig om man vil delta eller ikke. Det er dermed en form for selvseleksjon, noe som kan svekke argumentet om at man med loddtrekning kanskje når ut til de som vanligvis ikke deltar. Stratifisert utvalg brukes derfor som et annet middel for å få en enda bedre utvelgelse, selv om ikke dette heller er uproblematisk.

Det finnes også en annen grunn til å bruke stratifisering. I enkelte saker kan det være gode grunner for å sørge for at noen absolutt blir representert i panelet, og kanskje til og med overrepresentert i forhold til befolkningen generelt. Innenfor borgerpanelene i Canada er det for eksempel sterke tradisjoner for å sørge for å ha med personer som identifiserer seg selv som urfolk, og de er også ofte overrepresentert. Når det gjelder enkelte spørsmål, ser vi også at det finnes gode grunner for å sørge for at enkelte er (over)representert i panelene. I noen borgerpaneler om klimaspørsmålet i Storbritannia ble for eksempel visse grupper som var hardest rammet av klimakonsekvensene, overrepresentert. Dersom man søker etter en statistisk representativ gruppe, kan man ende opp med at marginaliserte grupper blir en minoritet i selve borgerpanelet, noe som kan være problematisk (\protect\hyperlink{ref-mansbridge_should_1999}{Mansbridge 1999}), spesielt siden det ofte er marginaliserte grupper som blir hardest rammet av visse beslutninger. Derfor kan det være gode argumenter for å bevege seg bort fra å lage en eksakt miniatyr av samfunnet som skal representeres, siden de forskjellene som er i samfunnet, kan bli gjenspeilet i borgerpanelene. For mer om hvordan de forskjellige borgerpanelene gjennomfører sitt lotteri, og hvem de representerer, se Ohren (\protect\hyperlink{ref-ohren_representative_nodate}{n.d.}).

\hypertarget{ulike-typer-borgerpaneler}{%
\subsubsection{Ulike typer borgerpaneler}\label{ulike-typer-borgerpaneler}}

Her skal vi se nærmere på et utvalg av teknikker, der ambisjonen er å forene bredde- og dybdehensynet, altså slike teknikker som er skissert i fjerde kategori i tabell \ref{tab:tbl-borgerpaneler}. Hvordan er ordningene designet, og hvordan søker de å løse dilemmaet mellom bred, representativ deltakelse på den ene siden og kravet om samtale og deliberasjon på den andre siden? Nærmere bestemt gir vi en kort presentasjon av henholdsvis borgerjury, planleggingscelle, lekfolkskonferanse, borgersamlinger og deliberative meningsmålinger. I forbindelse med presentasjonen av planleggingsceller gir vi også en kort fremstilling av charrette og deltakende budsjettering.

\hypertarget{borgerjury}{%
\paragraph{Borgerjury}\label{borgerjury}}

Vi starter med den desidert mest brukte formen for borgerpanel, nemlig det vi kaller borgerjury. Av alle typene borgerpanel som har blitt gjennomført, er det denne modellen som har blitt brukt mest. Dette er også tilfellet i Norge.

Modellen ble utviklet av Ned Crosby i USA på 1970-tallet og er som det engelske navnet tilsier, inspirert av juryordningen. I denne modellen inviteres alt fra 12 til 50 mennesker. Igjen, så er dette en for liten gruppe til å oppnå noen form for statistisk representativitet, men som tidligere forklart, bruker man diverse stratifiseringskategorier for å speile befolkningen som den er plukket ut fra, best mulig. De som blir trukket ut, samles da i 2--5 dager, enten i ett strekk eller fordelt over en lengre periode.

Denne modellen følger de fasene som ble beskrevet tidligere, ganske tett. Først går man gjennom en kunnskaps- og læringsfase, så har man en refleksjon- og diskusjonsfase, før man sammenfatter dette og konkluderer i en rapport.

I en borgerjury er det fokus på at det skal lages et helhetlig dokument, i form av en borgerrapport. Den inneholder ofte anbefalinger om hva medlemmene av borgerjuryen mener er det beste for samfunnet som helhet. Man strever ikke nødvendigvis etter konsensus blant deltakerne, men fokuset er å skrive en rapport som de fleste kan stille seg bak, og at det også finnes rom for å skrive motsvar fra de som ikke er enige med flertallet i panelet. Noen ganger stemmer man også over diverse forslag. Mer vesentlig argumenter denne rapporten for hvorfor borgerjuryen har kommet frem til det de har. Dette er essensielt, siden det man vil med en borgerjury (og de metodene som brukes for å skrive en rapport), er at det skal inspirere til deliberasjon også hos de som ikke deltar. De som ikke deltar i borgerpanelet, kan lese rapporten og danne seg sine egne meninger ut fra faktagrunnlaget, informasjonen og argumentasjonen fra borgerjuryen. Ofte ønsker man også at de som har bestilt borgerjuryen, gir svar på hvorfor de velger å følge opp, eventuelt ikke å følge opp, forslaget. En slik prosess kan derfor bidra til en større diskusjon i samfunnet som helhet, og det er også poenget med de borgerpanelene som lager en rapport til sist, som borgerjuryer.

Borgerjuryer har som sagt blitt brukt ofte, og de har dermed blitt brukt i mange sammenhenger og mange ulike saker. Ofte knyttes de også opp mot større prosesser og andre former for medvirkning. Et godt eksempel på hvordan slike borgerpaneler knyttes opp mot andre prosesser, er Oregon Citizens Initiative Review.

I Norge er det per i dag blitt gjennomført 8 borgerjuryer, dog med litt ulike benevninger. Dette inkluderer en borgerjury om problemet med overvann, en knyttet til et områdeløft, en knyttet til samfunnsdelen av en kommuneplan, en knyttet til revisjon av en småhusplan og en med et medarbeiderpanel i en bedrift. Noen av disse vil bli diskutert nærmere i kapittel \ref{sec:andre}.

\hypertarget{planleggingscelle}{%
\paragraph{Planleggingscelle}\label{planleggingscelle}}

Planleggingscellen og borgerjuryen ble utviklet mer eller mindre parallelt, men uavhengig av hverandre. Den første planleggingscellen ble avviklet i Tyskland i 1972. Samlet sett engasjeres langt flere deltakere gjennom planleggingscellene enn gjennom borgerjuryer. Normalt vil en prosess involvere 150--200 deltakere, inndelt i celler på 25 deltakere. Men i enkelte tilfeller har så mange som 600 deltakere vært involvert i én og samme prosess, fordelt på 24 planleggingsceller. Et større antall deltakere vil øke muligheten for proporsjonal representativitet, det vil si at sammensetningen av deltakerne i planleggingscellen gjenspeiler sammensetningen i befolkningen.

Sammenlignet med borgerjuryen har planleggingscellen vært klarere innrettet mot fysisk planlegging. Planleggingscellen skiller seg også fra borgerjuryen ved at det legges større vekt på arbeid i smågrupper. Videre opererer planleggingscellene ikke som samlede juryer. De enkelte deltakerne roterer mellom smågruppene, slik at det ikke skal kunne feste seg en sosial struktur innad i gruppen.

Sluttresultatet av prosessen er en rapport som ikke bare gir en oversikt over de synspunktene som er kommet frem, men der deltakernes holdninger også blir kvantifisert, slik at det er mulig å trekke konklusjoner om mønstre i meningsdannelsen. Sammenlignet med borgerjuryen er planleggingscellen tettere knyttet til offentlig beslutningstaking. Det offentlige finansierer planleggingscellene, og beslutningstakere for det offentlige har på forhånd forpliktet seg til å ta hensyn til rådene fra deltakerne, selv om de ikke forplikter seg til å følge dem.

Planleggingscellen har en del fellestrekk med det som kalles «charrette». I Norge er det blant annet gjennomført charretter i Oslo i forbindelse med planlegging av Majorstulokket i 2004 og utbyggingen av Bjørvika i 2008. Charrette er betegnelsen på en prosess som involverer en serie folkemøter om et bestemt tema, og den kan vare i opptil en uke. Prosessen munner ut i en plan for det aktuelle arealet. I de norske eksemplene har kommunen vært initiativtaker. Men til forskjell fra planleggingscellen og borgerjuryen har rekrutteringen til charrette foregått gjennom åpen invitasjon, det vil si selvutvelgelse. En charrette er derfor en samtalebasert deltakelsesarena, men den er ikke basert på normen om representativitet.

Det finnes også en del likhetstrekk mellom planleggingscelle og ordningen med deltakende budsjettering, som først ble utviklet i den brasilianske byen Porto Alegre. I Norge er det gjennomført få prosjekter med deltakende budsjettering, men det er gjort forsøk med slike prosesser i Fredrikstad kommune. Budsjetteringsprosessen tar utgangspunkt i bydelsvise allmøter, der innbyggerne diskuterer problemer og finner frem til mulige løsninger. Møtedeltakerne drøfter seg frem til hvilke oppgaver som skal prioriteres. På allmøtene velges det bydelsforsamlinger, og i denne fasen knyttes arbeidet opp mot tilsvarende prosesser i andre bydeler. I den første fasen selvrekrutteres deltakerne i budsjetteringsprosessene, mens de i neste fase velges, det vil si at deltakelsen er indirekte. Som med charrette kan man si at deltakende budsjettering gir rom for meningsbrytning, men deltakerne er ikke valgt ut med tanke på sosial og demografisk representativitet. Like fullt har det vist seg at budsjettprosessene har aktivisert mange deltakere som tidligere har vært passive. Kommuneansatte har oppsøkt bydeler og oppfordret innbyggerne til å delta. I tillegg har det vist seg at fattige bydeler er kommet bedre ut av den deltakende budsjetteringsprosessen. Prosessen har med andre ord hatt en omfordelende effekt.

\hypertarget{lekfolkskonferanser}{%
\paragraph{Lekfolkskonferanser}\label{lekfolkskonferanser}}

På 1980-tallet utviklet det danske teknologirådet sin egen modell, kalt lekfolkskonferanse. Denne modellen ble utviklet for å deliberere over kontroversielle vitenskapelige og/eller teknologiske utviklinger (\protect\hyperlink{ref-ryan_defining_2014}{Ryan and Smith 2014}), men den har tatt for seg andre typer temaer også. Denne modellen ble utviklet med tanke på å oppnå to mål: Det som kom ut av deliberasjonen, skulle gi beslutningstakerne en bedre forståelse av den sosiale konteksten til fremvoksende teknologier, og prosessen skulle også stimulere til informert offentlig debatt om teknologiproblemer (\protect\hyperlink{ref-1885-53297}{Hendriks 2005}).

Som man ser ut fra tabell \ref{tab:tbl-opprinnelse}, har en lekfolkskonferanse 10--25 tilfeldig uttrukne borgere som deltar, og prosessen varer alt fra 3 til 8 dager. På en lekfolkskonferanse er det ikke fokus på å få til et representativt utvalg av befolkningen, men derimot på å sikre et så stort mangfold som mulig. Derfor bruker de stratifisert utvelgelse, men i stedet for å velge proporsjonalt fra befolkningen, er det mer fokus på å oppnå spredning. Borgerne må sende inn svar på om de ønsker å delta eller ikke, og hvis de svarer ja, må de som regel skrive et brev om seg selv. Dette brevet danner også et grunnlag for utvelgelsen, og noen ganger velges det personer som har tatt uvanlige valg i livet, eller som har et spesielt forhold til tematikken som blir diskutert.

Prosessen i seg selv kan vi si er delt inn i to faser. Den første fasen kalles den forberedende fasen. I denne fasen blir det fokus på å lære om tematikken til konferansen. Det som derimot er noe annerledes her sammenlignet med de andre modellene, er at man ikke bare skal få kunnskap og lære om tematikken, men man skal også diskutere seg frem til hvilken type spørsmål lekfolkskonferansen skal ta for seg. På en lekfolkskonferanse vil det da ikke nødvendigvis være et konkret spørsmål som stilles til deltakerne fra starten av, men heller en bred tematikk. Tematikken kan for eksempel være «genteknologi», og gjennom den forberedende fasen skal de få kunnskap og lære om denne tematikken, for så å finne spørsmålet som de selv skal ta med videre inn i andre fase. Slik sett setter lekfolkskonferansen agendaen selv i mye større grad enn i noen av de andre typene borgerpanel.

I den andre fasen samles deltakerne igjen. Eksperter og øvrige personer presenterer saken i et offentlig forum, og etterpå blir de blir utspurt av deltakerne på lekfolkskonferansen. Deretter trekker deltakerne seg tilbake for å drøfte det som har blitt sagt, og om noe må utdypes. I den siste fasen skal deltakerne bli enige om en rapport. En av tingene som skiller lekfolkskonferanser fra de andre modellene, er at man prøver å oppnå en konsensus. Det engelske navnet gir uttrykk for dette: Consensus Conferences. Derfor er resultatet av konferansen ofte en konsensuserklæring. Det er imidlertid viktig at det ikke er tvang for å oppnå konsensus, men at deltakerne blir «oppmuntret til å utforske hvor langt de kan følge hverandres argumenter» (\protect\hyperlink{ref-fixdal_consensus_1997}{Fixdal 1997, 370}). Dersom gruppen ikke kommer til enighet, vil det bli laget en splittet erklæring.

I Norge har vi hatt fire lekfolkskonferanser. Bruken av disse var også mye av grunnen til at det ble opprettet et norsk teknologiråd.
I de senere år har bruken av lekfolkskonferanser vært noe dalende. De ble nok mest brukt på 1990-tallet og begynnelsen av 2000-tallet. Dette har nok sammenheng med rollen til det danske teknologirådet, som endret seg noe på den tiden. Modellen er imidlertid tatt i bruk igjen, og man kommer nok til å se en økning av denne typen borgerpanel i perioden fremover, noe som kan ses i sammenheng med den økende interessen rundt borgerpanel og demokratiske innovasjoner generelt.

\hypertarget{borgersamlinger}{%
\paragraph{Borgersamlinger}\label{borgersamlinger}}

Både i faglitteraturen og i praksis blir borgersamlingsmodellen nå sett på som den mest robuste og forseggjorte modellen for borgerpanel (\protect\hyperlink{ref-elstub_mini-publics_2014}{Elstub 2014}). Som \protect\hyperlink{ref-fournier_henk_van_der_kolk_when_2011}{Fournier Henk Van Der Kolk et al.} (\protect\hyperlink{ref-fournier_henk_van_der_kolk_when_2011}{2011}) skriver, er dette den eneste modellen som klarer å kombinere at den har en stor gruppe med vanlige mennesker, en lang periode med læring og deliberasjon og en kollektiv beslutning som har stor betydning for hele det politiske systemet. Dette underbygges av trenden med at det etableres egne begreper knyttet til denne modellen (som «Climate Assemblies»), og vi har sett at det har blitt gjennomført store nasjonale borgersamlinger i Irland, Danmark, Storbritannia, Frankrike, Tyskland, Canada, Spania og Ungarn, for å nevne noen. Man har også sett dette på overstatlig nivå, for eksempel i EU, og det kanskje mest ambisiøse eksemplet var den såkalte Global Climate Assembly, som var den eneste demokratiske medvirkningsprosessen på FNs klimakonferanse i 2021 (COP26).

Dette er store prosesser, der man gjerne inviterer 100--150 mennesker som skal møtes over flere helger, og i enkelte tilfeller over flere år. De tar ofte for seg komplekse saker, som grunnlovsendringer, klimapolitikk og så videre. Ofte er disse prosessene knyttet opp mot andre demokratiske innovasjoner, og da spesielt folkeavstemninger. Den mest kjente borgersamlingen er nok den irske borgersamlingen i 2016. Der ble 99 mennesker tilfeldig plukket ut gjennom en loddtrekning, og de ble speilet på kjønn, alder og geografi. Borgersamlingen i Irland satt i to år og tok for seg en rekke spørsmål, alt fra klimasaker til abortspørsmålet.

Selve prosessen i en borgersamling er ikke ulik den som vi ser hos borgerpaneler. Også her har man ofte de samme fasene, med kunnskap og læring, refleksjon og diskusjon og beslutning til slutt. De er derimot ofte mye mer grundige, og de har i tillegg en ekstra fase. Dette er også den eneste modellen som har inkorporert en fase med offentlige høringer i prosessen (\protect\hyperlink{ref-curato_deliberative_2021}{Curato et al. 2021}). Dette kan gjøres gjennom tradisjonelle offentlige høringer eller gjennom en nettportal der man kan sende inn sine innspill, som så blir tatt opp i borgersamlingen. På den måten er det kanskje en mye sterkere kobling mellom borgersamlingen og samfunnet generelt i denne prosessen enn det er i andre borgerpaneler.

I Norge har vi ikke hatt noen borgersamling enda. Det nærmeste vi har kommet, er kanskje «Trondheimspanelet», som hadde noen elementer av borgersamling, spesielt da med høringsprosessen. Men dette vil fortsatt være definert som et borgerpanel.

Borgersamlinger er store prosesser som er svært kostbare og ekstremt krevende. Dette har nok sammenheng med at man trekker ut en større gruppe deltakere enn i de andre prosessene som er diskutert her, samtidig som de må de oppfylle de samme kravene til prosessen i seg selv. Dette er nok grunnen til at vi i den siste tiden hovedsakelig har sett disse på nasjonalt nivå.

\hypertarget{deliberative-meningsmuxe5linger}{%
\paragraph{Deliberative meningsmålinger}\label{deliberative-meningsmuxe5linger}}

Deliberative meningsmålinger skiller seg fra de ovenstående gruppene av teknikker ved at de tar utgangspunkt i meningsmålinger som en slags modell, mens borgerjuryer og planleggingsceller heller representerer en videreføring av ulike former for gruppebeslutninger. «Deliberativ» vil si at det gis anledning til å kommunisere med andre og tenke seg om før man gjør seg opp en mening.

Når folk svarer på opinionsundersøkelser, får de liten anledning til å tenke gjennom spørsmålene de besvarer. For de fleste ville det også ha vært naturlig å diskutere spørsmålene med andre før de tar endelig stilling. I en meningsmåling er det derimot et mål at intervjuobjektene ikke skal utsettes for påvirkning fra andre. Kritikerne av meningsmålinger hevder at svarene på opinionsmålinger har liten verdi, og at de bare gir uttrykk for overfladiske betraktninger hos de spurte. Da den deliberative meningsmålingen ble utviklet, var ambisjonen å fysisk samle et tilfeldig valgt utvalg av befolkningen og gi dem anledning til å drøfte de spørsmålene de skulle ta stilling til. I praksis har dette betydd at mellom 100 og 500 deltakere har blitt rekruttert. Endelig ønsket man å undersøke hvordan deltakelse på en slik samling påvirket holdningene til deltakerne.

Deltakerne har besvart et identisk spørreskjema både i forkant og i etterkant av samlingen. Ved å foreta målinger på ulike tidspunkter er det mulig å etterspore hvilken effekt det har hatt på deltakerne å være med på samlingen. Det store antallet deltakere gjør det utfordrende å gjennomføre større deler av samlingen i plenum. Den deliberative meningsmålingen er derfor hovedsakelig basert på arbeid i smågrupper.

Sluttresultater fra deliberative meningsmålinger er formidlet og brukt på ulike måter. For det første betraktes den samlede opinionen som kommer til uttrykk blant deltakerne etter høringen, som mer verdifull for beslutningstakerne enn den opinionen som kommer til uttrykk gjennom tradisjonelle opinionsundersøkelser. Grunnen til dette er at deltakerne presumptivt gir uttrykk for mer reflekterte holdninger etter at de har deltatt på en deliberativ meningsmåling. For det andre er det i seg selv verdifullt å se hvordan holdningene hos deltakerne har beveget seg i løpet av samlingen. Forsvarerne av deliberative meningsmålinger hevder at dette forteller noe om hvordan et hvilket som helst annet tilfeldig valgt utvalg ville ha endret holdninger dersom de hadde hatt anledning til å delta på et tilsvarende arrangement.

Koblingen til formelle beslutningsarenaer har vært ulik, men i hovedsak benyttes resultatene fra deliberative meningsmålinger som innspill i en bredere politisk debatt. Derfor har formidlingen av samlingene vært viktig. For eksempel har utdrag fra flere av de deliberative meningsmålingene vært fjernsynsoverført.

Siden den første deliberative meningsmålingen i Storbritannia i 1994 er det gjennomført tilsvarende arrangementer i en rekke land, deriblant Australia, Danmark, Finland, Island og USA. I Norge er det gjennomført to deliberative meningsmålinger, og disse omtales i denne håndboken. I tillegg er det gjennomført tre såkalte «utvidede folkehøringer» på lokalt nivå i Norge, men disse manglet enkelte av elementene som kreves for å være kvalifisert til den formelle (og beskyttede) betegnelsen «Deliberative Poll».

Erfaringene med deliberative meningsmålinger tilsier at det er krevende å rekruttere et fullstendig representativt utvalg av deltakere. Uten en svært målrettet rekruttering risikerer man at utvalget får mange av de samme bakgrunnstrekkene som folkevalgte forsamlinger. Imidlertid viser erfaringene også at en stor andel av deltakerne ikke hadde tidligere politisk erfaring, og for mange var det første gang de sto foran en større forsamling og holdt et politisk innlegg. Det har også skjedd en bevegelse i deltakernes holdninger som følge av at de har deltatt, men holdningsendringene skjer ikke alltid i form av en samlet opinionsendring, det vil si at mange skifter mening, men ikke nødvendigvis i samme retning.

\newpage

\hypertarget{del-ii-praktisk-gjennomfuxf8ring-av-borgerpanel}{%
\section{DEL II: PRAKTISK GJENNOMFØRING AV BORGERPANEL}\label{del-ii-praktisk-gjennomfuxf8ring-av-borgerpanel}}

\includegraphics{figs/1x/Del2.png}

Denne delen trekker veksler på tre ulike borgerpaneler som vi selv har arrangert. Borgerpanelene har variert i omfang, tematikk og møteform. Vi bruker dem vekselvis som eksempler, og derfor vil vi først gi en kort beskrivelse av de tre arrangementene før vi går mer i detalj om den praktiske gjennomføringen av et borgerpanel.

\newpage

\hypertarget{tre-eksempler}{%
\subsection{Tre eksempler}\label{tre-eksempler}}

\hypertarget{byborgerpanelet}{%
\subsubsection{Byborgerpanelet}\label{byborgerpanelet}}

\begin{quote}
Borgerpanel om bydelsnivå i Bergen
\end{quote}

Dette borgerpanelet referer vi til som \emph{Byborgerpanelet}. Byborgerpanelet ble en realitet som følge av arbeidet til et byrådsoppnevnt utvalg med mandat til å se på nærdemokratiet i Bergen (\protect\hyperlink{ref-fimreitebyen}{Fimreite 2018}). Beskrivelsen er delvis hentet fra \protect\hyperlink{ref-arnesenloddet}{Arnesen, Fimreite, and Aars} (\protect\hyperlink{ref-arnesenloddet}{2021}), hvor man også kan finne mer utfyllende informasjon om Byborgerpanelet.

Utvalget, som utførte sitt arbeid våren 2017, var organisatorisk uavhengig av kommunen og sammensatt av forskere, tidligere lokalpolitikere, mediefolk og representanter for frivilligheten i byen -- både den organiserte og den uorganiserte.

Et panel basert på tilfeldig uttrekk blant kommunens innbyggere var ett av utvalgets forslag for å øke lokal deltakelse og medvirkning. Forslaget vakte interesse i byens politiske miljø, og i september 2017 fattet bystyret vedtak om et forsøk med innbyggerpanel. Bystyret bestemte samtidig at panelet skulle gi konkrete innspill i saken om innføring av politiske bydelsstyrer i kommunen fra 2020 (byrådssak 245/17 og bystyresak 231/17). Bergen kommune har siden 1970-årene hatt ulike former for bydelsorganisering -- både politisk og administrativt. Fra 2011 ble ordningen med politiske bydelsutvalg avskaffet. I samarbeidsplattformen til partiene som utgjorde byrådet i perioden 2015--2019, sto det imidlertid at gjeninnføring av slike utvalg skulle vurderes. Derfor var bydeler også et viktig tema i Nærdemokratiutvalgets arbeid (\protect\hyperlink{ref-fimreitebyen}{Fimreite 2018}).

Byborgerpanelet ble organisert som et samarbeidsprosjekt mellom Bergen kommune, Universitetet i Bergen og NORCE og ble gjennomført i slutten av april 2018. Med utgangspunkt i byrådets forslag til videre arbeid med reformen (Byrådssak 245/17) ble fire dimensjoner ved bydelsorganisering presentert for panelet. Det gjaldt hvor mange bydeler skal kommunen ha; hvilke oppgaver bydelene skal ta seg av; i hvor stor grad skal bydelene selv bestemme hva pengene de får overført skal brukes til; hvordan representantene til bydelsstyrene skal rekrutteres.

\hypertarget{deliberativ-meningsmuxe5ling-i-bergen-2021}{%
\subsubsection{Deliberativ meningsmåling i Bergen 2021}\label{deliberativ-meningsmuxe5ling-i-bergen-2021}}

\begin{quote}
Borgerpanel om utvikling av havneområde i Bergen og om etablering av bilfrie soner.
\end{quote}

Beskrivelsen av dette borgerpanelet er hentet fra rapporten «Deliberativ meningsmåling i Bergen», hvor mer utfyllende informasjon om prosjektet foreligger (\protect\hyperlink{ref-arnesendelib2021}{Arnesen, Fimreite, and Skiple 2021}). En deliberativ meningsmåling er en variant av borgerpanel, som omtalt tidligere.

Arrangementet ble gjennomført som et ledd i det forskningsrådsfinansierte prosjektet Demokratisk innovasjon i praksis: Forskning på medvirkning og legitimitet i kommunale beslutningsprosesser (Demovate) (prosjektnummer 295892). Formålet med prosjektet var å utvide verktøykassen for lokaldemokratiet i kommunen ved å ta i bruk nye metoder for involvering av innbyggerne i politikken. Prosjektansvarlig var Bergen kommune, og faglig ble prosjektet ledet av forskningsinstitusjonen NORCE. Samarbeidspartnere var Universitetet i Bergen og Stanford University.

Deliberativ meningsmåling i Bergen 2021 ble initiert på bakgrunn av vedtak i Bergen bystyre i møtet 24.10.2018, sak 235/18. Byrådet behandlet videre saken i møtet 29.04.2021, sak 1135/21, og fattet vedtak om at programmet for bilfrie soner i bydelene og arealstrategi for Dokken skulle brukes som diskusjonstemaer i borgerpanelet.

Sakene som skulle inngå i undersøkelsen, samt i den deliberative meningsmålingen, ble bestemt av Bergen kommune i samråd med forskerne. Det ble utarbeidet et omfattende informasjonsmateriale om de to sakene som ble sendt til paneldeltakerne i forkant av arrangementet. I dette materialet ble det gitt informasjon om hva saken dreide seg om. Det ble videre lagt frem konkrete forslag som skulle være diskusjonstemaer i den deliberative meningsmålingen. For disse forslagene ble det presentert for- og motargumenter. Arrangementet skulle i utgangspunktet gjennomføres fysisk, men covid 19-pandemien satte en effektiv stopper for disse planene. Den deliberative meningsmålingen ble derfor gjennomført digitalt i juni 2021, hvor 84 personer fullførte deltakelsen via videokonferanseplattformen Zoom.

Byrådet forpliktet seg til å ta stilling til innspillene fra borgerpanelet i de politiske saksfremleggene som følger sakene, hvor innspillene skulle være selvstendige vedlegg i saksgrunnlaget frem mot endelig politisk behandling.

\hypertarget{deliberativ-meningsmuxe5ling-i-norge-2022}{%
\subsubsection{Deliberativ meningsmåling i Norge 2022}\label{deliberativ-meningsmuxe5ling-i-norge-2022}}

\begin{quote}
Borgerpanel om bruk av kunstig intelligens i forvaltningen og om teknologier for å redusere klimagassutslipp
\end{quote}

Sommeren 2022 ble det gjennomført en digital deliberativ meningsmåling med deltakere fra hele landet. Arrangementet var i regi av forskningsinstituttet NORCE, i samarbeid med Stanford University i USA. 218 deltakere fra hele landet samlet seg online for å diskutere enten

\begin{enumerate}
\def\labelenumi{\arabic{enumi})}
\tightlist
\item
  rettferdig bruk av kunstig intelligens i forvaltningen, eller\\
\item
  teknikker for å redusere karbondioksid i luften.
\end{enumerate}

Det første temaet var del av det forskningsrådsfinansierte prosjektet \emph{Public Fairness Perceptions of Algorithmic Governance} (prosjektnr. 314411), mens det andre var del av det EU-finansierte prosjektet \emph{Ocean-based Negative Emission Technologies -- analyzing the feasibility, risks, and co-benefits for stabilizing the climate (OceanNETs)}.

Borgerpanelet ble gjennomført ved hjelp av en plattform utviklet av Stanford University, som var skreddersydd for formålet. Det var første gang at denne plattformen ble tatt i bruk i Europa.

Deltakerne ble rekruttert ved tilfeldig uttrekk fra Folkeregisteret. I første omgang fikk meningsmålingsbyrået Ipsos oppdraget med å ringe til personer på listen og rekruttere deltakere. Senere ble innbyggere som ikke hadde gitt respons, kontaktet via Digipost.

\newpage

\hypertarget{valg-av-sakstyper}{%
\subsection{Valg av sakstyper}\label{valg-av-sakstyper}}

\begin{quote}
Anne Lise Fimreite
\end{quote}

\includegraphics{figs/1x/spm.png}

\hypertarget{prinsipp-og-refleksjon}{%
\subsubsection{Prinsipp og refleksjon}\label{prinsipp-og-refleksjon}}

Valg av spørsmål og saker som behandles av borgerpaneler, har vist seg å være svært viktig for hvordan panelene vurderes av innbyggerne --- ikke bare av de som deltar, men også av de som ikke deltar. Hvilke saker som behandles, er også avgjørende for hvordan det politiske systemet forholder seg til panelene og tar imot råd og anbefalinger (og i noen tilfeller beslutninger) fra dem. For at panelene skal ha det litteraturen omtaler som god kvalitet, må meningsutvekslingen og diskusjonen mellom deltakerne fungere, men i tillegg må også relasjonen mellom panelene og det politiske systemet fungere. Det betyr at systemet --- uavhengig av om det er et kommunestyre, et parlament eller EU-systemet --- må ta panelenes råd og anbefalinger på alvor og innlemme dem i den videre politiske behandlingen. Hvis det ikke skjer, vil panelenes legitimitet som demokratisk innovative arenaer fort tape terreng. Hvilke saker og saksfelt panelene tar opp, kan ha stor betydning i så måte.

Ulike spørsmål og sakstyper har vært behandlet i borgerpaneler verden over. Noen paneler har, som omtalt i del 1, vært innrettet mot store overordnede sakskomplekser som endringer i valgordninger og grunnlovsformuleringer (f.eks. i Canada, Irland og Island). Andre har vært innrettet mot overordnede forhold, men på mer avgrensede saksfelter som velferds- og familiepolitikk (f.eks. i Irland og Skottland). Mange paneler har vært innrettet mot helt konkrete saker som bygging av veier og broer, hvilke trasevalg som skal tas, og hvilke omfang utbyggingen skal ha. Saker av mer teknisk karakter har også vært tatt opp, bl.a. av et panel i Belgia som behandlet regionale klimatilpasningstiltak. Noen paneler har videre tatt opp nedleggelse eller flytting av offentlige tilbud som sykehus og skoler. Budsjettforslag og offentlige planer har også vært behandlet av slike paneler. I tillegg til slike relativt konkrete saker finnes det også eksempler på at paneler har tatt opp utforming av offentlig styringsstruktur, både på lokalt og på regionalt nivå. Spørsmål om etableringen av et nytt regionalt folkevalgt nivå i England var gjenstand for diskusjon i borgerpanelet i 2015. I tillegg finnes det også et eksempel på at EU i 2018 ga et stort deliberativt panel i oppgave å utforme et spørreskjema som skulle avdekke hva som kjennetegner en «europeisk borger».

Det er altså stor variasjon i hvilke saker og sakstyper som behandles av disse panelene. Pragmatisme, og til en viss grad politiske behov, har ofte vært mer førende for valgene enn klare prinsipper. En kritikk som gjerne rettes mot denne typen panel, er at den korte tiden panelene (tross alt) har til rådighet, gjør at komplekse saker blir overflatisk behandlet. Uansett opplegg for panelene har ikke deltakerne nok tid, nok informasjon eller nok kunnskap til å sette seg inn i sakene på en fullgod måte sammenlignet med permanente politiske organer. Disse organene har dessuten ekspertstøtte til å opplyse om og utrede sakene de behandler. Et retningsgivende prinsipp for valg av saker kan derfor være graden av kompleksitet. Et annet prinsipp som trekkes frem, er at saker som har et klart utfallsrom, er å foretrekke. Det kan f.eks. være saker som går ut på om en vei skal bygges eller ikke, om et flyktningmottak skal etableres eller ikke, om et konkret klimatilpasningstiltak skal innføres eller ikke, eller om en skole skal legges ned eller ikke. Her kan prinsippene helt klart stå mot hverandre. Det er også opplagt at også saker med klare utfall kan være svært komplekse å ta standpunkt til. Dermed er ikke kompleksitet alene avgjørende for om et panel fungerer godt eller ikke. Andre forhold spiller også inn.

Det har likevel vist seg å være en tendens til at saker som har et klart utfall, på den måten at man kan ta et standpunkt for eller imot, lettere får gjennomslag i det politiske systemet etter behandling i borgerpaneler enn saker av mer vurderende karakter. Det betyr ikke at politikerne nødvendigvis følger panelenes anbefalinger i de nevnte sakene, men at det er større sjanse for at panelets innspill blir innlemmet i den videre politiske sakshåndteringen. Samtidig finnes det eksempler på at anbefalinger gjort av paneler knyttet til utforming av lokal styringsstruktur er blitt tatt videre i det politiske systemet, både her hjemme og i utlandet, blant annet i det tidligere nevnte eksemplet om å innføre et regionalt folkevalgt nivå i England.

En faktor som kan ha betydning for hvilke saker og sakstyper panelene skal behandle, er når i en saksprosess de blir involvert. Tidlig i saksprosessen kan panelene behandle spørsmål som har med formulering av et saksfelt å gjøre, mens det senere i prosessen kan være snakk om å komme med mer konkrete anbefalinger og råd i retning av hvordan saker bør løses. Uansett tidspunkt i saksprosessen er det avgjørende om panelene representerer en reell påvirkningsmulighet eller kun fungerer som sandpåstrøere (se mer i kapitlet \ref{sec:etter} om oppfølging etter gjennomføring).

Bruk av paneler er gjerne styrt av et opplevd behov for (større) kontakt mellom det politiske systemet og innbyggerne. Slike behov kan være knyttet til kartlegging av synspunkter hos befolkningen, enten som helhet eller blant deler av den, til tilbakemeldinger i vanskelige saker og til reell meningsutveksling mellom ulike synspunkter for å kunne fremme så gode anbefalinger som mulig. Men litteraturen viser også at paneler har vært brukt til forsvar for beslutninger som allerede er fattet, eller for å legitimere disse. Innledningsvis i dette delkapitlet nevnte vi at kvaliteten på et panel er avhengig av to forhold:

\begin{itemize}
\tightlist
\item
  meningsutvekslingen internt i panelene, og\\
\item
  relasjonen mellom panelene og det politiske systemet.
\end{itemize}

Kvaliteten kan knyttes til panelenes legitimitet -\/-- både overfor de som deltar, overfor de innbyggerne som panelene representerer, og overfor det politiske systemet. Som tidligere nevnt er det viktig for kvaliteten at det politiske systemet faktisk tar sakene videre, men det er også viktig at panelene oppfattes som reelle, på den måten at tilbakemeldingene deres blir lyttet til, og at de ikke kun brukes for å bekrefte eller avkrefte allerede fattede politiske vedtak. Saker og sakstyper er en viktig side ved dette.

\hypertarget{vuxe5re-erfaringer}{%
\subsubsection{Våre erfaringer}\label{vuxe5re-erfaringer}}

Byborgerpanelet 2018 dreide seg om en sakstype som det gjerne \emph{ikke} anbefales å ta inn i et borgerpanel, nemlig organisatoriske forhold. Panelet tok opp gjeninnføring av bydeler i kommunens styringsstruktur. Saken hadde ikke hatt noe særlig offentlig oppmerksomhet, og det var derfor ikke gitt at de som deltok i panelet, hadde noen form for kjennskap til den. Utenom en kort omtale om at det var denne saken som skulle tas opp, ble det ikke gitt noe informasjon om sakskomplekset i invitasjonsbrevet. All informasjon ble gitt ved innledningen til selve panelet. Informasjonen var utarbeidet i samarbeid mellom representanter for Bergen kommune og forskerne. Det ble utformet et manus som ble lest opp til deltakerne, der fire dimensjoner ved bydeler ble presentert.

Med utgangspunkt i byrådets forslag til videre arbeid med reformen (Byrådssak 245/17) skulle deltakerne ta stilling til
- Hvor mange bydeler skal kommunen ha?\\
- Hvilke oppgaver skal bydelene ta seg av?\\
- I hvor stor grad skal bydelene selv bestemme hva pengene de får overført, skal brukes til?\\
- Hvordan skal representantene til bydelsstyrene rekrutteres?

I tråd med erfaringene fra paneler i andre land som har behandlet organisatoriske og strukturelle forhold, var redegjørelsen for dimensjonene gjort nokså konkret. Det ble også understreket overfor deltakerne at situasjonen de skulle forholde seg til, var at bydelsstyrer skulle gjeninnføres, og at det de skulle ta stilling til, var hvordan de da ønsket at dette skulle organiseres. Panelets tema var ikke for eller imot bydeler, men organiseringen av bydeler i kommunens styringsstruktur.

Erfaringen fra Byborgerpanelet 2018 er at saken/sakstypen fungerte i dette forumet. Deltakerne tok standpunkt til undersøkelsene og debatterte i gruppesesjonen. Det var også mulig å utarbeide en anbefaling til kommunen (byrådet) om hvordan fremtidig bydelsorganisering skulle være. Ved avslutningen av panelet informerte representanter fra Bergen kommune om hvordan anbefalingene fra Byborgerpanelet ville bli fulgt opp i den videre politiske prosessen.

Sakene som skulle inngå i den deliberative meningsmålingen i Bergen 2021, ble bestemt av Bergen kommune i samråd med forskerne. Det ble pekt ut to saker:

\begin{itemize}
\tightlist
\item
  fornying av havneområdet Dokken i Bergen sentrum, og\\
\item
  utvikling av bilfrie områder i Bergen utenom sentrum.
\end{itemize}

Begge sakene må kunne klassifiseres som planleggingssaker, og begge sakene dreier seg om arealdisponering --- hva arealet skal brukes til, og hvem som skal bestemme dette. Sakene har relativt ulike tidshorisonter. Dokken-utbyggingen er helt i startfasen, og utbyggingen antas ikke å være ferdigstilt før om 30 år. Konseptet med bilfrie soner er på gang i deler av kommunen, og den politiske bestillingen om å utvide dette til andre deler av kommunen enn sentrumssoner er gjort. Bindende beslutninger i denne saken er derfor ikke like langt unna som i Dokken-utbyggingen.

Det ble utarbeidet et omfattende informasjonsmateriale om de to sakene i samarbeid mellom de aktuelle avdelingene i Bergen kommune og forskerne. Informasjonsmaterialet ble sendt til paneldeltakerne ca. én uke før selve arrangementet. I dette materialet ble det gitt nøktern informasjon om hva sakene dreide seg om. På bakgrunn av informasjonen ble det videre lagt frem konkrete forslag til spørsmål som skulle være diskusjonstemaer i den deliberative meningsmålingen. For forslagene ble det presentert både for- og motargumenter. Det var sju forslag knyttet til Dokken-utbyggingen og ni knyttet til spørsmålet om bilfrie soner. Heller ikke dette panelet skulle ta stilling til om Dokken-utbyggingen skulle realiseres, eller om bilfrie soner skulle iverksettes. Inngangen var gitt at utbyggingen finner sted, og gitt at bilfrie soner etableres, hvordan skal de da utformes?

I brevet som fulgte med informasjonsmaterialet, ble det videre sagt noe om hvordan innspillene fra den deliberative meningsmålingen vil bli behandlet i det kommunale systemet. Byrådet i Bergen fattet i april 2021 et helt konkret vedtak om dette, som det ble referert til i nevnte brev.

Erfaringene fra denne digitale deliberative meningsmålingen i Bergen er at begge sakene fungerte i dette forumet. Det er likevel en tendens til at det fungerte litt bedre å diskutere og komme med anbefalinger i saken om Dokken-utbyggingen enn i saken om bilfrie soner. Forskjellene er ikke store, og sakstypene er nokså like i og med at begge dreier seg om arealutnytting i fremtiden. Dokken-utbyggingen er et større problemfelt, som nærmest dreier seg om å bygge en helt ny bydel, og om hva innholdet i den skal være. Her er det et langtidsperspektiv som opplagt gjør at saken blir mindre konkret, men samtidig spennende og utfordrende. Deltakerne hadde også mange og viktige spørsmål til ekspertpanelet som møtte dem i løpet av diskusjonen.

Bilfrie soner dreide seg om en utvidelse av et konsept som allerede er i ferd med å bli innført i sentrumsnære områder. Denne saken er derfor langt mer konkret og nærmere i tid. Likevel var det en tendens til at engasjementet i diskusjonen var mer moderat her, selv om deltakerne også i denne saken hadde flere og relevante spørsmål til ekspertene. En opplagt forklaring kan være at begge sakene er omfattende og komplekse, og at det å innlemme to slike saker i et avgrenset panel på én og samme dag ble i overkant for mange av deltakerne. Det at panelet var digitalt, forsterket dette bildet av «overload».

Isolert sett fungerte sakstypen --- arealdisponering i fremtiden --- imidlertid godt for diskusjon i panelene, spesielt med et «løfte» fra det politiske systemet om å ta innspillene deres på alvor. Lærdommen må være å konsentrere panelet rundt én sak og heller vie denne litt mer tid, blant annet til å gå mer i dybden i samtalene mellom paneldeltakerne og ekspertene.

I motsetning til de to foregående eksemplene ble den deliberative meningsmålingen i Norge i 2022 gjennomført av forskere uten samarbeid med noen kommune. Spørsmålene ble valgt ut fra hvilke forskningsspørsmål de to prosjektene ønsket svar på, og var ikke direkte koblet til noen politisk prosess. Like fullt er det grunner til å anta at både holdninger til bruk av kunstig intelligens i forvaltningen og teknikker for å redusere karbondioksid i luften vil være av interesse for politiske beslutningstakere.

De to nevnte temaene er relativt kompliserte og dessuten lite diskutert i befolkningen generelt. Dersom man bare gjennomfører en vanlig spørreundersøkelse, risikerer man at folk gir «placebo-svar», altså bare svarer et eller annet uten egentlig å ha reflektert nok over temaene til å kunne ha en informert oppfatning. Deliberative meningsmålinger egner seg godt til slike typer temaer, da det gir vanlige innbyggere anledning til å bruke tid til å tenke seg om og diskutere med andre. Fageksperter på henholdsvis kunstig intelligens og negative utslippsteknologier bidro også med oppklarende refleksjoner, som deltakerne kunne velge å lytte til.

De foreløpige resultatene viser at deltakerne selv mener at de vet mer om temaene enn de faktisk gjør. Når det gjelder holdninger til bruk av kunstig intelligens, endret de oppfatning og ble mer positive til det. Holdningene til teknologier for karbonfangst og -lagring endret seg ikke, selv ikke etter at de hadde tilegnet seg mer kunnskap om temaet. Å skape holdningsendring er imidlertid ikke et mål i seg selv på slike arrangementer, målet er å skape visshet om at holdningene er informerte. Det lyktes dette borgerpanelet med.

\newpage

\hypertarget{fysisk-eller-digitalt}{%
\subsection{Fysisk eller digitalt?}\label{fysisk-eller-digitalt}}

\begin{quote}
Sveinung Arnesen
\end{quote}

\includegraphics{figs/1x/Deltakelsesform.png}
Et veivalg ved gjennomføringen av borgerpanelet er om det skal gjøres ved at deltakerne møtes fysisk et sted, eller om det skal foregå digitalt. Dette er viktig fordi det har stor betydning for organisering, kostnader og hvilke data som produseres fra arrangementet. Det finnes klare fordeler og ulemper knyttet til begge løsningene, som vi beskriver i dette kapitlet.

Når man velger enten fysisk eller digitalt borgerpanel, tar man samtidig også valg som har implikasjoner for kostnadene ved prosjektet, for deltakelse, for data og for hvilken kompetanse prosjektgruppen trenger å ha eller skaffe seg. Grovt sett kan man si at sammenlignet med et fysisk arrangement har et digitalt arrangement (vesentlig) lavere kostnader, lavere deltakelsesrate, bedre datafangst og høyere kompetansekrav til arrangøren. Det er imidlertid mulig å navigere innenfor dette området og redusere betydningen av veivalget digitalt vs.~fysisk på de nevnte faktorene. For eksempel kan den økonomiske kompensasjonen for deltakere på et digitalt arrangement økes, noe som vil øke deltakelsesraten og kostnadene ved arrangementet.

Før covid 19-pandemien var det helt klart mest vanlig å gjennomføre borgerpaneler ved fysisk oppmøte. Under pandemien ble man tvunget til å tenke annerledes, og digitale løsninger presset seg frem. Pandemien markerer slik sett et tidsskille. Den har synliggjort at fordelene ved digitale arrangementer er store, og til dels fått samfunnet til å omstille seg, slik at ulempene er mindre enn før.

\hypertarget{prinsippermuxe5l-som-veileder-valgene}{%
\subsubsection{Prinsipper/mål som veileder valgene}\label{prinsippermuxe5l-som-veileder-valgene}}

\hypertarget{kvalitet-puxe5-diskusjonene}{%
\paragraph{Kvalitet på diskusjonene}\label{kvalitet-puxe5-diskusjonene}}

Fysiske møter er den møteformen som har blitt studert mest, og som vi vet mest om når det gjelder beste praksis. Et typisk fysisk opplegg vil innebære at deltakerne deles inn i mindre grupper på 5--15 personer. Gruppene får helst egne rom, hvor de kan snakke uforstyrret fra andre grupper. Dersom dette er vanskelig å løse rent praktisk, kan gruppene eventuelt deles inn i egne soner i større rom.

Det er liten tvil om at et fysisk gruppemøte oppleves som mest naturlig for oss mennesker. Man kan snakke direkte til de andre i gruppen, oppfatte reaksjonene deres og la diskusjonen flyte naturlig mellom deltakerne. En gruppemoderator ser til at diskusjonen foregår på sømmelig vis, at gruppen følger agendaen, og at alle deltakerne får sjansen til å bli hørt.

Digitale møter legger noen bånd på gruppedynamikken, i den forstand at deltakerne må be om ordet, si det de har på hjertet, og så overlate ordet til noen andre. Gruppene er satt sammen av folk som ikke kjenner hverandre fra før, og det kan ta lengre tid å bryte isen på nettet enn når man er i samme fysiske rom. Det er vanskelig å fange opp ikke-verbale reaksjoner hos de andre gruppemedlemmene, og dynamikken i diskusjonen kan også lide av tekniske utfordringer med dårlig lyd og/eller bilde.

Det er altså gode grunner for å hevde at kvaliteten på selve gruppediskusjonen jevnt over er høyere ved fysiske gruppemøter enn ved digitale møter. Likevel er ikke dette helt svart/hvitt, for det finnes digitale plattformer som legger til rette for overraskende gode gruppediskusjoner. En slik plattform er den såkalte Stanford Online Deliberation Platform (SODP).

SODP ligner i sitt uttrykk på andre videokonferanseplattformer som Zoom eller Microsoft Teams -- personer kommer sammen i digitale grupper, hvor de kan se og høre hverandre via skjermen. Men til forskjell fra andre plattformer har SODP i tillegg skreddersydde løsninger for å gjennomføre en deliberativ meningsmåling. For eksempel har den en automatisk moderator som geleider deltakerne gjennom diskusjonen. Moderatoren passer på at deltakerne får like mye snakketid. Den gir enkelte dult til deltakerne, slik som å oppfordre dem til å ta ordet hvis det er lenge siden forrige gang de snakket. Den passer også på å fordele tiden på alle undertemaene, slik at gruppene kommer gjennom alle punktene på sin tilmålte tid. SODP oversettes til deltakernes språk, inkludert all tekst- og lydveiledning som benyttes for å lede deltakerne gjennom de ulike stegene i prosedyren knyttet til den deliberative meningsmålingen.

Denne plattformen er per i dag ikke allment tilgjengelig for bruk, da den fortsatt befinner seg i forsknings- og utviklingsfasen. De mest tilgjengelige digitale alternativene er Zoom og Microsoft Teams, og også disse er det fullt mulig å bruke som plattform i et digitalt borgerpanel. En fordel med disse plattformene er at mange er kjent med hvordan de fungerer, og at de teknisk fungerer godt sammen med det aller meste av utstyr deltakerne måtte ha hjemme. Såkalte \emph{break out rooms} gir mulighet for å dele deltakerne inn i passelig store grupper.

Etter pandemien har innbyggerne generelt blitt mer komfortable med å gjennomføre møter på nettet, både med tanke på det tekniske og det sosiale. Det er også grunn til å forvente at digitale løsninger blir både bedre og mer utbredt i fremtiden, og at de etter hvert vil tette gapet til fysiske møter. Inntil videre holder vi imidlertid en knapp på fysiske møter når det gjelder kvalitet på gruppediskusjonene.

\hypertarget{kostnader}{%
\paragraph{Kostnader}\label{kostnader}}

Den aller største fordelen med å gjennomføre et digitalt arrangement er de reduserte kostnadene. Det er nemlig vesentlige besparelser å hente på ikke å måtte betale reise, kost og losji for deltakerne. Leie av arrangementssteder med grupperom og saler for fellesøkter faller også bort. Det er ingen lik formel som kan brukes på alle arrangementer for å beregne hvor store besparelser som kan gjøres. Momenter som påvirker kostnadene, er blant annet hvor lang reisevei deltakerne har, om man som arrangør har tilgang på gratis eller billige lokaler som kan brukes, hvilke avtaler man kan få i stand for kost og losji, varighet på arrangementet og antall deltakere. Rekrutteringskostnadene (kostnader ved å få tak i folk og eventuelt kompensere dem økonomisk for å delta) vil ikke påvirkes nevneverdig av hvorvidt arrangementet er fysisk eller digitalt.

\hypertarget{deltakelse}{%
\paragraph{Deltakelse}\label{deltakelse}}

Den digitale møteformen hever noen terskler og senker andre. Samlet sett mener vi det er risiko for lavere deltakelse i digitale borgerpaneler enn i sine fysiske motstykker.

Digitale møter krever en viss teknisk kompetanse hos deltakerne. De må ha tilgang på en stabil internettilkobling og kunne håndtere en PC eller et nettbrett. En stadig større gruppe av befolkningen i Norge håndterer dette teknisk bra, men det finnes fortsatt systematiske forskjeller med tanke på hvem som blir utelatt fordi de ikke har de tekniske mulighetene eller ferdighetene som trengs.

For andre er det enklere å delta digitalt enn fysisk. Man slipper å transportere seg selv til et bestemt sted, og dermed sparer man tid og penger. Man kan også gjennomføre arrangementet i komfortable omgivelser som man velger selv, for de fleste vil det si i sitt eget hjem. Om man har lite tid til rådighet, har vansker med å transportere seg til bestemmelsesstedet eller rett og slett er mindre glad i fysiske gruppetilstelninger, er digitale arrangementer å foretrekke.

En tommelfingerregel er at halvparten av de påmeldte faktisk deltar på borgerpanelet. Dette tallet varierer imidlertid stort og påvirkes av en rekke faktorer som handler både om egenskaper ved arrangementet så vel som egenskaper ved deltakerne. En frykt er at frafallet blant de påmeldte er større ved et digitalt arrangement enn ved et fysisk arrangement, ettersom fysisk deltakelse krever mer dedikasjon, slik at påmelding til et slikt opplegg oppleves som mer bindende i utgangspunktet.

\hypertarget{kompetansekrav-hos-arranguxf8ren}{%
\paragraph{Kompetansekrav hos arrangøren}\label{kompetansekrav-hos-arranguxf8ren}}

Det kreves en velorganisert arrangør for å dra i gang et borgerpanel, enten det gjennomføres fysisk eller digitalt. Digitale arrangementer stiller noen ekstra krav til arrangøren som det er verdt å gjøre oppmerksom på. Den digitale plattformen må beherskes, slik at dagen forløper uten større tekniske problemer. De første minuttene i oppstarten av arrangementet er kritiske både med tanke på at det tekniske med plattformen skal fungere, og med tanke på at deltakerne klarer å komme seg inn på den uten problemer. Brukerfeil må påregnes, spesielt ved oppstart, og man må ha IT-support som kan hjelpe de som har problemer med mikrofon, høyttalere, kamera osv. Det er ikke på IT-siden at det skal spares på ressursene, da store tekniske utfordringer på selve arrangementsdagen kan velte hele prosjektet.

Det kreves også at arrangøren har kontroll på deltakernes rett til personvern. Ved et digitalt arrangement er det flere hensyn å ta i så måte. Arrangøren har såkalt databehandlingsansvar og må se til at personopplysninger og sensitive data som kan knyttes til deltakerne, blir behandlet i henhold til forskriftene. Sensitive data er blant annet informasjon om deltakernes politiske holdninger, religiøse overbevisning osv. Et digitalt arrangement vil potensielt kunne spre denne typen informasjon enten mens arrangementet pågår, eller i etterkant hvis lyd og bilde blir lagret. Den databehandleransvarlige må sørge for at personlig identifiserbar informasjon ikke kommer på avveie.

Noen vil ha tilgang til et personvernombud i sin egen organisasjon, som vil kunne svare på denne typen spørsmål, mens andre er ikke like godt rustet.

\hypertarget{data}{%
\paragraph{Data}\label{data}}

En stor fordel med digitale borgerpaneler er at de legger til rette for opptak av gruppediskusjoner. Slike data er nyttige fordi de i ettertid kan analyseres og gi presise tilbakemeldinger på hvilke holdninger som kommer til uttrykk blant deltakerne i løpet av et slikt arrangement. Denne informasjonen er nyttig for å finne ut hva deltakerne mener, og kan også brukes til å studere gruppedynamikk, holdningsendringer og andre atferdsmessige forskningsspørsmål. Så fremt personvernmessige hensyn er avklart og ivaretatt, er dette rike data som kan vise seg å bli svært nyttige for arrangøren.

\hypertarget{vuxe5re-erfaringer-1}{%
\subsubsection{Våre erfaringer}\label{vuxe5re-erfaringer-1}}

Byborgerpanelet 2018 ble gjennomført fysisk, mens Deliberativ meningsmåling i Bergen 2021 og Deliberativ meningsmåling i Norge 2022 var digitale. I det fysiske borgerpanelet satt deltakerne i mindre grupper uten gruppeledere til stede, og derfor er det ikke mulig å vurdere kvaliteten på gruppediskusjonene. De digitale gruppediskusjonene kunne vi følge både i sanntid og i ettertid på opptak. I 2021 hadde vi gruppeledere som sørget for at deltakerne holdt seg til temaet og fulgte agendaen, mens i 2022 var gruppelederne erstattet med en automatisk moderator. Vi ble positivt overrasket over hvor godt diskusjonene fungerte med den automatiske moderatoren. Enkelte grupper fungerte bedre enn andre, men generelt var de fleste delaktige og ga uttrykk for sine meninger. Opptakene fra diskusjonene blir slettet, men før det blir de transkribert og anonymisert, slik at diskusjonene er tilgjengelige for allmennheten i skriftlig form. Dette gjør det mulig å studere i detalj hva som diskuteres, og hvordan gruppediskusjonene fungerer. Strengt tatt er dette mulig å gjøre også med fysiske borgerpaneler, men det er mer teknisk krevende og ikke vanlig å gjøre i praksis. Vår erfaring er derfor at digitale arrangementer gjør det enklere å evaluere kvaliteten på diskusjonene i etterkant.

Kostnadsbesparelsene ved å arrangere den deliberative meningsmålingen i Norge 2022 var store. Vi trengte ikke å betale deltakerne for reise, kost og losji, noe som ville ha vært en stor utgiftspost. I dette tilfellet ville en fysisk gjennomføring ha innebåret å fly inn flesteparten av de over 200 deltakerne fra hele landet, som ville ha hatt behov for overnatting og mat i 1--2 dager. Et grovt anslag er at de totale kostnadene var en femtedel av hva de ville ha vært for et tilsvarende fysisk borgerpanel. Til sammenligning var det små eller ingen besparelser i 2021. Alle deltakerne holdt til i Bergen og ville hatt kort reisevei og heller ikke hatt behov for overnatting. Vi trengte ikke å leie fysiske lokaler eller å tilby lunsj, men måtte ha IT-personell i forkant av arrangementet og på selve dagen til å sørge for den tekniske gjennomføringen.

Deltakelsen på de digitale arrangementene var vesentlig lavere enn på det fysiske. Av de inviterte til det fysiske arrangementet var det nærmere 20 prosent som fullførte deltakelsen, mens det for de digitale arrangementene i 2021 og 2022 var henholdsvis 5 prosent og 1 prosent som fullførte. Når det er sagt, er det ikke mulig å slå fast uten videre at det var det digitale formatet som trakk ned deltakelsen på de digitale arrangementene. Til det var det for mange andre ting som varierte. Det fysiske arrangementet ga også deltakerne dobbelt så høy økonomisk kompensasjon (2000 kroner vs.~1000 kroner), og det varte også vesentlig kortere (2 timer vs.~henholdsvis 6 og 5 timer). Det trengs mer forskning -- helst randomiserte kontrollerte studier -- før vi kan isolere effekten av fysisk vs.~digitalt format på deltakelsen.

Personvernhensyn skulle vise seg å bli kimen til mange utfordringer for de digitale arrangementene. I 2020 kom den såkalte Schrems II-dommen i EU, som la sterke begrensninger på overføring av personopplysninger ut av EU/EØS-området. (\^{} Omtalen av personvernutfordringene er hentet fra rapporten «Deliberativ meningsmåling i Bergen» \protect\hyperlink{ref-arnesendelib2021}{Arnesen, Fimreite, and Skiple 2021}.) Den deliberative meningsmålingen i Bergen 2021 var opprinnelig ment å gjennomføres fysisk sommeren 2020, men ble på grunn av covid 19-pandemien utsatt ett år. I løpet av senhøsten 2020 ble det klart at denne samlingen ikke kunne planlegges som fysisk arrangement i 2021 heller. Arbeidet med alternative måter å avholde innbyggerpanel på ble intensivert. Nettbaserte løsninger ble her -- som på så mange andre felt -- det åpenbare alternativet.

Våre samarbeidspartnere ved Stanford University i California --- som har merkevarepatent på deliberativ meningsmåling --- har utviklet en nettbasert løsning for en slik meningsmåling kalt Stanford Online Deliberation Platform (SODP). Plattformen ligner i sitt uttrykk på Zoom eller Microsoft Teams, men den har i tillegg mer skreddersydde funksjoner for gjennomføring av meningsmålingen. SODP var vårt førstevalg for å arrangere en nettbasert meningsmåling i regi av DEMOVATE-prosjektet i Bergen.

Forskningsinstituttet NORCE sitt personvernombud vurderte at SODP var en løsning som kunne velges, gitt de spesielle forholdene pandemien gav, men Bergen kommunens personvernombud vurderte dette annerledes. Vurderingen var basert på at løsningen ikke ivaretok deltakernes personvern godt nok, blant annet fordi dataene som ble samlet inn (innloggingsdata, innholdsdata og tekniske data), ville ha blitt overført til og oppbevart på en server i USA. Dette ble vurdert å være i strid med EUs personvernforordning (GDPR).

SODP-løsningen måtte dermed forkastes i denne omgang, og valget falt på å gjennomføre arrangementet på nettet ved bruk av videokonferanseplattformen Zoom. Basert på blant annet nærmere ett års erfaring med undervisning på Zoom anbefalte personvernombudet hos NORCE denne løsningen som datamessig sikker. Oppsettet av meningsmålingen via Zoom-plattformen og håndteringen av de innsamlede dataene i ettertid ble teknisk og sikkerhetsmessig håndtert av NORCE sin IT-avdeling. Dataene er lagret på sikre servere ved hhv. UiB (SAFE) og UiO (TSD). IT-avdelingen sto også for den tekniske støtten i forbindelse med gjennomføringen av panelet 12. juni 2021.

Beslutning om å velge denne løsningen ble tatt relativt sent i prosessen, gitt at arrangementet var satt til juni 2021. For å lette ansvarsstrukturen knyttet til behandling av personopplysninger overtok NORCE også de delene av arbeidet med å rekruttere deltakere som Bergen kommune var tiltenkt å gjennomføre.

Den deliberative meningsmålingen i Norge 2022 året etter var et rent forskningsprosjekt, og ansvarsstrukturen for behandling av personopplysninger var dermed enklere å forholde seg til. Etter en personvernkonsekvensutredning gjennomført av Sikt -- kunnskapssektorens tjenesteleverandør -- kunne vi som de første i Europa benytte oss av SODP-løsningen.

Nå som Schrems II har fått tid å virke, vil de fleste kommuner og organisasjoner være bedre forberedt på problemstillingen om overføring av data ut av EU/EØS. Våre erfaringer eksemplifiserer imidlertid hvilken type utfordringer som kan oppstå i forbindelse med personvern, og at det er viktig å være tidlig ute med å avklare problemstillinger knyttet til lagring og overføring av personopplysninger. Politiske holdninger regnes som sensitive data, og spesielt med det digitale formatet er det mange skjær i sjøen som må kartlegges før man legger ut på seilas.

Vår erfaring er at det er god hjelp for forskerne å få hos Sikt. Offentlige institusjoner har gjerne en personvernansvarlig som kan hjelpe til med å navigere i dette farvannet. Det er viktig å inkludere personvernansvarlige tidlig i prosessen for å sikre at personvernhensynene blir ivaretatt.

\newpage

\hypertarget{rekruttering}{%
\subsection{Rekruttering}\label{rekruttering}}

\begin{quote}
Jon Kåre Skiple
\end{quote}

\includegraphics{figs/1x/Rekruttering.png}

\hypertarget{hvorfor-er-rekruttering-av-deltakere-viktig}{%
\subsubsection{Hvorfor er rekruttering av deltakere viktig}\label{hvorfor-er-rekruttering-av-deltakere-viktig}}

Rekruttering av deltakere er nøkkelen for å utløse det demokratiske potensialet til borgerpanel. Grunnpilarene til borgerpaneler er nemlig demokratisk likhet, alle innbyggere skal ha lik mulighet til å delta, og representativitet, det deliberative beslutningsorganet skal speile befolkningen som helhet (\protect\hyperlink{ref-flanigan_fair_2021}{Flanigan et al. 2021}). Representativitet blir i denne sammenhengen forstått som deskriptiv representativitet (\protect\hyperlink{ref-fishkin_democracy_2018}{James S. Fishkin 2018b}), altså deskriptive karakteristikker ved befolkningen, som kjønn, alder, bosted og utdanning. Representasjon i borgerpaneler er viktig fordi det er antatt å gi en variasjonsbredde i meningsutvekslingen som reflekterer holdningene i befolkningen som helhet (\protect\hyperlink{ref-fishkin_democracy_2018}{James S. Fishkin 2018b}).
Hvordan man rekrutterer deltakere, er avgjørende for i hvilken grad man evner å oppnå idealene om demokratisk likhet og representativitet. Den rekrutteringsformen som er antatt å gi best resultat, er tilfeldig uttrekk fra populasjonen, fordi den gir alle innbyggere lik mulighet til å delta, uavhengig av hvem de er. Samtidig kan den reelle muligheten til å delta i borgerpaneler være ulikt fordelt blant befolkningen, og lik mulighet til å delta er ikke er ensbetydende med å oppnå representativitet. På samme måte som at noen grupper i befolkningen er mindre tilbøyelige til å benytte seg av stemmeretten sin ved politiske valg, er det sannsynligvis systematiske skjevheter i hvem som velger å delta i borgerpaneler. Demokratisk likhet og representativitet er med andre ord idealer man etterstreber, men som man ikke nødvendigvis oppnår.
I prosessen med å rekruttere deltakere til borgerpaneler står man som arrangør ovenfor mange små og store valg som kan få betydning for hvor mange deltakere man klarer å rekruttere, i hvilken grad man oppnår representasjon og demokratisk likhet, kostnadene ved gjennomføring og hvilken legitimitet borgerpanelet får i befolkningen og blant politikerne. Ofte medfører valgene at man er nødt til å gjøre avveininger mellom faktorer knyttet til kostnader og den praktiske gjennomføringen på den ene siden og representativitet, demokratisk likhet og legitimitet på den andre siden.
I dette kapitlet beskriver vi de valgene man står ovenfor som arrangør av borgerpanel i forbindelse med rekruttering av deltakere. Vi diskuterer hvordan ulike valg påvirker antallet deltakere, økonomiske kostnader, gjennomføring, demokratisk likhet, representativitet og legitimitet. Basert på en forskningsundersøkelse gjennomført i Bergen kommune viser vi hvordan valg knyttet til rekruttering kan påvirke hvorvidt innbyggere velger å delta i borgerpaneler, samt påvirke legitimiteten til borgerpanelet (\protect\hyperlink{ref-arnesenloddet}{Arnesen, Fimreite, and Aars 2021}). Til slutt presenterer og diskuterer vi våre egne erfaringer knyttet til rekruttering av deltakere i forbindelse med gjennomføringen av borgerpanel i Bergen kommune. Målet vårt er at disse innsiktene skal gi fremtidige arrangører av borgerpaneler et godt beslutningsgrunnlag i rekrutteringsarbeidet.

\hypertarget{viktige-steg-og-valg-i-rekrutteringsprosessen}{%
\subsubsection{Viktige steg og valg i rekrutteringsprosessen}\label{viktige-steg-og-valg-i-rekrutteringsprosessen}}

Arbeidet med å rekruttere deltakere til borgerpaneler kan brytes opp i en stegvis prosess. For hvert enkelt steg står man ovenfor valg og avveininger som kan ha viktige konsekvenser ikke bare for idealene om demokratisk likhet og representativitet, men også for den praktiske gjennomføringen av borgerpanelet.

\hypertarget{definere-populasjonen}{%
\paragraph{Definere populasjonen}\label{definere-populasjonen}}

Før man starter arbeidet med å rekruttere deltakere, er det viktig å avklare hva som er rammen til borgerpanelet, fordi denne avgrensningen definerer hvilken populasjon man ønsker å rekruttere fra. Rammen for et borgerpanel vil ofte være ett geografisk definert område, slik som ett land, en region eller en kommune. Det er innenfor den valgte rammen man ønsker at politikkforslagene eller løsningene som utvikles i borgerpanelet, skal være virkningsfulle. I tillegg til den geografiske dimensjonen vil alder som regel være en avgrensing som styrer fra hvilken populasjon man vil rekruttere deltakere. Det er ofte mest aktuelt å rekruttere fra den stemmeberettigede delen av befolkningen.

\hypertarget{sette-muxe5l-og-kartlegge-tilgjengelige-ressurser}{%
\paragraph{Sette mål og kartlegge tilgjengelige ressurser}\label{sette-muxe5l-og-kartlegge-tilgjengelige-ressurser}}

Før man går i gang med å rekruttere deltakere, er det viktig å sette seg mål for rekrutteringen og skaffe seg oversikt over hvilke ressurser man har tilgjengelig i rekrutteringsarbeidet. Med mål sikter vi her til helt konkrete egenskaper ved rekrutteringen, som hvor mange deltakere man ønsker å rekruttere. Med ressurser menes her både den økonomiske og materielle kapasiteten til arrangøren. Hvilke ressurser man har tilgjengelig, setter begrensninger for hvilke mål det er realistisk å oppnå. Ofte vil man måtte justere målene sine underveis basert på faktorer som kostnader ved økonomisk kompensasjon til deltakerne.

Dersom man skal arrangere et fysisk borgerpanel, er arrangementslokalet en viktig begrensning man bør ha i mente under rekrutteringsarbeidet. Å arrangere et borgerpanel stiller store krav til lokalet, fordi man både skal samle en stor gruppe mennesker i plenum og spre de samme menneskene ut i mindre grupper. Den typen offentlige lokaler som er aktuell for å gjennomføre fysiske borgerpaneler, har som regel plassbegrensning og begrenset tilgjengelighet. Siden disse faktorene legger klare føringer for hvor mange deltakere man kan rekruttere, bør man justere målsettingen tilsvarende. Hvis man planlegger å leie lokaler for å gjennomføre borgerpaneler, må en regne med en betydelig økonomisk kostnad.

Nettbasert gjennomføring av borgerpaneler er et høyst reelt alternativ som delvis eller helt fjerner begrensinger knyttet til arrangementslokalet. Man må imidlertid være oppmerksom på at nettbasert gjennomføring kan ha konsekvenser for hvilken representativitet man oppnår. Dersom terskelen for å delta er høyere etter at en har oppnådd en bestemt alder, hvor datakunnskapene er begrenset, kan nettbasert gjennomføring i større grad enn fysisk gjennomføring føre til et skjevt utvalg med begrenset representativitet. Omvendt skjevhet -- underrepresentasjon av yngre -- er derimot mer sannsynlig ved fysisk gjennomføring, slik at det ikke er åpenbart om nettbaserte eller fysiske arrangementer gir best representativitet.

Antallet deltakere i borgerpanelet er helt sentralt for representativitet, variasjon i meningsutveksling, kostnader og legitimitet. Jo færre personer som deltar, jo vanskeligere blir det å oppnå representativitet, variasjon i meningsutveksling og legitimitet. Samtidig er et økt antall deltakere forbundet med høyere kostnader og større krav til gjennomføring. Derfor er det en balansekunst å sette et realistisk mål for hvor mange deltakere en ønsker å rekruttere, noe som kompliseres ytterligere av at man ikke nødvendigvis vet hvor mange i befolkningen som ønsker å delta på slike arrangementer. Erfaringer fra gjennomføring av borgerpaneler i Norge tilsier at man bør forvente lav deltakelsesvillighet i befolkningen. Innenfor rimelige økonomiske rammer kan man kanskje forvente å kunne rekruttere en plass mellom 50 og 300 mennesker.

Som arrangør av et borgerpanel kan man benytte ulike typer virkemiddel for å øke antall deltakere. Et verktøy man vet vil øke deltakelsen, er (større) økonomisk kompensasjon. Jo høyere økonomisk kompensasjon man får ved deltakelse, jo flere deltakere får man, men nøyaktig hvor mange flere deltakere man får per krone, er ikke åpenbart. Dessuten vil det være et negativt forhold mellom den økonomiske kompensasjonen man kan gi hver enkelt deltaker, og antall deltakere -- jo flere deltakere, jo mindre blir den økonomiske kompensasjonen per deltaker. Dette er viktig å ha i mente, ikke bare med tanke på å rekrutteringen til borgerpanelet, men også for å for å få deltakerne til å gjennomføre borgerpanelet. Samtidig er det viktig å påpeke at målet om å skaffe flere deltakere ikke er den eneste grunnen til å gi deltakerne økonomisk kompensasjon. Det er tidkrevende å delta i borgerpaneler, og ut fra et rettferdighetsperspektiv kan det argumenteres for at det er på sin plass med en passende økonomisk kompensasjon.

Det er nødvendig å sette en økonomisk ramme for arrangementet som inkluderer avsatte midler til deltakerkompensasjon. Usikkerheten knyttet til hvor mange deltakere man klarer å rekruttere (svarprosent), kan komplisere budsjettarbeidet i forbindelse med rekruttering til borgerpanelet betydelig.{[}1{]} Det finnes ulike måter å håndtere denne usikkerheten på. Svarprosenten i tidligere gjennomførte borgerpaneler er nyttig som et utgangspunkt, men kan likevel ikke være mer enn nettopp det, fordi svarprosenten sannsynligvis varierer betydelig mellom ulike typer borgerpanel, avhengig av faktorer som hvor borgerpanelet arrangeres, tema og økonomisk kompensasjon. Derfor kan det være formålstjenlig å legge inn en økonomisk buffer for å håndtere en situasjon der svarprosenten blir for høy gitt budsjettrammen, eller å legge inn forbehold om at deltakelse i borgerpanelet (og dermed økonomisk kompensasjon) er betinget av antall påmeldte deltakere, i invitasjonsbrevet. En slik førstemann-til-mølla-tilnærming har imidlertid potensielle slagsider mot prinsippet om lik mulighet til å delta og representativitet.

Økonomisk kompensasjon er ikke det eneste insentivet man kan bruke for å øke antallet deltakere. I invitasjonen til borgerpanelet kan man fremheve fordeler ved deltakelse i borgerpanelet, som potensiell politisk innflytelse og muligheten til å skaffe seg innsikt i viktige samfunnsspørsmål gjennom ekspertinnlegg og diskusjon med medborgere. I hvilken grad slike insentiver øker rekrutteringen, er nok avhengig av temaet for borgerpanelet. Det er gode grunner til å tro at folk er mer opptatt av politisk innflytelse innenfor politikkområder som er viktige for dem. Markedsføring i media er en annen mulighet som kan benyttes for å skape blest rundt borgerpanelet.

Det er ikke gitt at alle som takker ja til å delta i et borgerpanel, faktisk kommer til å delta. For å være en ressurs i borgerpanelet må de nødvendigvis stille opp og fullføre selve arrangementet. Med tanke på rekruttering er deltakerkontakt derfor en viktig budsjettpost. Med deltakerkontakt mener vi all kommunikasjon mellom arrangør og potensielle deltakere, inkludert infrastrukturen som etableres for påmelding og kontakt. Vi diskuterer deltakerkontakt mer utdypende nedenfor.

\hypertarget{velge-rekrutteringsform}{%
\paragraph{Velge rekrutteringsform}\label{velge-rekrutteringsform}}

Med unntak av i helt spesielle tilfeller er det ikke praktisk gjennomførbart eller nødvendig å rekruttere hele populasjonen. I stedet rekrutterer man et vesentlig mindre utvalg av populasjonen. Her står man som arrangør ved et viktig veiskille med tanke på hvordan man skal rekruttere deltakere. I forskningslitteraturen er dette gjerne fremstilt som et valg mellom to rekrutteringsformer (\protect\hyperlink{ref-mao_deciding_2013}{Mao and Adria 2013}; \protect\hyperlink{ref-ryfe_participation_2012}{Ryfe and Stalburg 2012}):

\begin{itemize}
\tightlist
\item
  tilfeldig uttrekk fra populasjonen, og\\
\item
  selvseleksjon.
\end{itemize}

Tilfeldig utvalg viser til en prosess hvor man lar en datamaskin trekke et tilfeldig utvalg av en gitt størrelse fra populasjonen. Prosessen er tilfeldig fordi alle i populasjonen har lik sjanse for å bli trukket. Arrangøren må informere datamaskinen om hvor stort utvalget skal være. Størrelsen på utvalget man trekker ut, påvirker hvor mange deltakere man rekrutterer, og den forventede svarprosenten vil veilede valget om utvalgsstørrelsen. Arrangøren sender deltakelsesinvitasjon til alle i utvalget.

Selvseleksjon viser til en prosess hvor utvalget blir generert ved at arrangøren sender ut en åpen invitasjon, der innbyggerne selv velger å melde seg på. Selvseleksjon gir alle i populasjonen lik sjanse til å delta, men bare under forutsetning av ubegrenset deltakerkapasitet, noe som er usannsynlig i forbindelse med borgerpaneler. Med deltakerbegrensing kan hver enkelt sin sjanse til å kunne delta bli påvirket av faktorer som når man åpner posten eller e-posten sin, og når man har mulighet til å melde seg på.

Ut fra prinsippet om å gi alle i populasjonen lik mulighet til å delta er tilfeldig utvalg en klar vinner. Men selv når tilfeldig utvalg benyttes for å rekruttere deltakere, er ikke «lik mulighet til å delta» ensbetydende med at alle i populasjonen faktisk har lik mulighet til å delta. Faktorer som familie- og arbeidsforhold, økonomi og helse kan åpenbart påvirke muligheten til å delta.

Selv om tilfeldig utvalg i teorien er en overlegen metodisk tilnærming med tanke på representativitet, er det gode grunner til å tro at det ikke er tilfeldig hvem som velger å takke ja til invitasjonen om å delta i borgerpanelet, slik at det i praksis ikke er åpenbart at en vil oppnå høyere representativitet med tilfeldig utvalg (\protect\hyperlink{ref-boulianne_beyond_2018}{Boulianne 2018}). Det finnes utvalgsmetoder som tar sikte på å bøte på manglende representativitet ved å trekke ut flere (eller færre) fra bestemte grupper av befolkningen enn hva et tilfeldig uttrekk vil tilsi. Dette kan enten være basert på forventninger om svarprosent i ulike samfunnsgrupper eller på å vurdere representativiteten fortløpende og justere utvalget etter hvert som deltakere melder seg på. Når en over- og undertrekker fra ulike grupper av befolkningen, er det ikke lenger lik sjanse til å delta blant befolkningen. Man øker representativiteten på bekostning av demokratisk likhet. Forskere tilknyttet organisasjonen The Sortition Foundation har utviklet en utvalgsalgoritme som maksimerer idealene om representativitet og lik sannsynlighet for å delta (\protect\hyperlink{ref-flanigan_fair_2021}{Flanigan et al. 2021}).

Hvilken rekrutteringsform man velger å benytte, kan få konsekvenser for legitimiteten til borgerpaneler. Det er grunn til å tro at andre innbyggere og politikere vil være skeptiske til et borgerpanel dersom det mangler representativitet, og særlig dersom borgerpanelet består av en selektiv gruppe mennesker. Selvseleksjon kan sende signaler om at en gruppe mennesker utøver større politisk innflytelse gjennom borgerpanelet enn hva de ellers ville gjort, noe som oppleves som urettferdig. En hypotese er derfor at tilfeldig utvalg er viktig for å øke legitimiteten til borgerpaneler.

Avgjørelsen om rekrutteringsform har også økonomiske sider. Det er grunn til å tro at kostnadene ved å bruke tilfeldig utvalg som rekrutteringsstrategi er høyere enn ved selvseleksjon. For det første krever tilfeldig utvalg at man kjøper uttrekk fra Folkeregisteret fra en dataleverandør, med mindre man selv har tilgang til Folkeregisteret. For det andre krever tilfeldig utvalg mer arbeid av arrangøren i forbindelse med å sende ut individuelle invitasjoner og pleie kontakten med de potensielle deltakerne.

\hypertarget{rekruttere-deltakere}{%
\paragraph{Rekruttere deltakere}\label{rekruttere-deltakere}}

Som arrangør må man bestemme seg for om man skal bruke eksterne tilbydere for å rekruttere deltakere, eller om man skal rekruttere selv. Eksterne tilbydere har gjerne dataverktøy og infrastruktur tilgjengelig, noe som forenkler rekrutteringsarbeidet. Selv om det er en økonomisk kostnad ved å bruke eksterne tilbydere, er det ikke sikkert at den overstiger kostnaden ved å gjennomføre arbeidet på egen hånd. Hvis man skal gjennomføre rekrutteringen selv, er det en forutsetning at man har kompetanse på databehandling og tilgang til gode systemer for utsending av invitasjoner og registrering av påmeldte deltakere.

Selve invitasjonen til å delta i borgerpanelet vil som regel være de potensielle deltakernes første møte med arrangementet. Invitasjonen spiller dermed en nøkkelrolle i rekrutteringen. Den skal både fange oppmerksomheten til potensielle deltakere, fremstå som seriøs og være informativ med tanke den videre prosessen for å melde seg på arrangementet. Dersom invitasjonen sendes ut til et tilfeldig utvalg av populasjonen, vil den gjerne være personlig, mens dersom man bruker selvseleksjon, kan den rette seg mot alle innbyggere. Det finnes ulike alternativer for å formidle invitasjonen, men for å nå ut til flest mulig og for å øke sannsynligheten for at de potensielle deltakerne vurderer invitasjonen som seriøs, kan det være hensiktsmessig å benytte seg av enten brevpost eller digital e-post gjennom Digipost. Selv om man benytter eksterne tilbydere i rekrutteringsarbeidet, vil invitasjonen som regel utvikles av arrangøren selv, i hvert fall delvis.

Påmeldingen til arrangementet bør utformes med det prinsipp at det skal være enklest mulig for deltakeren. Påmeldingsprosessen involverer som regel også at det etableres en kanal for kommunikasjon mellom deltaker og arrangør, for å formidle spørreundersøkelsen og informasjonsmateriell. Dersom påmeldingsprosessen oppleves som kronglete, risikerer man å miste deltakere. Gjennom kontakten som etableres mellom potensielle deltakere og arrangøren, kan man også forsøke å øke antallet deltakere ved å sende ut påminnelser om invitasjonen til arrangementet. Det er også viktig at arrangøren svarer på spørsmålene de potensielle deltakerne har om arrangementet.

\hypertarget{en-empirisk-studie-om-hvordan-man-kan-uxf8ke-deltakelse-og-legitimitet}{%
\subsubsection{En empirisk studie om hvordan man kan øke deltakelse og legitimitet}\label{en-empirisk-studie-om-hvordan-man-kan-uxf8ke-deltakelse-og-legitimitet}}

Vi gjennomførte en spørreundersøkelse blant innbyggerne i Bergen kommune for å få empirisk kunnskap om hvordan man kan utforme borgerpanelet for å øke antallet deltakere og gi beslutningene til borgerpanelet større legitimitet. Med utforming av borgerpanelet mener vi nøkkelvalg med tanke på rekruttering og gjennomføring av arrangementet. Slike valg kan være praktiske (hvilken dag borgerpanelet arrangeres), økonomiske (kompensasjon), prosessuelle (hvilken rekrutteringsform som benyttes) og substansielle (hvilken politisk sak som er tema).
Spørreundersøkelsen ble utført av Respons Analyse i tidsrommet 16. Desember 2019 til 13. Januar 2020. Intervjuene ble gjennomført per telefon med utgangspunkt i et representativt utvalg av stemmeberettigede innbyggere i Bergen kommune. Til sammen ble 12 405 personer oppringt, og 900 besvarte undersøkelsen. Det gir en svarprosent på 15,4 prosent. Deskriptive karakteristikker ved respondentene indikerer overrepresentasjon av personer med høy utdanning.

Respondentene i undersøkelsen ble presentert en beskrivelse av et fiktivt borgerpanel som varierer langs sju dimensjoner:

\begin{itemize}
\tightlist
\item
  beslutningsgrunnlag\\
\item
  gjennomføringsdag\\
\item
  deltakerantall\\
\item
  økonomisk kompensasjon\\
\item
  rekrutteringsform\\
\item
  offentlig avstemming\\
\item
  sak
\end{itemize}

Innenfor hver av disse sju dimensjonene blir respondentene eksponert for ulike verdier som varierer tilfeldig, slik at vi kan identifisere årsakssammenhenger. Faktorene innenfor dimensjonene ble valgt med utgangspunkt i reelle alternativer for gjennomføring av et planlagt borgerpanel i Bergen kommune. Etter beskrivelsen av borgerpanelet ble om lag halvparten av respondentene spurt hvor sannsynlig det er at de vil delta i borgerpanelet, mens den andre halvparten ble spurt i hvilken grad de mener politikerne bør vektlegge resultatet fra borgerpanelet når de skal fatte sin beslutning. Det siste spørsmålet er inkludert for å måle legitimiteten til borgerpanelet.

Resultatene fra surveyeksperimentet tegner et ganske entydig bilde av hvordan man kan øke deltakelsen i borgerpanel, nemlig ved å øke den økonomiske kompensasjonen. Effektene av økonomisk kompensasjon er betydelige. Sammenlignet med et borgerpanel hvor deltakerne ikke mottar økonomisk kompensasjon, og hvor sannsynligheten for deltakelse er om lag 26 prosent, øker sannsynligheten for deltakelse med om lag 12 prosentpoeng med en kompensasjon på 200 kroner i timen, 15 prosentpoeng med en kompensasjon på 500 kroner i timen og 22 prosentpoeng med en kompensasjon på 1000 kroner i timen. Disse tallene indikerer dermed at en kan oppnå en relativt betydelig økning i sannsynligheten for deltakelse ved å gi en økonomisk kompensasjon på 200 kroner i timen, men at økningen i sannsynligheten for deltakelse deretter avtar med et enda høyere beløp. Det er også viktig å merke seg at ingen økonomisk kompensasjon er den egenskapen ved borgerpanelet som gir den laveste sannsynlighet for deltakelse relativt til det overordnede gjennomsnittet på 38 prosent sannsynlighet for deltakelse. Innbyggere forventer med andre ord kompensasjon for deltakelsen.

Utover økonomisk kompensasjon har man begrenset med virkemidler for å øke deltakelsen. Det verdt å verdt å merke seg at offentlig avstemming til en viss grad kan avskrekke innbyggerne fra å delta. Offentlig avstemming er nok uaktuelt i de fleste borgerpaneler. Dette resultatet er likevel viktig, fordi det kan være en indikasjon på at innbyggere generelt kvier seg for å eksponere sine politiske meninger offentlig (\protect\hyperlink{ref-jacquet_explaining_2017}{Jacquet 2017}). Selv om resultatene gir inntrykk av at arrangøren i liten grad kan påvirke sannsynligheten for deltakelse utover å gi økonomisk kompensasjon, er det viktig å huske på at dette er med utgangspunkt i de dimensjonene og faktorene som er inkludert i vårt fiktive borgerpanel. Det fiktive borgerpanelet som respondenten blir eksponert for, fanger ikke opp alle mulige dimensjoner, og de faktorene som er inkludert, fanger heller ikke opp den variasjonsbredden som finnes innenfor de dimensjonene som er inkludert.

Resultatene fra surveyeksperimentet viser også at valg av rekrutteringsform kan ha betydning for legitimiteten til borgerpanel. Når deltakerne i borgerpanelet er rekruttert ved et tilfeldig utvalg, mener respondentene at politikerne i større grad bør vektlegge resultatet fra borgerpanelet når de skal fatte den endelige beslutningen i saken, sammenlignet med et borgerpanel hvor deltakerne er rekruttert ved åpen påmelding og selvseleksjon. Vi må likevel ikke overdrive effekten av tilfeldig utvalg. Forskjellen mellom tilfeldig utvalg og åpen påmelding er relativt liten når det gjelder legitimitetsoppfatninger.

\hypertarget{vuxe5re-erfaringer-2}{%
\subsubsection{Våre erfaringer}\label{vuxe5re-erfaringer-2}}

Til Byborgerpanelet i Bergen 2018 var bystyrets bestilling at panelet skulle avholdes som en fysisk samling med inntil hundre tilfeldig uttrukne personer over 18 år. For å kunne oppfylle denne bestillingen foretok Folkeregisteret et tilfeldig uttrekk av tusen personer med bostedsadresse i Bergen. Fra denne basen ble 430 invitasjonsbrev sendt ut via posten -- også her ble tilfeldighetsprinsippet lagt til grunn. Med utgangspunkt i utfordringer knyttet til rekruttering rapportert fra utenlandske paneler besluttet kommunen at deltakerne i panelet skulle honoreres som om de var medlemmer i «ordinære politiske utvalg». Honoraret ble satt til 2000 kroner for oppmøte. Dette var den eneste formen for kompensasjon som ble gitt, noe det ble gjort oppmerksom på i invitasjonsbrevet. Påmelding til panelet skjedde enten ved telefonisk henvendelse til kommunen eller via internett. Både telefonnummeret og nettadressen var oppgitt i invitasjonsbrevet. De som ble invitert til Byborgerpanelet, representerte et tverrsnitt av innbyggerne i Bergen, i og med at alle over 18 år hadde like stor sannsynlighet for å bli trukket ut, uavhengig av hvor lenge de har bodd i kommunen, kjønn, alder eller andre forhold.

Tabell \ref{tab:tbl-byborgerpanel} viser at rundt 20 prosent svarte positivt på invitasjonen. Tabellen viser også et relativt lite frafall fra dem som var påmeldt, til dem som møtte opp. 78 deltakere deltok i panelet, en fremmøteprosent på rundt 90. Gjennomsnittlig alder synker med åtte år når vi sammenligner dem som ble invitert, med dem som faktisk deltok. Et annet moment som fremgår av tabell 10.1, er forskjellen i deltakelse mellom kvinner og menn. Kjønnsbalansen er om lag lik blant de inviterte, men gapet mellom menn og kvinner øker til 14 prosentpoeng når vi ser på dem som faktisk deltok, noe som er stikk motsatt av det vi som oftest ser i utenlandske paneler. Også i norske besluttende organer på lokalt plan, som kommunestyrer og formannskap, er kvinner underrepresentert. Når det gjelder valgdeltakelsen, er det derimot nå kun i de eldre aldersgruppene at menn deltar i større grad enn kvinner.

Deltakelse i Byborgerpanelet i Bergen i 2018 er basert på tilfeldig utvalg, og muligheten for selvseleksjon er derfor mindre enn i deltakelseskanaler der innbyggerne inviteres åpent, for eksempel folkemøter. Til tross for en slik rekrutteringsmekanisme peker de to variablene som er kjent i dette materialet, kjønn og alder, mot en skjevhet blant dem som valgte å delta i panelet, sammenlignet med populasjonen. Vel å merke er forskjellen for kjønn ikke statistisk signifikant, noe vi tilskriver det forholdsvis lave antallet deltakere. Aldersforskjellen mellom dem som ble invitert, og dem som deltok, er derimot statistisk signifikant.

\begin{table}[!h]

\caption{\label{tab:tbl-byborgerpanel}Inviterte og deltakende i Byborgerpanelet 2018}
\centering
\resizebox{\linewidth}{!}{
\begin{tabular}[t]{llll}
\toprule
\textbf{Variabel} & \textbf{Invitert} & \textbf{Påmeldt} & \textbf{Deltok}\\
\midrule
\cellcolor{gray!6}{Antall} & \cellcolor{gray!6}{433} & \cellcolor{gray!6}{87} & \cellcolor{gray!6}{78}\\
 
Andel av inviterte & 100\% & 20\% & 18\%\\
 
\cellcolor{gray!6}{Kjønnsbalanse} & \cellcolor{gray!6}{49\% menn, 51\% kvinner} & \cellcolor{gray!6}{42\% menn, 58\% kvinner} & \cellcolor{gray!6}{43\% menn, 57\% kvinner}\\
 
Gjennomsnittlig alder & 48 & 40 & 40\\
\bottomrule
\multicolumn{4}{l}{\rule{0pt}{1em}\textsuperscript{1} Tabell hentet fra Arnesen, Fimreite og Aars (2021).}\\
\end{tabular}}
\end{table}

Til den digitale deliberative meningsmålingen i 2021 rekrutterte vi deltakere i samarbeid med Bergen kommune. Populasjonen for det aktuelle borgerpanelet var innbyggere i Bergen kommune over 18 år. Borgerpanelet ble gjennomført som en deliberativ meningsmåling hvor man i tillegg til deltakere til selve borgerpanelet også rekrutterer deltakere til en kontrollgruppe. Både deltakerne i borgerpanelet og i kontrollgruppen svarer på en spørreundersøkelse før og etter arrangementet, men deltakerne i kontrollgruppen deltar ikke på arrangementet. Dette forskningsdesignet stiller større krav til rekrutteringsprosessen sammenlignet med et scenario hvor man bare rekrutterer deltakere til et borgerpanel, blant annet fordi man trenger flere deltakere, og fordi risikoen for frafall underveis er større. Vi støtte på betydelige problemer knyttet til rekruttering av deltakere, og vår erfaring er derfor særlig nyttig med tanke på hvilken type justeringer man kan gjøre underveis i rekrutteringsprosessen.

Vårt mål var å rekruttere et tilstrekkelig antall personer til både borgerpanelet og kontrollgruppen, slik at vi kunne gjennomføre robuste statistiske tester av effekten av å delta i borgerpanelet sammenlignet med å være i kontrollgruppen. Vi ønsket minimum å oppnå 100 deltakere i borgerpanelet, men håpte på opp mot 200 deltakere. Tilsvarende tall eller mer var ønskelig for kontrollgruppen. Videre ønsket vi at alle innbyggere i Bergen kommune skulle ha like stor sjanse for å delta, og at deltakerne i borgerpanelet skulle være representative for innbyggerne i Bergen kommune over 18 år. En av grunnene til at vi ønsket at alle innbyggere skulle ha lik sjanse til å delta, er at vi skal bruke dataene til å forske på hvem som deltar i borgerpaneler, og hvem som ikke deltar.

For å øke insentivet for å delta hadde vi på forhånd bestemt at hver enkelt deltaker i borgerpanelet skulle motta en økonomisk kompensasjon på 500 kroner, og at de var med i trekningen av en pengepremie på 10 000 kroner. For å gi et økonomisk insentiv til deltakelse også i kontrollgruppen var alle deltakere (både i borgerpanelet og kontrollgruppen) også med i trekningen av en pengepremie på 5000 kroner. Vi utformet også invitasjonsbrevet med tanke på å øke sannsynligheten for deltakelse, blant annet ved å poengtere at deltakelse i borgerpanelet gir mulighet til å gi politikerne innspill i viktige lokalpolitiske spørsmål. Et utsnitt av første del av invitasjonsbrevet er gjengitt i tekstboksen nedenfor.

Tilfeldig uttrekk var den eneste aktuelle rekrutteringsformen, særlig fordi vi i tillegg til å arrangere et borgerpanel skulle forske på hvem som deltar, og hvem som ikke deltar i borgerpaneler. Vi søkte Skatteetaten om å få gjøre et tilfeldig uttrekk av innbyggere med bostedsadresse i Bergen fra Folkeregisteret og betalte den eksterne tilbyderen Tietoevry for å gjøre uttrekket og oversende listene til oss. Deretter sendte vi ut individuelle invitasjoner til personene i uttrekket i mai 2021. Ett brev inviterte kun til deltakelse i spørreundersøkelsen (kontrollgruppen), mens det andre inviterte i tillegg til å være med på den deliberative meningsmålingen i juni.

Vi rekrutterte deltakere til et nettbasert borgerpanel midt i koronapandemien. Derfor var vi svært usikre på hvilken svarprosent vi kunne forvente. Byborgerpanelet 2018 hadde en deltakelse på 20 prosent, men i hvilken grad dette tallet var dekkende for deltakelsesvilligheten i populasjonen vi nå skulle rekruttere fra, var uklart. Dessuten var det uklart i hvilken grad svarprosenten ville variere mellom de to invitasjonene. Vi antok at svarprosenten til borgerpanelet ville være lavere enn til kontrollgruppen, men var usikre på hvor mye. Vi begynte derfor med et uttrekk på 2500 personer. Med en svarprosent på 16 prosent for begge invitasjonene ville vi da oppnå 200 per gruppe. Denne svarprosenten var ikke urimelig høy gitt forhåndskunnskapen vår.

Det viste seg imidlertid raskt at denne svarprosenten var i overkant optimistisk, og at det var vesentlig enklere å rekruttere til deltakelse i undersøkelsen enn til deltakelse i panelet. Vi så oss derfor nødt til å sette i gang en rekke tiltak for å få rekruttert nok deltakere, særlig til borgerpanelet. Vi sendte derfor ut 1500 nye invitasjoner, slik at det totale utvalget som ble invitert til slutt, var 4000. Dette var mulig fordi vi satt på et reserveuttrekk fra Folkeregisteret. Vi doblet også den økonomiske kompensasjonen for deltakelse i borgerpanelet fra 500 til 1000 kroner. For å skape blest om arrangementet var både byrådsleder og prosjektleder aktive i lokale medier for å oppfordre til deltakelse i panelet. Det ble også foretatt en purring, enten via brev eller tekstmelding.

Disse tiltakene økte rekrutteringen, selv om det ikke er klart i hvilken grad hvert enkelt tiltak virket. Vi endte til slutt opp med at 138 personer meldte seg til deltakelse i borgerpanelet, noe som tilsvarer en svarprosent på rundt 5 prosent. I kontrollgruppen meldte 168 personer seg til deltakelse. 124 personer i kontrollgruppen svarte på forundersøkelsen, og i deltakergruppen gjorde 118 personer det samme. På selve deliberasjonsarrangementet den 12. juni var det 90 deltakere. Vi opplevde altså betydelig frafall underveis i prosessen.

Vi har ikke analysert direkte sammenlignbare tall for hele populasjonen og kan derfor ikke med sikkerhet bekrefte nøyaktig hvor representativ gruppen er med tanke på deskriptiv representasjon. Når det er sagt, stemmer profilen til gruppen tilsynelatende godt overens med innbyggerne i Bergen med tanke på kjønnsfordeling, statsborgerskap og alder. Vi noterer oss at det er god deltakelse blant de yngre, en gruppe som ellers har tendens til å være underrepresentert på kommunevalg. Dette kan skyldes at yngre er mer komfortable med digital gjennomføring. Pandemien har økt bruken av nettbasert kommunikasjon og med dette senket terskelen for deltakelse i alle aldersgrupper, men vi tror fortsatt at det finnes en forskjell i teknologisk kunnskap som påvirker aldersgruppene ulikt.

Av våre tre eksempler var det den deliberative meningsmålingen i Norge 2022 som hadde høyest deltakelse i form av antall, med 218 deltakere. På samme tid var det den med lavest prosentvis deltakelse, på 1 prosent. I likhet med i 2021 trakk vi ut personer tilfeldig fra Folkeregisteret, men i stedet for å sende fysiske brev overlot vi til et rekrutteringsbyrå å ringe opp mulige deltakere fra listen. 435 personer takket ja til å delta, og blant dem var det xx som faktisk gjennomførte. Dette antallet var ikke tilstrekkelig til å kunne gjøre de statistiske analysene vi hadde planlagt, og vi bestemte oss derfor for å gjøre en oppfølgingsrunde i september 2022. Denne gangen sendte vi digitale brev via Digipost. Dette viste seg å være en kostnadseffektiv løsning. Deltakelsesraten var omtrent den samme, og til slutt endte vi opp med totalt 218 personer som hadde fullført deltakelse i denne deliberative meningsmålingen. I skrivende stund er det søkt om å få koblet på data fra Statistisk sentralbyrå (SSB) for å gjøre en frafallsanalyse av hvilke sosiodemografiske kjennetegn deltakerne har, og om de skiller seg fra dem som ikke ønsket å delta.

\newpage

\hypertarget{informasjon-til-deltakerne}{%
\subsection{Informasjon til deltakerne}\label{informasjon-til-deltakerne}}

\begin{quote}
Henrik Litleré Bentsen
\end{quote}

\includegraphics{figs/1x/Informasjon.png}

Deliberasjon er å veie og utforske alle sider i en sak opp mot hverandre i en grundig, åpen og kritisk drøfting av en sak eller et tema. Tilgang til god og balansert informasjon om konkurrerende sider og argumenter ved en sak er sentralt for å skape grunnlag for god diskusjon og for å sette en ramme for hva som skal diskuteres.

Det finnes flere fremgangsmåter for å sikre at deltakere i borgerpaneler får tilgang til god og balansert informasjon (\protect\hyperlink{ref-fishkin2018democracy}{James S. Fishkin 2018a}). For det første er det vanlig at det utarbeides et informasjonsmateriale som deltakerne skal lese før de tar del i selve deliberasjonen. Ideelt sett utarbeides eller kvalitetssjekkes informasjonsmaterialet av en rådgivende gruppe bestående av ulike eksperter eller «stakeholders» som har interesser i saken som skal diskuteres. For det andre er selve deliberasjonen et viktig sted hvor informasjon om temaet eller saken synliggjøres for deltakerne. Samtidig som vi ikke kan forvente at borgerne i utgangspunktet vil ha oversikt over alle argumentene som ekspertene inkluderer i informasjonsmaterialet, kan vi heller ikke forvente at ekspertene vil ha full oversikt over alle mulige argumenter eller hensyn som borgerne kan ha. Derfor er det sannsynlig at deltakerne gjennom deliberasjonen også vil legge til flere argumenter, enten for eller imot. For det tredje er det nyttig at nye argumenter og fakta som kommer frem under deliberasjonen, blir avklart eller besvart av eksperter underveis. Gjennom at deltakerne får mulighet til å få deres egne argumenter eller faktaspørsmål besvart av eksperter eller beslutningstakere underveis i deliberasjonen, vil de kunne få et bedre grunnlag for å komme frem til en informert beslutning.

Borgerpaneler er designet ut fra disse tre metodene. Ved gjennomføring av et borgerpanel utarbeides det først et balansert informasjonsmateriale som formidles til deltakerne i forkant av deliberasjonen. Deretter deltar borgerne i smågruppediskusjoner, der folk deler sine egne argumenter og påstander for eller imot de ulike forslagene. Underveis i deliberasjonen får deltakerne også mulighet til å få deres argumenter og påstander om fakta besvart av en gruppe eksperter som representerer konkurrerende sider av saken. I tillegg til ekspertene deltar det moderatorer i gruppediskusjonene, som kan bidra til å styre deliberasjonen i riktig retning og gi innspill til deltakerne underveis om temaene som diskuteres.

Når man skal gi informasjon til deltakerne, er det viktig at informasjonen er balansert, etterrettelig og åpent tilgjengelig. Informasjonsmaterialet bør være balansert ved at det fremlegger de sterkeste, evidensbaserte argumentene som (kan) finnes for og imot sakene som skal opp til deliberasjon (\protect\hyperlink{ref-fishkin2021deliberative}{James S. Fishkin 2021}). I prinsippet bør alle argumentene som inkluderes i informasjonsmaterialet, møtes av motargumenter. Argumentene i seg selv bør være informative og understøttet (i så stor grad som mulig) av nøyaktige fakta (\protect\hyperlink{ref-fishkin2005experimenting}{James S. Fishkin and Luskin 2005}). Det bør også være åpenhet om hvor informasjonen er hentet fra, slik at dette kan etterprøves. Informasjonsmaterialet bør være offentlig tilgjengelig for alle, og ikke bare for deltakerne i prosjektet. En vanlig prosedyre er at informasjonsmaterialet legges ut på nettsidene til forskningsprosjektet som har ansvar for å gjennomføre borgerpanelet, eller på hjemmesidene til eventuelle andre ansvarlige aktører.

\hypertarget{veivalg-og-avveininger}{%
\subsubsection{Veivalg og avveininger}\label{veivalg-og-avveininger}}

Borgerpaneler kan brukes til en rekke ulike typer saker, men det er ofte anbefalt å bruke denne fremgangsmåten for å belyse og gi innspill til komplekse saker som innbyggere kan ha lite kunnskap eller informasjon om. En konkret avveining som oppstår når man skal utforme et informasjonsmateriale til slike typer saker, er hvor mye informasjon man skal gi, og hvor langt dokumentet skal være. Nettopp fordi borgerpaneler ofte egner seg best til saker folk ikke kan så mye om, kan man se for seg at informasjonsmaterialet raskt kan bli ganske omfattende. Her er det viktig å huske på at informasjonsmaterialet skal bidra til å legge rammene for gruppediskusjonene på selve arrangementsdagen. Dersom informasjonsmaterialet er svært omfattende, kan det føre til at man ikke kommer gjennom alt, eller at ulike grupper må velge ut og prioritere hvilke deler av informasjonsmaterialet de skal bruke som utgangspunkt for diskusjonene. Dette kan være uheldig når man i borgerpaneler ønsker å legge til rette for at deltakerne skal få mulighet til å komme frem til en så informert beslutning som mulig.

Et annet hensyn å ta er om typen av og lengden på informasjonsmaterialet kan tilfredsstille behovet til ulike typer personer, for eksempel både de som er vant med å lese mye, og de som er vant med å lese lite. Dersom informasjonsmaterialet er svært langt og omfattende, vil det kunne føre til at enkelte grupper eller personer ikke vil kunne møte like godt forberedt til diskusjonene. Tilsvarende viktig er det at innholdet i informasjonsmaterialet ikke er så komplisert at ikke alle som deltar, kan forstå og ta stilling til saken som skal diskuteres. For å hensynta folk som ikke er vant til å lese mye, eller som ikke er like gode til å lese, kan det vurderes om det skal lages en videoversjon av informasjonsmaterialet (\protect\hyperlink{ref-fishkin2021deliberative}{James S. Fishkin 2021}).

En annen avveining man må ta når man utvikler informasjonsmaterialet, er i hvilken grad og hvordan man involverer de som har ekspertise og interesser i saken. Ideelt sett bør informasjonsmaterialet gjennomgås av en balansert rådgivende gruppe av eksperter som representerer ulike synspunkter om saken. Man kan imidlertid se for seg ulike måter å gjøre dette på. En vanlig fremgangsmåte i deliberative meningsmålinger er at det settes ned et utvalg av fageksperter, og eventuelt personer som representerer ulike «stakeholders», som har ansvar for å utforme selve informasjonsmaterialet. Det kan imidlertid være både tid- og ressurskrevende å sette sammen et slikt utvalg. Ekspertene bør kompenseres for arbeidet, og de må også sette av tid til å arbeide med dette i det vi kan anta er en ellers travel hverdag. En annen måte å gjøre dette på (og som vi brukte i gjennomføringen av Den deliberative meningsmålingen i Bergen 2021) er at informasjonsmaterialet først utarbeides av medlemmer av prosjektgruppen, men at det senere kvalitetssjekkes av ekspertutvalget, som da gis anledning til å gi innspill.

\hypertarget{vuxe5re-erfaringer-3}{%
\subsubsection{Våre erfaringer}\label{vuxe5re-erfaringer-3}}

I forbindelse med den deliberative meningsmålingen i Bergen ble det utarbeidet et femten sider langt dokument med utfyllende informasjon om de to sakene som ble sendt til paneldeltakerne i forkant av deliberasjonen. Dette materialet ble offentliggjort på kommunen sine nettsider (informasjonsmaterialet er tilgjengelig her). Det var spesielt viktig at informasjonsmaterialet ble formidlet til deltakerne først etter at den første spørreundersøkelsen ble stengt. På den måten kunne ikke svarene i spørreundersøkelsen bli påvirket av informasjonsmaterialet deltakerne fikk utdelt.

Informasjonsmaterialet ble utviklet i samarbeid mellom forskere fra NORCE og ansatte i Bergen kommune som i det daglige arbeider henholdsvis med utviklingsprogrammet for Dokken og i Bymiljøetaten. Som grunnlag for rapporten ble det tatt utgangspunkt i offentlige dokumenter og fagnotater knyttet til de to sakene, samt forskningsartikler om temaene som bidro til å belyse ulike sider ved sakene. Utkast av informasjonsmaterialet ble sirkulert i de to respektive avdelingene i kommunen, slik at de som hadde innsikt i disse prosjektene, kunne bidra med innspill. Vår erfaring er at personene som var involvert i utviklingen av disse prosjektene, hadde god innsikt i argumentene på begge sider av sakene. Det endelige utkastet ble deretter oversendt til fagekspertene, som deltok på selve arrangementsdagen, slik at de fikk anledning til å gi innspill før informasjonsmaterialet ble ferdigstilt og formidlet til deltakerne.

En sentral utfordring i utviklingen av informasjonsmaterialet var å bestemme seg for hvilke konkrete forslag det skulle settes søkelys på. Begge sakene som skulle diskuteres, er store saker med en rekke ulike sider som kan løftes frem til diskusjon. Dokken-saken handler om å transformere havneområdet Dokken i Bergen til en helt ny bydel innen 2050. En slik transformasjon innebærer en rekke problemstillinger -- alt fra hvilket formål den nye bydelen skal ha, til hvordan man skal navngi gatene. Saken om bilfrie soner er også et langtidsprosjekt som omhandler kommunens satsning på å utvikle et bilfritt byliv både i sentrum og i bydelene. For at gruppediskusjonene ikke skulle bli for overordnede eller abstrakte, valgte vi å sette søkelys på konkrete forslag som allerede var løftet frem i offentlige planer og i media. Disse forslagene skulle være diskusjonstemaer i den deliberative meningsmålingen. Totalt var det 8 forslag knyttet til utbyggingen på Dokken og 9 knyttet til bilfrie soner. Forslagene er oppsummert i Tabell 1. Til hvert forslag ble det i informasjonsmaterialet presentert konkrete for- og motargumenter i tabeller, som også fungerte som en oppsummering av argumentene for og imot som ble diskutert.

\begin{table}[!h]

\caption{\label{tab:tbl-forslag}Forslag som ble diskutert i den deliberative meningsmålingen i Bergen 2021}
\centering
\resizebox{\linewidth}{!}{
\begin{tabular}[t]{ll}
\toprule
\textbf{Tema} & \textbf{Forslag}\\
\midrule
\cellcolor{gray!6}{} & \cellcolor{gray!6}{Sjøfronten skal prioriteres til rekreasjon for mennesker}\\
\cmidrule{2-2}
 
 & Sjøfronten skal prioriteres til naturformål og beskyttelse av dyreliv\\
\cmidrule{2-2}
 
\cellcolor{gray!6}{} & \cellcolor{gray!6}{Sjøen ved dokken skal fylles ut}\\
\cmidrule{2-2}
 
 & Gater, veier og plasser på Dokken skal få kvinnelige navn\\
\cmidrule{2-2}
 
\cellcolor{gray!6}{\multirow{-5}{*}{\raggedright\arraybackslash Arealutvikling på Dokken}} & \cellcolor{gray!6}{Kommunen bør gjøre ytterlige tiltak for å gjøre Bergen sentrum mer bilfritt}\\
\cmidrule{1-2}
 
 & Utviklingen av bilfrie områder bør fremover i større grad fokusere på bydelene og ikke bare sentrum\\
\cmidrule{2-2}
 
\cellcolor{gray!6}{} & \cellcolor{gray!6}{Bilfrie områder utenfor sentrum bør utvikles i de mest sentrumsnære strøkene}\\
\cmidrule{2-2}
 
 & Bilfrie områder bør etableres i mindre områder\\
\cmidrule{2-2}
 
\cellcolor{gray!6}{} & \cellcolor{gray!6}{Bilfrie områder bør etableres i større områder (bydelssentre eller lignende)}\\
\cmidrule{2-2}
 
 & Åpne for fleksibel bruk av parkeringsplasser\\
\cmidrule{2-2}
 
\cellcolor{gray!6}{} & \cellcolor{gray!6}{Større tilgang til bildelingsordninger i nærheten av bilfrie områder}\\
\cmidrule{2-2}
 
 & Skape avstand mellom parkering og bolig\\
\cmidrule{2-2}
 
\cellcolor{gray!6}{} & \cellcolor{gray!6}{Alle trange, tettbygde gater bør få 30-sone}\\
\cmidrule{2-2}
 
\multirow{-9}{*}{\raggedright\arraybackslash Bilfrie bydeler} & La beboerne i en gate/nabolag selv få avgjøre om det skal bli bilfritt eller ikke\\
\bottomrule
\end{tabular}}
\end{table}

I tillegg til informasjonsmaterialet fikk deltakerne informasjon fra, og mulighet til å stille spørsmål til, et panel av fageksperter. Ekspertutvalget for Dokken-saken bestod av representanter for vernemyndigheten, reiseliv, byplanlegging og arkitektur. For bilfrie soner bestod ekspertene av representanter for planmyndighetene og trafikketaten. På selve arrangementsdagen opplevde vi at ekspertene hadde poengterte og gode innlegg som tok for seg forslagene i informasjonsmaterialet på en god måte. Deltakerne stilte dem mange og varierte spørsmål. Vår erfaring er at ekspertene kunne svare på en god del av spørsmålene på direkten, men at det også kom spørsmål som ikke var innenfor ekspertisefeltet til noen av dem. I etterkant har vi også vurdert om det burde ha vært satt av mer tid til plenumssesjonen med fagekspertene. Årsaken til dette er at deltakerne og moderatorene oppfattet den påfølgende gruppediskusjonen som noe repeterende. Dersom vi hadde satt av mer tid til plenumssesjonen med fagekspertene, kunne gruppediskusjonen i stedet ha vært mer av det oppsummerende slaget.

Gruppediskusjonene ble videre ledet av en gruppe moderatorer. Disse ble hovedsakelig rekruttert fra prosjektgruppen, men også frivillige studenter fra Institutt for sammenlignende politikk ved UiB bekledde disse postene. Moderatorene ble i forkant «brifet» av prosjektleder og en erfaren representant for Stanford-teamet. I tillegg var det utarbeidet en skriftlig guide for moderatorene, slik at gruppene skulle få et mest mulig enhetlig uttrykk. I denne guiden ble det tatt opp temaer som hvordan få til gruppesamtaler, hvordan inkludere alle i samtalen, hvordan disponere tiden, og hvordan opptre som moderator. Moderatorene deltok også på et møte med fagekspertene i forkant av plenumssesjonen for å få avklart selve opplegget og innholdsmessige sider ved sesjonene. Vår erfaring er at moderatorene gjorde en god jobb med å lede deres respektive grupper. Under deliberasjonene gjorde moderatorene et viktig arbeid med å sørge for at hver enkelt gruppe fikk kommet seg gjennom og diskutert alle forslagene som ble presentert i informasjonsmaterialet. Moderatorene bidro også til at diskusjonene skjedde i en atmosfære av høflighet og gjensidig respekt. En viktig del av dette arbeidet var å begrense de deltakerne som viste seg å være spesielt snakkesalige, og samtidig legge til rette for at alle deltakere får slippe til med sine synspunkter.

\hypertarget{kort-om-informasjonsmaterialet-i-den-deliberative-meningsmuxe5lingen-i-norge-2022}{%
\subsubsection{Kort om informasjonsmaterialet i den deliberative meningsmålingen i Norge 2022}\label{kort-om-informasjonsmaterialet-i-den-deliberative-meningsmuxe5lingen-i-norge-2022}}

Informasjonsmaterialet til den deliberative meningsmålingen i Norge 2022 ble utformet av prosjektmedlemmene i de involverte forskningsprosjektene. Det ble laget to separate dokumenter -- ett for deltakerne som skulle diskutere bruk av kunstig intelligens i forvaltningen (\href{https://norceresearch.s3.amazonaws.com/Final-Informasjonsmateriell-til-diskusjon_September-versjon.pdf?v=1661350008}{informasjonsmaterialet er tilgjengelig her}), og ett for de som skulle diskutere teknologier for å redusere klimagassutslipp (\href{https://norceresearch.s3.amazonaws.com/fairgov_infoskriv_komp.pdf?v=1663060533}{informasjonsmaterialet er tilgjengelig her}). I likhet med informasjonsmaterialet til den deliberative meningsmålingen i Bergen fikk ekspertene innsyn i materialet og mulighet til å komme med forslag til endringer. Når materialet var ferdig, ble ekspertene oppfordret til å lese teksten, slik at de var godt forberedt på hva deltakerne hadde lest i forkant av arrangementsdagen. Strukturen i informasjonsmaterialet dannet også grunnlaget for agendaen som deltakerne skulle følge i gruppediskusjonene. Den nettbaserte plattformen som ble brukt denne gangen, var SODP. Den sørget for at deltakerne hadde god nok progresjon til å dekke alle spørsmålene som skulle diskuteres.

\newpage

\hypertarget{etter}{%
\subsection{Gjennomføring av panel}\label{etter}}

\begin{quote}
Anne Lise Fimreite
\end{quote}

\includegraphics{figs/1x/Gjennomføring.png}

\hypertarget{prinsipp-og-refleksjoner}{%
\subsubsection{Prinsipp og refleksjoner}\label{prinsipp-og-refleksjoner}}

Ambisjonen med borgerpaneler er å samle et (større) tilfeldig uttrukket utvalg av befolkningen og gi dem anledning til å bli informert og samtale om temaet som er valgt ut, og i de fleste tilfeller også komme med anbefalinger til politikerne om det som har blitt diskutert. For at en samling (enten den er fysisk eller digital) skal gi rom for dette, er selve opplegget svært viktig. I de panelene som er gjennomført, varierer det en del hvor lenge panelene er samlet. Det finnes eksempler på paneler som sitter sammen i flere dager med et omfattende program knyttet til informasjon om saker, presentasjon av ulike synspunkter og diskusjon i ulike grupper og grupperinger før anbefaling og råd fra panelet fremmes. Det finnes også eksempler på paneler som er samlet i relativt kort tid, f.eks. kun noen timer. Informasjonen gis da enten i forkant av samlingen eller i relativt komprimert form ved oppstart av panelet. Hvor lang tid som er satt av til diskusjon og samtale, varierer selvsagt også med varigheten til panelet. Det samme gjør bruken av undersøkelser i løpet av samlingen for å få tak i deltakernes synspunkter og eventuelle endringer av disse.

Borgerpaneler har vært organisert, tilrettelagt og gjennomført på mange ulike måter. Det er likevel tre element som på en eller annen måte går igjen, og som kan tjene som prinsipp for gjennomføringen: 1) Deltakerne må gis en innføring i hva saken de skal ta opp, dreier seg om, 2) de må ha anledning til å diskutere saken seg imellom, og 3) de må kunne gi sine anbefalinger til politikerne som skal ta de endelige avgjørelsene. Et sentralt trekk ved borgerpanel er punkt 2 om at deltakerne får anledning til å kommunisere med andre og tenke seg om før de gjør seg opp en endelig mening. Når det gjelder gjennomføringen, har det også vist seg svært viktig å ha et bevisst forhold til hvordan deltakernes tilbakemeldinger samles inn (punkt 3). Det er av stor betydning å sikre at alle blir hørt i like stor grad. Erfaringsmessig viser det seg for eksempel at gruppesamtaler kan være effektive i mange sammenhenger, men at de også kan skape dynamikker som undertrykker enkeltpersoners muligheter til å ytre sine meninger. Det er også viktig å understreke at når f.eks. en kommune inviterer innbyggerne til å diskutere en sak, kan prosessen være vel så viktig som den konkrete anbefalingen som fremmes. Derfor er det ikke alltid nødvendig å be deltakerne om å diskutere (eller votere) seg frem til enighet om anbefalingene.

\hypertarget{vuxe5re-erfaringer-4}{%
\subsubsection{Våre erfaringer}\label{vuxe5re-erfaringer-4}}

Byborgerpanelet 2018 ble gjennomført som et fysisk panel våren 2018, mens de deliberative meningsmålingene i hhv. Bergen 2021 og Norge 2022 ble gjennomført digitalt. Arrangementene har mange fellestrekk, men også noen særegenheter som det er verdt å merke seg når man står overfor valg av deltakelsesform i et panel.

2018-panelet tok som tidligere omtalt opp fremtidig bydelsorganisering i Bergen kommune. Panelet ble gjennomført i UiBs lokaler i Bergen sentrum og hadde en varighet på to timer. Ved registrering fikk deltakerne et påkoblingsnummer de skulle bruke i den elektroniske surveyen på samlingen. Det viste seg at noen av deltakerne ikke var helt komfortable med bruken av datamaskin, noe som var en viktig del av opplegget. Det fremgikk ikke av invitasjonsbrevet at undersøkelsen skulle besvares elektronisk. Vurderingen var at slik informasjon kunne virke fremmedgjørende for noen. Det bør imidlertid finnes assistanse i lokalet, slik at alle får den hjelpen de trenger for å komme i gang med undersøkelsen. Det kan også vurderes om undersøkelsen skal kunne fylles ut manuelt. Manuell utfylling vil medføre noe koding i etterkant, men de aller fleste vil sannsynligvis velge en elektronisk løsning. Et slikt tilbud vil dermed ikke medføre den helt store ekstrakostnaden. Det gjør også opplegget mer inkluderende. Noen plattformer som brukes ved denne typen undersøkelser, kan imidlertid ikke betjenes manuelt. Da vil et alternativ være at deltakere som ikke behersker digitale verktøy, eller som f.eks. er svaksynte, får lest opp svaralternativene og så får hjelp til inntastingen.

Byborgerpanelets opplegg fulgte ellers malen som erfaringene fra paneler andre steder har vist seg å fungere, og som er presentert ovenfor, det være seg informasjon om saksfeltet, diskusjon av saksfeltet og anbefalinger fra deltakerne. Informasjonen om saksfeltet var todelt. Først ble det gitt et kort sammendrag av innholdet i en rapport som behandlet tilstanden til lokaldemokratiet i Bergen kommune. Diskusjonen om gjeninnføring av bydeler ble sett på som en del av dette saksfeltet, og neste bolk av informasjon dreide seg om bydelsorganisering. I tråd med erfaringer fra andre paneler ble det her lagt vekt på å være så konkret som mulig. Etter at informasjonen var gitt, gjennomførte deltakerne den (digitale) surveyen knyttet til de fire dimensjonene ved bydelsorganiseringen. Undersøkelsen var utformet som et såkalt conjoint-eksperiment. Dette innebar at paneldeltakerne ble presentert for to ulike typer bydeler (bydelspar) som de skulle ta stilling til. Spørsmålet var: «Hvilken av de to foretrekker du?». Bydelene de ble presentert for, varierte hver gang tilfeldig langs de fire dimensjonene antall, handlingsrom, oppgaver og rekruttering. Hver deltaker ble bedt om å sammenligne 10 bydelspar før gruppediskusjonen. Det var satt av ca. 20 minutter til å svare på denne. Det varierte veldig om deltakerne trengte denne tiden.

Etter at undersøkelsen var besvart, ble deltakerne delt inn i mindre grupper for å diskutere bydelsorganisering. Det var satt av 30 minutter til diskusjonen. Gruppeinndelingen var foretatt helt tilfeldig, og det var ikke utpekt noen moderator, men forskerne var innom alle gruppene for å sette i gang diskusjonen. Det ble lagt vekt på viktigheten av å være innom alle dimensjonene i gruppediskusjonen.

Etter gruppediskusjonen ble deltakerne igjen samlet og bedt om å gjennomføre den digitale undersøkelsen en gang til. Igjen var det satt av 20 minutter til dette. Også nå ble deltakerne bedt om å sammenligne 10 bydelspar. Samlet ga alle deltakerne over 3000 svar om hvilke bydeler som foretrekkes, og hvilke som ikke foretrekkes, gitt egenskapene til de to bydelstypene som sammenlignes. De to undersøkelsene dannet sammen grunnlag for å utarbeide en felles anbefaling fra panelet om hvordan bydeler bør organiseres i fremtiden i Bergen kommune. Anbefalingen ble oversendt Bergen kommune ved byrådsleder fra prosjektledelsen vel en måned etter at panelet ble avholdt, og fulgte saken videre til endelig beslutning i bystyret. Denne saksgangen ble det opplyst om ved panelets avslutning.

Av tidsmessige og økonomiske årsaker ble opplegget i Byborgerpanelet gjennomført svært kompakt med en varighet på totalt to timer. Begrunnelsen var en antakelse om at det vil være lettere å rekruttere folk til et to-timersopplegg enn til et panel som strekker seg over én eller flere dager. Det kompakte opplegget gjorde at informasjonen som ble gitt, kan ha blitt oppfattet som summarisk og overflatisk av noen. Det var i liten grad rom for å presentere alternative sider ved bydelsorganiseringen eller å slippe til innledere med ulike synspunkter når det gjaldt både nærdemokratitiltak og bydeler.

Det andre panelet vi skal si noe om erfaringene fra, er den digitale deliberative meningsmålingen som ble gjennomført i samarbeid med Bergen kommune sommeren 2021. Temaet for denne meningsmålingen var som vi har presentert i tidligere kapitler, Dokken-utbyggingen i Bergen sentrum og innføring av bilfrie soner utenfor sentrumskjernen i Bergen. Også i et digitalt borgerpanel er grunnidéen å få vite hva innbyggerne mener når de har fått tenkt seg om og vurdert argument for et politisk forslag. Opplegget er derfor nokså likt et fysisk panel. Et representativt utvalg ble invitert til å diskutere med sine medborgere i et digitalt rom. Deltakerne presenteres for de to sakene i et informasjonsskriv i forkant, og tar så stilling til saken(e) i en spørreundersøkelse i forkant av samlingen. På selve samlingen blir de delt inn i grupper ledet av en moderator, hvor saken diskuteres. Etter diskusjonen får gruppene anledning til å stille spørsmål til fageksperter i en plenumssesjon, før de diskuterer saken igjen. Mot slutten av arrangementet (eventuelt etter arrangementet) svarer deltakerne på en spørreundersøkelse lik den de svarte på før arrangementet startet. Dette gir mulighet til å måle effekten av gruppediskusjonene (deliberasjonen). En kontrollgruppe som ikke deltok på arrangementet, og som dermed ikke fikk tilgang til diskusjonene og presentasjonene der, fikk samme undersøkelse samtidig med deltakerne, slik at tidsaspektet i seg selv til en viss grad er kontrollert for.

Moderatorene for gruppene var prosjektmedarbeidere og statsvitenskapsstudenter fra UiB. Moderatorene ble i forkant av samlingen brifet av prosjektleder og en svært erfaren amerikansk professor som har gjennomført denne typen samlingen over hele verden. Det var også utarbeidet en skriftlig guide for moderatorene. Hensikten med både brifingen og guiden var at samtlige grupper skulle få et mest mulig enhetlig uttrykk.

Noen dager før arrangementet fikk deltakerne tilsendt en Zoom-lenke de skulle bruke på selve dagen. Arrangement var fra kl. 10 til kl. 16, og lenken ble åpnet litt før kl. 10. Hele dagen var teknisk kompetent personale på plass i en fysisk «operasjonssentral». Gitt at det var nærmere 90 deltakere i panelet, oppsto det få alvorlige problemer med tilkoblingen til den digitale plattformen som ikke enkelt lot seg løse.

Deltakerne ble ønsket velkommen i plenum og gitt praktisk informasjon om dagsorden og planene for dagen. Deretter ble de umiddelbart delt inn i mindre grupper med moderator. (her ble \emph{Breakout}-funksjonen i Zoom brukt). Den første gruppesamlingen varte i 45 minutter. Mot slutten sendte hver gruppe inn spørsmålene de ønsket å stille til ekspertene i den påfølgende plenumssesjonen. Ekspertene til begge plenumssesjonene hadde i forkant av samlingen hatt møter med de som skulle lede sesjonene. Hensikten med disse møtene var å informere om opplegget samt å diskutere innholdsmessige sider ved sesjonene, bl.a. for å hindre for mye overlapping mellom innlederne. Spørsmålene fra gruppene ble sendt til sesjonsleder av moderatorene. Til dette formålet ble e-post og chattefunksjonen i Zoom brukt. Det var også mulig for gruppedeltakerne å stille spørsmål direkte til ekspertene, enten ved «håndsopprekking» eller via chattefunksjonen under plenumssesjonen.

Plenumssesjonen startet med at ekspertene, som i Dokken-utbyggingen representerte vernemyndighetene, reiselivet, byplanleggings- og arkitekturkompetanse, holdt korte innlegg om Dokken-utbyggingen fra eget ståsted. Det kom mange spørsmål til ekspertene, både før og under plenumssesjonen. Det virket ikke som at deltakerne var særlig hemmet av det digitale formatet. Etter plenumsrunden var det ny gruppediskusjon før det ble avholdt en 45 minutters lunsjpause.

Etter lunsj startet arrangementet opp igjen med nye gruppesamlinger. Denne gangen var temaet bilfrie soner. Opplegget ellers var nøyaktig det samme som før lunsj. Denne gangen var ekspertene representanter for kommunens planmyndighet og trafikketat.

Etter den siste gruppesamlingen om bilfrie soner ble det avholdt en kort fellessamling. Det ble gitt informasjon om hvordan kommunen kom til å bruke de synspunktene og anbefalingene som kom frem gjennom den deliberative meningsmålingen. Panelet ble avsluttet med at deltakerne gjennomførte den siste undersøkelsen.

Anbefalingene fra panelene ble rapportert til Bergen kommune i en egen rapport som ble oversendt byrådet vinteren 2022. Det ble også avholdt et eget seminar for kommunens øverste politiske og administrative ledelse, der prosjektet DEMOVATE ble lagt frem i februar 2022.

Det tredje og siste eksempelet var den deliberative meningsmålingen i Norge 2022. Mye av oppsettet var likt fra den deliberative meningsmålingen i Bergen 2021. Et viktig unntak var at dette arrangementet foregikk på Stanford Online Deliberation Platform (SODP). I motsetning til Zoom, som mange hadde kjennskap til fra før, var SODP ukjent for deltakerne. Vi var forberedt på at det kunne dukke opp tekniske utfordringer for brukerne, men dette merket vi lite til. Det var en trygghet at vi hadde åpnet for at deltakerne kunne teste mikrofon og kamera dagene i forkant av arrangementet. Vi hadde også to drop in-timer, hvor de kunne snakke med oss personlig dersom de hadde spørsmål av teknisk eller annen art. Det var få som benyttet seg av drop in-tilbudet, men noen få ville nok ha falt fra dersom de ikke hadde fått denne hjelpen.

Det tok noen minutter før deltakerne ble komfortable med funksjonene på plattformen. De måtte melde på forhånd at de ville snakke, og dersom de brukte mer enn sin tilmålte tid, ble de kuttet av. Etter hvert fikk de dreisen på det, og teknisk fungerte opplegget i hovedsak godt. Plattformen viste deltakerne hvor langt de var kommet i agendaen, og hvilke spørsmål som sto for tur. Personer som ikke hadde snakket, ble automatisk plukket opp av den innebygde moderatoren, som da oppfordret dem til å ta ordet.

En annen forskjell fra 2021 var at vi hadde delt deltakerne inn i to hovedgrupper. De som skulle diskutere kunstig intelligens, møtte opp på lørdagen, mens de som skulle diskutere negative utslippsteknologier, deltok på søndagen. Arrangementet varte fra kl. 10--15. Tabell \ref{tab:tbl-dagsplan} viser hvordan agendaen for lørdagen ut.

\begin{table}[!h]

\caption{\label{tab:tbl-dagsplan}Dagsplan for deliberativ meningsmåling i Bergen 2021}
\centering
\resizebox{\linewidth}{!}{
\begin{tabular}[t]{llll}
\toprule
\textbf{Starttidspunkt} & \textbf{Sluttidspunkt} & \textbf{Aktivitet} & \textbf{Beskrivelse}\\
\midrule
\cellcolor{gray!6}{10:00} & \cellcolor{gray!6}{10:10} & \cellcolor{gray!6}{Velkommen} & \cellcolor{gray!6}{}\\
\cmidrule{1-4}
10:10 & 10:55 & Dokken: Første gruppesamling & Opptelling av deltakere, gjennomgang av agenda, forberede spørsmål til panel\\
\cmidrule{1-4}
\cellcolor{gray!6}{11:00} & \cellcolor{gray!6}{11:45} & \cellcolor{gray!6}{Panel} & \cellcolor{gray!6}{Moderator:  Anne Lise Fimreite}\\
\cmidrule{1-4}
11:50 & 12:30 & Dokken: Andre gruppesamling & Debrief og videre diskusjon i gruppen\\
\cmidrule{1-4}
\cellcolor{gray!6}{12:30} & \cellcolor{gray!6}{13:15} & \cellcolor{gray!6}{Lunsj} & \cellcolor{gray!6}{}\\
\cmidrule{1-4}
13:15 & 14:00 & Bilfrie områder: Første gruppesamling & Gjennomgang av agenda, forberede spørsmål til panel\\
\cmidrule{1-4}
\cellcolor{gray!6}{14:10} & \cellcolor{gray!6}{14:55} & \cellcolor{gray!6}{Panel} & \cellcolor{gray!6}{Moderator: Jacob Aars}\\
\cmidrule{1-4}
15:05 & 15:50 & Bilfrie områder: Andre gruppesamling & Debrief og videre diskusjon i gruppen.\\
\cmidrule{1-4}
\cellcolor{gray!6}{15:50} & \cellcolor{gray!6}{16:00} & \cellcolor{gray!6}{Avslutning} & \cellcolor{gray!6}{Avslutning}\\
\bottomrule
\end{tabular}}
\end{table}

Søndagen var tilnærmet identisk. Vi merket at det å sitte fremfor en PC en hel dag tæret på deltakerne, og anbefaler derfor ikke lengre opplegg enn fem timer. Det er også viktig å ha en lang nok lunsjpause.

Den største forskjellen mellom de tre panelene er åpenbart at det ene er fysisk, mens de to andre er digitale. Andre viktige forskjeller er også

\begin{itemize}
\tightlist
\item
  varighet -- to vs.~seks timer\\
\item
  saksomfanget (én vs.~to saker)\\
\item
  sakstype (arealdisponering vs.~organisasjonsstruktur vs.~kunstig intelligens i beslutninger), og\\
\item
  måten informasjonen om saken(e) som skulle behandles i panelet, ble gitt på (direkte i panelet vs.~skriftlig informasjon i forkant).
\end{itemize}

Heller enn å gå inn i en vurdering av hvilke av disse oppleggene som fungerte best, vil vi peke på at både digitale og fysiske paneler har vist seg å fungere. De to panelene i Bergen førte begge til at kommunen fikk råd og anbefalinger som ble tatt inn i den videre saksbehandlingen. Våre erfaringer er variasjoner over det tredelte opplegget --- informasjon, deliberasjon, råd --- som har vist seg å fungere mange andre steder, og som også har gjort det i våre tre borgerpaneler.

\newpage

\hypertarget{hvordan-muxe5le-resultater}{%
\subsection{Hvordan måle resultater}\label{hvordan-muxe5le-resultater}}

\begin{quote}
Sveinung Arnesen
\end{quote}

\includegraphics{figs/1x/måling.png}

\hypertarget{prinsippermuxe5l-som-veileder-valgene-1}{%
\subsubsection{Prinsipper/mål som veileder valgene}\label{prinsippermuxe5l-som-veileder-valgene-1}}

Det er viktig at man allerede i planleggingsfasen har det klart for seg hva som skal komme ut av borgerpanelet. For de fleste formål er deltakernes holdninger selve sluttproduktet som borgerpanelet skal levere, og som beslutningstakerne skal agere på. Hvordan holdningene måles, kan påvirke hvilke konklusjoner som trekkes fra borgerpanelet. Det kan også ha mer utilsiktede konsekvenser på villigheten til å delta i borgerpaneler.

Vi anbefaler å måle deltakernes holdninger individuelt, i en anonym spørreundersøkelse som sendes ut mot slutten av arrangementet. Ved å gjennomføre en spørreundersøkelse får man helt konkrete svar på holdningene til gruppen.

\hypertarget{klare-svar}{%
\paragraph{Klare svar}\label{klare-svar}}

Ofte er selve målet med gjennomføringen at beslutningstakerne -- for eksempel politikere i en kommune -- skal sitte igjen med klare svar om hvilke holdninger deltakerne har etter at de har fått reflektert rundt spørsmålene. Dersom man gjennomfører en spørreundersøkelse, vil deltakerne måtte ta stilling til helt konkrete spørsmål fra arrangøren. En godt utformet spørreundersøkelse gir rikelig med informasjon om deltakernes holdninger og om deres sosiale bakgrunn.

Det er også mulig å måle om deltakerne endrer holdning av å delta i borgerpanelet. Dette er noe forskere ofte er opptatt av, for å kunne måle om deliberasjonen har hatt noen effekt på deltakernes holdninger. For å kunne gjøre slike analyser må deltakerne få to spørreundersøkelser -- en før arrangementet og en etter.

Om man virkelig ønsker å gå vitenskapelig til verks, måler man eventuelle holdningsendringer opp mot en kontrollgruppe som ikke har deltatt på borgerpanelet. Dette for å kontrollere at det er deltakelse i borgerpanelet som har ledet til holdningsendringene, og ikke for eksempel noe annet som kan ha skjedd i verden i tiden som har gått mellom første og andre spørreundersøkelse.

Det er en fordel å involvere personer med kompetanse på utforming av spørreundersøkelser. Å lage spørsmål som både er forståelige for respondentene, som fungerer godt til diskusjon, og som faktisk gir svars på det arrangørene vil vite, høres kanskje tilforlatelig ut, men dette er en kunst som de færreste er utlært i. Det er svært nyttig å få med på laget noen som har gjort dette før, enten personer innomhus, meningsmålingsbyråer eller surveyforskere. Disse vil også ha kjennskap til verktøy som kan lette arbeidet. Mange organisasjoner har avtaler med selskaper som kan tilby de tekniske løsningene som kreves for å lage en spørreundersøkelse, og kanskje kan de også konsulteres i design av spørsmål og analyser av svarene.

\hypertarget{gruppeholdning-vs.-samling-av-individuelle-holdninger}{%
\paragraph{Gruppeholdning vs.~samling av individuelle holdninger}\label{gruppeholdning-vs.-samling-av-individuelle-holdninger}}

Hvis deltakerne svarer i en spørreundersøkelse, uttrykker de sine individuelle holdninger. Svarene deres blir anonymisert på samme måte som i vanlige spørreundersøkelser. Det er i prinsippet ikke noe i veien for at de svarer uten anonymitet. Da må de «stå til ansvar» for holdningene sine, slik valgte politikere må gjøre. Noen vil kanskje mene at dette er positivt, og at det styrker legitimiteten til arrangementet. Det er verdt å påpeke at eksperimenter vi har gjennomført, viser at villigheten til å delta på arrangementer synker dersom deltakerne må offentliggjøre sine holdninger.

Noen former for borgerpaneler -- for eksempel borgerjuryer -- legger opp til at deltakerne skal uttrykke sine holdninger samlet som gruppe. Dette kan for eksempel gjøres ved at de blir enige om en felles uttalelse. En kritikk mot et slikt opplegg er at enkeltdeltakere kan få uforholdsmessig stor innvirkning på gruppens holdninger slik de uttrykkes. Mange vegrer seg for å gå inn i opphetede diskusjoner med andre og vil vike for sosialt press fra dominerende deltakere eller det de oppfatter som majoriteten.

\hypertarget{analyser-av-gruppediskusjoner}{%
\paragraph{Analyser av gruppediskusjoner}\label{analyser-av-gruppediskusjoner}}

Dersom gruppediskusjonene tas opp og transkriberes, inneholder dette datamaterialet mye informasjon om hva deltakerne mener om sakene som diskuteres. Vi anbefaler imidlertid ikke at dette brukes som et direkte mål på deltakernes holdninger. Dette fordi denne delen av arrangementet nettopp er en arena for å løfte frem for- og motargumenter, før man har gjort seg opp en bestemt mening. Hvis man skal bruke datamaterialet til å måle holdningene til deltakerne, må dette i så fall kommuniseres tydelig til deltakerne på forhånd, og da risikerer man at diskusjonene blir mindre åpne og dynamiske enn man hadde tenkt. Det er dessuten utfordrende å analysere ustrukturert tekstmateriale på en objektiv måte.

Datamaterialet kan imidlertid fungere godt for å eksemplifisere holdninger som kommer til uttrykk i den påfølgende spørreundersøkelsen. Tekstmaterialet kan gjerne brukes til å visualisere hva diskusjonene handlet om, for eksempel i en ordsky. Materialet innbyr også til mer kompliserte analyser av forskningsspørsmål knyttet til gruppedynamikk, betydningen av agendasetting osv.

\begin{figure}
\centering
\includegraphics{figs/ordsky-bilfri.png}
\caption{Ordsky fra gruppediskusjon om bilfrie soner}
\end{figure}

\hypertarget{vuxe5re-erfaringer-5}{%
\subsubsection{Våre erfaringer}\label{vuxe5re-erfaringer-5}}

I alle våre tre eksempler har vi benyttet oss av spørreundersøkelser både før og etter gruppediskusjonene. Vi brukte Qualtrics, som er en internasjonal leverandør av spørreundersøkelser. Qualtrics er bare ett av flere selskaper som leverer denne typen tjenester. Vi valgte dem fordi vi hadde kjennskap til dem fra før, og fordi de kunne levere enkelte spesialtjenester som vi hadde behov for i våre undersøkelser.

Det ble lagt opp til et såkalt pretest/posttest-kontrollgruppedesign. Det vil si at vi målte holdningene til deltakerne både før og etter arrangementet. På denne måten kan vi måle om deltakernes holdninger endrer seg etter å ha diskutert sakene og fått tid til å tenke seg om. For å utelate at eventuelle holdningsendringer kommer som en konsekvens av andre forhold som har skjedd i mellomtiden, hadde vi med en kontrollgruppe som også gjennomførte spørreundersøkelsene, men som ikke deltok på selve arrangementet. Et slikt oppsett er et av de mest vanlige eksperimentelle designene (\protect\hyperlink{ref-cook2002experimental}{Cook, Campbell, and Shadish 2002}).

\begin{table}[!h]

\caption{\label{tab:tbl-design}Eksperimentelt design for deliberativ meningsmåling i Norge 2022}
\centering
\resizebox{\linewidth}{!}{
\begin{tabular}[t]{llll}
\toprule
\textbf{Gruppe} & \textbf{Pretest} & \textbf{Eksperimentkomponent} & \textbf{Posttest}\\
\midrule
\cellcolor{gray!6}{Bruk av kunstig intelligens i forvaltningen} & \cellcolor{gray!6}{Spørreundersøkelse} & \cellcolor{gray!6}{Gruppediskusjon om kunstig intelligens} & \cellcolor{gray!6}{Spørreundersøkelse}\\
 
Negative utslippsteknologier & Spørreundersøkelse & Gruppediskusjon om negative utslippsteknologier & Spørreundersøkelse\\
\bottomrule
\end{tabular}}
\end{table}

I den deliberative meningsmålingen i Norge 2022 lot vi deltakerne fra henholdsvis kunstig intelligens-gruppen og negative utslippsteknologier-gruppen være kontrollgruppe for hverandre. Fordelen med dette var at vi ikke trengte å rekruttere egne respondenter som kun skulle være i en kontrollgruppe uten å få være med på å diskutere noen av sakene.

Svarene som ble vektlagt overfor politikerne i 2018 og 2021, var svarene deltakerne ga etter at de hadde gjennomført gruppediskusjonen. Samtidig ble det også formidlet i hvilken grad deltakerne endret holdninger fra før gruppediskusjonen, og om slike holdningsendringer bevegde seg mer enn eventuelle holdningsendringer i kontrollgruppen. I 2022 var vi spesielt opptatt av forskjellen i holdningsendring mellom kontrollgruppene og behandlingsgruppene, og vi analyserte resultatene ut fra dette perspektivet.
 

\newpage

\hypertarget{oppfuxf8lging-av-arrangement}{%
\subsection{Oppfølging av arrangement}\label{oppfuxf8lging-av-arrangement}}

\begin{quote}
Anne Lise Fimreite
\end{quote}

\includegraphics{figs/1x/Oppfølging.png}

\hypertarget{antakelser-og-refleksjoner}{%
\subsubsection{Antakelser og refleksjoner}\label{antakelser-og-refleksjoner}}

Litteraturen om borgerpaneler presenterer en rekke antakelser, ikke bare om hvordan panelets utforming kan tenkes å påvirke beslutningstakernes vurdering av panelene, men også om hvordan politikerne kan tenkes å ta imot de anbefalingene som fremkommer fra panelene. (dette kapitlet bygger i noen grad på \protect\hyperlink{ref-arnesenloddet}{Arnesen, Fimreite, and Aars 2021}). Det finnes også antakelser knyttet til hvordan politikerne og andre i systemet velger å ta panelets anbefalinger videre. Stort sett er antakelsene knyttet til at paneler med det som omtales som høy indre kvalitet (bl.a. knyttet til representativ rekruttering og gode vilkår for meningsutveksling), også vil ha høy ekstern kvalitet, dvs. at paneler med disse kjennetegnene vil kunne påvirke politiske beslutninger i tråd med sine anbefalinger. Empiriske erfaringer viser imidlertid at hvordan panels anbefalinger påvirker systemets beslutninger, ikke bare er avhengig av trekk som antas å ha med den indre kvaliteten å gjøre, men at det i aller høyeste grad også er avhengig av hvordan panelet knyttes til det vi kan kalle det politiske systemet i vid forstand. Med det politiske systemet mener vi her beslutningstakere, politiske institusjoner og velgere/innbyggere.

Nærmest uavhengig av hvilke funksjon panelet er tiltenkt, og hvilke saker det behandler, trekker forskningen frem hvor viktig det er at panelet aksepteres av dette politiske systemet. Ett forhold trekkes i så måte frem som særlig viktig, nemlig kapasiteten på mottakersiden til å inkludere panelets anbefalinger i pågående prosesser. Etter å ha studert fire paneler på ulike kontinenter konkluderte forskerne Curato og Böker med at borgerpaneler faktisk kan ha potensielt uheldige konsekvenser for et politisk system og for den videre demokratiske konteksten i et samfunn dersom det mangler aksept og kapasitet i samspillet mellom panelet og systemet. Erfaringene fra et panel i Belgia viser at uansett hvilke vurderinger politikerne som mottok panelets anbefalinger, hadde av dets arbeid, så var ikke det representative systemet upåvirket av panelets arbeid. Politikerne ble tvunget til å tenke gjennom hvilken rolle borgerpanelet kan ha i systemet -- og om de avviser dets anbefalinger. Disse forholdene gjør det viktig å påpeke at et panel ikke er avsluttet idet arrangementet er ferdig og rapporten med dets anbefalinger er overlevert oppdragsgiver. Oppfølging i ettertid av et panel er også essensielt om det skal ha noen plass i det politiske systemet.

Det er også viktig at de som har deltatt i borgerpanelene, følges opp. De som har satt av tid og engasjert seg i denne typen demokratisk aktivitet, har krav på å bli fulgt opp også etter at arrangementet er ferdig. Først i fremst gjelder dette rent praktisk, f.eks. ved at løfter om honorarer og premier innfris i løpet av en rimelig (og avtalt) tidshorisont. Det er også viktig at deltakerne får tilgang til rapporten med panelets anbefalinger som utarbeides, enten ved at de får den tilsendt i posten, eller ved at de får tilsendt en lenke til rapporten. Etter at rapporten er oversendt oppdragsgiver, er det også ønskelig at paneldeltakerne får mulighet til å se hvordan anbefalingene til panelet blir fulgt opp gjennom den politiske prosessen. Dette kan f.eks. gjøres ved at det opprettes en nettside for det gitte borgerpanelet, eller ved at deltakerne tilskrives når en sak er sluttbehandlet i det politiske systemet. Aktiv bruk av media (redaksjonelle så vel som sosiale) kan også være en måte å informere deltakerne på om saksgangen og saksutfallet.

Denne oppfølgingen er avgjørende, ikke bare som en gest overfor de som har satt av tid til deltakelse, men også for legitimiteten til ordningen med borgerpanel. Dersom anbefalingene ikke blir brukt aktivt i de politiske prosessene, er veien kort til at både panelene og anbefalingene kun blir oppfattet som interessante for de involverte forskerne, bl.a. for å finne ut om informasjon og diskusjon påvirker synspunktene. Selv om det kan være svært interessant i seg selv å finne ut av dette (også for de som har deltatt), er det å kunne få å si sitt i politiske saker noe helt annet.

Undersøkelser både her hjemme og i utlandet viser at innbyggere som ikke deltar i panelene, er opptatt av hvordan panelene er satt sammen og utformet, og hvilke saker de tar opp. Men de er også opptatt av at panelets diskusjoner, eventuelle avstemninger og anbefalinger blir offentliggjort. For at borgerpaneler skal fungere med tanke på den videre demokratiske konteksten og som skoler i demokrati, ved å være noe innbyggerne er oppmerksom på at de selv kan bli trukket ut til å delta i, må de være kjent. En del av arbeidet med å gjøre panelet og dets arbeid kjent må gjøres som oppfølging i etterkant av gjennomføringen.

\hypertarget{vuxe5re-erfaringer-6}{%
\subsubsection{Våre erfaringer}\label{vuxe5re-erfaringer-6}}

Begge panelene som ble arrangert i samarbeid med Bergen kommune (Byborgerpanelet 2018 og den deliberative meningsmålingen i Bergen 2021), var godt forankret i kommunens politiske ledelse i forkant. Byborgerpanelet var foreslått i en rapport om kommunens nærdemokrati våren 2017 , men det var bystyret selv som i behandlingen av denne rapporten aktivt valgte å arrangere et byborgerpanel. Det var også bystyret som selv pekte ut fremtidig bydelsorganisering som aktuell sak for byborgerpanelet. Slik sett var den politiske oppmerksomheten til stede i forkant av panelet.

Også den deliberative meningsmålingen i 2021 kom som et resultat av et politisk ønske fra bystyret i og med at politikerne aktivt tok del i beslutningen om at Bergen kommune skulle være med i prosjektet DEMOVATE. Den konkrete utforming av de panelene som var planlagt å skulle inngå i DEMOVATE, var delegert til kommunens utøvende apparat, og først og fremst til byrådslederens kontor. Koronapandemien satte en stopper for planene om å avvikle et fysisk panel våren/sommeren 2020, men forarbeidene til et slik arrangement var kommet godt i gang, og aktuelle saker var diskutert med kommunens ledelse. Da pandemien gjorde at fysisk panel måtte utsettes, og etter hvert skrinlegges, og tidspunktet ble forskjøvet med ett år, ble andre saker mer aktuelle for panelet. Hvilke saker som skulle behandles, var gjennom en intern prosess i Bergen kommune før dette ble diskutert med forskerne, og det ble fattet en endelig beslutning om hvilke to saker det skulle være.
Kommunen var dermed svært involvert i å velge ut saker både i byborgerpanelet og i den deliberative meningsmålingen 2021. Sakene var aktuelle for dem, både på det politiske og administrative nivået. Dette ble klart formidlet til paneldeltakerne både i invitasjonsbrevet og på selve arrangementet. Også allmennheten ble informert om panelene via lokalpressen og NRK Hordaland. Oppfølgingen av de to panelene i det politiske systemet skulle dermed ha de beste forutsetninger.

Den praktiske oppfølgingen i form av utbetaling av honorarer og utsending av rapporten var et samarbeid mellom forskningsinstitusjonen og kommunen. Det var et til dels et tidkrevende arbeid å samle inn all nødvendig informasjon fra deltakerne og få sendt ut honorarer til alle, men dette ble gjennomført, selv om en del av deltakerne mente at det tok for lang tid. I denne forbindelse er det viktig å informere om prosessen og prosedyrene, slik at ikke folk forventer å få pengene inn på konto dagen etter at panelet ble arrangert. Det er også viktig å informere om at det kan ta tid å skrive en rapport. Disse praktiske sidene av oppfølgingen er likevel svært håndterlige. Gode rutiner som er fastlagt i forkant, er til god hjelp.

Ingen av de to panelene er imidlertid i særlig grad informert om til allmennheten utover det som ble gjort i forkant. Her kunne det helt klart ha vært gjort en bedre jobb, både med å informere fra selve arrangementene og om hva panelene kom frem til. Panelene utfører tross alt demokratisk arbeid på vegne av fellesskapet, og da er det rimelig at fellesskapet til en viss grad får ta del i arbeidet.

Oppfølgingen fra kommunens side har vi noe begrenset innsikt i. Det vi vet, er at Byborgerpanelets anbefalinger fulgte saken frem til det ble fattet vedtak, men vi vet samtidig at anbefalingene ikke ble diskutert i noen særlig grad i de representative politiske foraene. En masteroppgave ved UiB våren 2021 avdekket at politikerne i liten grad husket at de hadde bedt om Byborgerpanelet, og at de i enda mindre grad var opptatt av og kunne huske anbefalingene fra panelet. Dette gjaldt både politikere i bystyret og i byrådet. Stabilitet i politisk oppmerksomhet har i mange endringssituasjoner og reformer vist seg å være svært avgjørende. Utskifting av byrådsleder i forkant av valget i 2019 kan ha ført til at den politiske oppmerksomheten rundt Byborgerpanelet (og bydelsorganiseringen) ble noe mindre i den politiske ledelsen. Selv om denne endringen kom i forkant av et valg, vil valgsykluser alltid være noe som kan føre til endringer i den politiske ledelsen, og dermed også i oppmerksomheten rundt igangsatte forsøk. Dette er noe som må tas hensyn til også i planleggingen av hvordan borgerpanelet skal følges opp.

Rapporten fra den deliberative meningsmålingen 2021 ble oversendt kommunen vinteren 2022, og det ble arrangert et møte mellom forskerne og kommunens administrative og politiske toppledelse i februar i 2022, der deler av rapporten ble presentert. Sluttføringen av rapporten, og ikke minst overleveringen av den til kommunen, må ses i lys av at høsten 2021 var sterkt preget av koronarestriksjoner. Avtalte møter for å diskutere fremdriften ble avlyst, og overleveringen ble noe forsinket. Det skal heller ikke stikkes under stol at det er vanskelig å få politisk (til dels også administrativ) oppmerksomhet i en situasjon så preget av krisehåndtering som både Bergen og resten av Norges kommuner har vært i store deler av DEMOVATEs prosjektperiode. I denne fasen har det også skjedd utskiftinger i den politiske ledelsen, så oppmerksomheten om prosjektet DEMOVATE måtte fanges på nytt. Kommunens prosjektansvarlige har også gitt uttrykk for at dette har vært krevende.

Stabilitet på oppdragssiden er viktig for oppfølging. Denne stabiliteten har administrasjonen i kommunen representert, og de vil også fremover være avgjørende for hvordan anbefalingene i sakene som ble behandlet i den deliberative meningsmålingen, følges opp. I denne forbindelse er det viktig å ha i mente at ingen av sakene som dette panelet tok opp, var dagsaktuelle i den forstand at det skal fattes bindende politiske beslutninger med det første. Det er likevel grunn til å peke på et dilemma knyttet til at et tiltak som er innrettet mot demokratisk innovasjon og deltakelse gjennom oppfølgingen, kan ende opp som et verktøy for administrasjonen i en kommune. Da kan den demokratiske innovasjonen få mer karakter av høring og dialog, og selve deltakelsesaspektet kan havne mer i bakgrunnen.

Den deliberative meningsmålingen i Norge 2022 har større avstand til politiske beslutningstakere. Den følger mer ordinære forskningsprosjekter, hvor formidlingen i første omgang rettes mot det akademiske miljøet både nasjonalt og internasjonalt. Samtidig er begge temaene i denne deliberative meningsmålingen absolutt av interesse for både allmennheten og det politiske systemet. Nettopp det at vanlige innbyggere har fått anledning til å diskutere komplekse temaer som ellers unngår de største overskriftene i media (i hvert fall inntil videre), gjør at resultatene fra arrangementet kan være med på å sette dagsorden for en større, allmenn debatt.

Som vi har diskutert flere steder i denne boken, har borgerpaneler mange steder vist at innbyggerne har kvalifiserte holdninger som kan gi et bredere spekter av erfaringer som kan danne grunnlag for politiske beslutninger. Bedre forankring av saker og høyere legitimitet av både prosesser og vedtak kan være en konsekvens av dette. Brukt på rett måte har borgerpaneler dessuten potensial til å aktivisere flere enn bare de som blir trukket ut til å delta. En betingelse for å få dette til er som diskutert i tidligere kapitler at deltakerne representerer et tverrsnitt av befolkningen. Dette har vist seg å være vanskelig å få til, og det er uansett ikke nok om borgerpanel skal ha en betydning i et representativt system på en måte som er mer varig enn enkeltforsøk. Våre erfaringer er at det må være avklart på forhånd hvilken plass et panels arbeid, og ikke minst dets anbefalinger, skal ha i det politiske systemets prosesser og strukturer. Empirien både fra utlandet og fra våre forsøk viser at heller ikke dette er uten utfordringer, men at det er helt avgjørende. Den viktigste delen av oppfølgingen starter derfor i forkant av et borgerpanel.

\newpage

\hypertarget{del-iii-andre-erfaringer}{%
\section{DEL III: ANDRE ERFARINGER}\label{del-iii-andre-erfaringer}}

\includegraphics{figs/1x/Del3.png}
Denne delen bringer inn perspektiver fra to andre vinkler enn de som er presentert så langt. Det første kapitlet er forfattet av et av medlemmene i forskningsprosjektet Demovate fra kommunen side. Kapitlet gir innsikt i hvordan det å befatte seg med medvirkning oppleves av kommuneadministrasjonen. Det andre kapitlet er skrevet av en PhD-kandidat som har vært sterkt involvert i flere borgerpaneler i andre norske kommuner enn den som har blitt beskrevet så langt.

\newpage

\hypertarget{borgerpanel-sett-fra-kommuneadministrasjonen}{%
\subsection{Borgerpanel sett fra kommuneadministrasjonen}\label{borgerpanel-sett-fra-kommuneadministrasjonen}}

\begin{quote}
Pål Bjørseth
\end{quote}

\includegraphics{figs/1x/Kommuneperspektivet.png}
Dette kapitlet handler om perspektiver fra kommuneadministrasjonen om medvirkning generelt, og borgerpanel spesielt. Selv om Bergen kommune har parlamentarisk styreform, er de aller fleste betraktninger relevante for kommuneadministrasjonen uavhengig av kommunens styreform. Andre deler av kommunens helhetlige lokaldemokratiarbeid omtales også for å kunne gi en bredere kontekst for hvilken funksjon borgerpaneler kan ha i det helhetlige medvirkningsarbeidet.

Dette kapitlet ble skrevet en måned før overleveringen av sluttrapporten i forskningsprosjektet. Etter at forskningsprosjektet er avsluttet, vil byrådet oversende sluttrapporten til bystyret med anbefaling om videre bruk og oppfølging. Følgelig er det ikke mulig å forskuttere disse anbefalingene ennå, og dette kapitlet vil derfor legge vekt på ulike perspektiver om bruk av borgerpaneler i kommunens medvirkningsapparat, og mer strukturelle grep som Bergen kommune har jobbet med for at medvirkningsverktøyene (f.eks. borgerpanel) skal fungere best mulig. Her er sentrale elementer en gjennomgående kunnskapsbasert tilnærming til medvirkning og at det bygges og deles kompetanse både på tvers av enheter i kommunen og mellom kommuner.

Som en del av kommunens fire roller som demokratisk arena, tjenesteyter, samfunnsutvikler og myndighetsutøver har medvirkning en svært viktig funksjon i lokaldemokratiet. Dette gjenspeiles i kommunelovens formålsparagraf (§ 1-1), som sier at «Loven skal legge til rette for det lokale folkestyret og et sterkt og representativt lokaldemokrati med aktiv innbyggerdeltakelse».

Lokaldemokratiet er i stadig utvikling, og det setter krav til at også medvirkningen må utvikle seg. Selv om «gullstandarden» og den viktigste temperaturmåleren på lokaldemokratiet er deltakelse i (frie) valg, er tilrettelegging for innbyggernes mulighet til deltakelse og innflytelse en oppgave som kommunene må utføre hver eneste dag mellom hvert valg. Det kan også argumenteres for at jo lavere valgdeltakelsen er, desto viktigere blir det med mobiliserende og samskapende medvirkning.

Hvordan passer borgerpaneler inn i det helhetlige arbeidet med medvirkning i Bergen kommune og blant øvrige medvirkningsmetoder? Og hvilke erfaringer har kommunen gjort seg, som kan være nyttige for andre kommuner? Det første og kanskje viktigste poenget er det grunnleggende positive i å ha gjennomført et prosjekt der vi ser på måter å videreutvikle lokaldemokratiet på gjennom en tidligere uprøvd medvirkningsmetodikk i Norge. Følgende sitat fra sluttrapporten til lokaldemokratiutvalget i 2017 (\protect\hyperlink{ref-fimreitebyen}{Fimreite 2018}) beskriver ansvaret kommunene har, på en utmerket måte:

\begin{quote}
At mye fungerer og er bra, kan ikke brukes som en sovepute for ikke å søke etter nye og innovative løsninger. Demokratiet vårt er et levende fenomen som må hegnes om og stadig fornyes.
\end{quote}

Det er fristende å omtale deliberative borgerpaneler som en parallell til at deltakelse i lokalvalg er «gullstandarden» for lokaldemokratisk deltakelse, siden disse borgerpanelene har noen særlige kvaliteter innenfor medvirkning. Dette handler blant annet om at et tilfeldig uttrukket utvalg som inviteres til å si sin mening, kan gi mer representative svar enn i åpne medvirkningsprosesser. Det andre hovedtrekket er at panelene gjennomføres som en mer opplyst diskusjon og samtale enn f.eks. ved en spørreundersøkelse. At borgerpanelene kan bidra til en mer opplyst samfunnsdebatt, er også et svært positivt trekk. Samtidig er ikke borgerpaneler den eneste formen for innbyggerinvolvering som kommunen kan holde på med. Andre ganger kan det være helt nødvendig å innrette medvirkning som et mobiliseringstiltak mot spesielt underrepresenterte grupper, eller at medvirkningen krever dyp samskaping med ekspertaktører utenfor kommuneadministrasjonen.

I en parlamentarisk styreform er de politiske premissene enda mer sentrale for administrasjonens faglige arbeid enn i formannskapsmodellen. I korte trekk er årsaken at byrådet erstatter kommunedirektøren som leder av kommunens administrasjon, og at byrådet innstiller i saker som skal behandles av bystyret (kommunestyret). Siden dette kapitlet skal beskrive perspektiver fra kommuneadministrasjonen, er ikke dette mulig i et parlamentarisk styringssystem uten en viss politisk kontekst.

Høsten 2015 var det byrådsskifte i Bergen, og i den politiske plattformen til det påtroppende byrådet Schjelderup ble det lansert en lokaldemokratireform med mål om økt demokrati og mer tverrfaglighet. Som del av kunnskapsgrunnlaget for arbeidet ble det utnevnt et bredt sammensatt utvalg ledet av professor Anne Lise Fimreite ved Universitetet i Bergen. Utvalgets mandat var blant annet å analysere den aktuelle situasjonen for lokaldemokratiet i Bergen. Utvalget skulle også gi mer konkrete innspill til arbeidet med lokaldemokratireformen i Bergen og komme med anbefalinger knyttet til etablering av bydelsutvalg (kommunedelsutvalg). I tillegg var det ønskelig at utvalget skulle jobbe med såpass omfattende temaer som hvordan man skal velge lokale representanter, politisk struktur sentralt i kommunen, representasjon av underrepresenterte grupper, tverrfaglig samordning, innspill fra befolkningen, erfaring fra andre byer og evaluering av den bergenske parlamentarismen.

Gjennom sluttrapporten fra 2017 konkluder lokaldemokratiutvalget med at lokaldemokratiet i Bergen ikke er i krise, men det pekes på at det har visse utfordringer, bl.a. at det er visse skjevheter i valgdeltakelse mellom bydelene. Rapporten sier videre «men at det ikke er krise og utmattelse i lokaldemokratiet, er ikke det samme som helt å friskmelde systemet. At mye fungerer og er bra, kan ikke brukes som en sovepute for ikke å søke etter nye og innovative løsninger. Demokratiet vårt er et levende fenomen som må hegnes om og stadig fornyes.»
Sluttrapporten fra lokaldemokratiutvalget har tre anbefalinger. Den første er knyttet til innretning av bydelsutvalg, mens de to siste er som følger:

\begin{itemize}
\item
  Vurdere å opprette et byborgerpanel: Utvalget anbefaler at det utredes en modell hvor tilfeldig uttrukne bergensere får mulighet til reell politisk påvirkning i enkeltsaker som byrådet definerer. Vårt forslag er å gjennomføre dette som et forskningsprosjekt (\ldots).
\item
  Uavhengig av modell: Holde fokus på de til dels store variasjonene i politisk deltakelse blant innbyggerne som er avdekket i rapporten.
\item
  Være bevisst på valgdeltakelsens geografiske skjevhet og hvilke konsekvenser det kan ha for politikkutforming, samt deltakelsen blant yngre og grupper uten høyere utdanning. Tilrettelegge for frie nedenfra-opp, ikke-kommersielle initiativer fra borgere, som fremmer fellesskap, omfavner engasjementet og senker terskelen for medvirkning i byrommet.
\end{itemize}

Forskningsprosjektet Demovate og forsøk med borgerpaneler er i så måte en direkte oppfølging av sluttrapporten. Den politiske bestillingen ble konkretisert gjennom saken «Et styrket nærdemokrati» i Bergen bystyre (bystyresak 235/18), der ett av vedtakspunktene er at «Bystyret ber om at byborgerpanel prøves videre ut som et forskningssamarbeid der det også søkes eksterne forskningsmidler.»

Vi er derfor i en privilegert posisjon som kan få til et samarbeid som lager forbindelseslinjer mellom kommunesektoren og akademia i dette prosjektet, og på den måten bidra til demokratisk innovasjon. Men et prosjekt varer ikke evig, og det er avgjørende å finne metoder for å videreføre de gode erfaringene fra dette prosjektet. Det gjelder både på forskningssiden og --- ikke minst -\/-- i praksisfeltet. Her er håndboken et grep for å mangfoldiggjøre våre erfaringer med oppsett, gjennomføring og funn fra borgerpaneler. Målet er at dette ikke bare skal være nyttig for andre enheter i den kommunen som vurderer å gjennomføre et borgerpanel, men at det også kommer til nytte for andre kommuner. Helt generelt bidrar dokumentasjon av gjennomføring av god praksis og beste praksis til at det blir lettere å dele erfaringer om medvirkning. En slik delekultur både internt og eksternt er i seg selv et godt eksempel på demokratisk innovasjon.

Kommunens arbeid på lokaldemokratifeltet har grovt sett fulgt to hovedspor. Det første sporet er videre arbeid med bydelsutvalg. Det andre hovedsporet er det som generelt kan omtales som «andre former for innbyggerinvolvering». Forsøk med borgerpanel er et konkret eksempel på en medvirkningsmetode som Bergen kommune har prøvd ut. Før vi går mer konkret inn på borgerpanel, er det nødvendig å beskrive det helhetlige arbeidet Bergen kommune gjør på medvirkningsfeltet, og som er sammenhengen borgerpanel inngår i.

Våren 2022 ble Bergen kommune tatt opp i et nasjonalt program for lokaldemokratiet i foregangskommuner, i regi av Kommunal- og distriktsdepartementet. I søknaden som bystyret behandlet i 2021, pekes det på at Bergen kommune gjennomfører en rekke involveringsaktiviteter som gir stor merverdi for det enkelte prosjektet hvor medvirkningsaktiviteten foregår, men det finnes et utviklingspotensial i å utnytte overføringsverdien til andre prosjekter i kommunen, få synergier mellom prosjekter og å kunne dele beste praksis med andre kommuner. Her er Demovate et godt eksempel på medvirkningsaktivitet som faktisk dokumenteres for å gis overføringsverdi.

Kommunen jobber nå med å ta et helhetsgrep i arbeidet med å videreutvikle byens innbyggerinvolvering. Dette krever koordinert innsats internt og at vi legger til rette for samarbeid med innbyggerne og andre aktører. Gitt størrelsen og kompleksiteten i Bergen kommunes organisasjon må vi sikre at eksisterende og nye initiativer ikke er enkeltstående prosjekter som er enten utydelige eller ukjente for andre deler av organisasjonen. Bergen kommune vil derfor bruke foregangskommunestatusen til å etablere et konsept med en helhetlig organisering av det lokaldemokratifaglige arbeidet som allerede gjøres i de ulike avdelingene og enhetene, og som kan utgjøre en faglig plattform for samordning og videreutvikling på området.

Formålet skal blant annet være å

\begin{itemize}
\tightlist
\item
  sikre god flyt av informasjon internt i organisasjonen\\
\item
  få en systematisk gjennomgang av kommunens nåværende tiltak\\
\item
  sette kommunens innbyggerinvolveringsprosjekter i system for å gi synergier mellom pågående prosjekter\\
\item
  få god oversikt over relevante nasjonale og internasjonale prosjekter\\
\item
  dele erfaringer og videreformidle beste praksis\\
\item
  legge til rette for kunnskapsdrevet utvikling på området\\
\item
  videreutvikle delingskultur, bl.a. gjennom å etablere en digital kunnskapsbank som skal være åpent tilgjengelig også for andre kommuner
  koordinere nettverksaktivitet med andre kommuner m.m.\\
\item
  identifisere aktuelle fremtidige prosjekter, initiativer og samarbeid
\end{itemize}

Selv om Bergen kommune begrunner behovet for denne systematikken med størrelsen og kompleksiteten i kommunens organisasjon, vil en grunnleggende systematikk i bunn være en forutsetning for å drive samordnet kunnskapsutvikling, uavhengig av kommunens størrelse.

Et annet sentralt tema som Bergen kommune løfter i det nasjonale prosjektet, er koblingen mellom innbyggerinvolvering og politisk organisering. Her pekes det både på formålsparagrafen i kommuneloven og bærekraftdelmål 16.7 om å «sikre lydhøre, inkluderende, deltakelsesbaserte og representative beslutningsprosesser på alle nivåer». Dette har en klar parallell til synet kommunen har på borgerpanelene.

Innbyggerinvolvering må skje på reelle arenaer for involvering og medvirkning, ikke bare være et avkrysningspunkt på en liste over aktiviteter som skal gjennomføres i f.eks. en planprosess. Involveringen bør derfor innrettes mot de delene av prosessen hvor det er lettest å implementere innspillene fra aktører utenfor kommuneorganisasjonen. Oppfølgingen av involveringen må være tillitsfull f.eks. gjennom de grepene som praktiseres for kommunens samfunnsplanlegging, og ved at innspillene blir forsvarlig utredet i saker som legges frem for folkevalgte organer.

Innen samfunnsplanleggingen har byrådet vedtatt en egen planinstruks som er gjeldende for all planlegging som krever politiske vedtak, men som ikke er arealrelatert (der gjelder egne prosessregler, blant annet i lys av plan- og bygningsloven). Planinstruksen slår blant annet fast at bred medvirkning er viktig for å sikre forsvarlige utredninger, og at det bidrar til gode planer og et best mulig grunnlag før beslutninger blir tatt. Vi benytter ekstern medvirkning for at offentlige myndigheter, innbyggere, organisasjoner og næringsliv skal kunne si sin mening. Ekstern medvirkning kan organiseres gjennom høringer, innspillseminar, dialogmøter og lignende.

Alle planer bør som hovedregel på høring, og høring skal som hovedregel være åpen for alle. Etter høringen skal høringsinnspillene oppsummeres i et merknadsskjema som skal følge den politiske saken som vedlegg. Skjemaet skal blant annet vise hvordan innspillet er vurdert, og hvordan innspill som er tatt helt eller delvis til følge, er innarbeidet i et nytt planforslag.

Planinstruksen legger altså føringer både for at planer skal legges ut på høring, og for at innspill som er gitt gjennom medvirkningen, skal vurderes i sin helhet, og at vurderingene er synlige i oppfølgingen av saken frem til endelig behandling i folkevalgt organ. Dette gir grunnlag for en forsvarlig utredning av innspill i tråd med kommuneloven § 13-1, samtidig som medvirkningen blir transparent, og dermed kan innspillene også vurderes (potensielt annerledes) av bystyret.

Byrådet har lagt vekt på dette perspektivet i en egen byrådssak som tar opp bruk av borgerpanel i utvalgte politiske prosesser i Demovate-prosjektet. I byrådsleders vurdering og anbefaling/konklusjon pekes det på følgende: Byrådsleder mener at bilfrie soner og arealstrategi for Dokken kan være velegnede som diskusjonstemaer i borgerpanelet, og ønsker å synliggjøre panelets innspill i den videre faglige og politiske oppfølgingen. Byrådet vil ta stilling til rådene/innspillene fra borgerpanelet i det politiske saksfremlegget som følger disse to casene, og innspillene skal være selvstendige vedlegg i saksgrunnlaget frem mot endelig politisk behandling. Dette vil etter byrådsleders mening gi borgerpanelet høy legitimitet, samtidig som det sikrer at innspillene fra panelet blir forsvarlig utredet iht. til kravene i kommuneloven.

I tillegg til begrunnelsene fra byrådssaken er det av betydning at medvirkningen skjer på et tidspunkt i de politiske prosessene hvor det ikke allerede er konkludert på bakgrunn av en lengre utredningsprosess og/eller politisk vurderinger. Dette er i tråd med poenget om at involveringen bør innrettes mot de delene av prosessen hvor det er lettest å implementere innspillene fra aktører utenfor kommuneorganisasjonen.

Det siste poenget i dette kapitlet er at prosjektet ga oss innsikt i personvernutfordringer knyttet til digital gjennomføring av borgerpanel. Selv om planen i utgangspunktet var fysisk gjennomføring, var dette ikke mulig pga. pandemien. Problemstillingene om personvern er omtalt i andre kapitler i denne håndboken. Grunnen til at det trekkes frem her, er at dette ikke er en prosjektspesifikk problemstilling. Digitalisering gir store muligheter for å bruke nye verktøy som skal gi økt medvirkning. Bergen kommune har pekt på at kommunen skal være en offensiv foregangskommune for digitale løsninger. Den digitale utviklingen gir muligheter for brukervennlige og effektive kommunale tjenester og styrking av åpenhet og deltakelse. Samtidig må kommunen være en ansvarlig innovatør, der vi blant annet sikrer innbyggernes rettigheter ved bruk av nye verktøy, spesielt knyttet til informasjonssikkerhet og personvern.

Dette perspektivet omtaler Bergen kommune i søknaden om å bli tatt opp i det nasjonale programmet for lokaldemokratiet i foregangskommuner. Det har tidligere blitt skrevet om hvordan beste praksis for metodevalg og gjennomføring av medvirkningsaktivitet bør dokumenteres og deles i større grad, både internt og med andre kommuner. Dette kan også gjelde hvordan nye løsninger og prosjekter vurderes opp mot personvernforordningen og kommunens krav til informasjonssikkerhet. På denne måten vil kommunen bygge en kultur for deling av systematikk knyttet til håndtering av nye og innovative forslag på medvirkningsområdet, som bidrar til at prosjektene kan prøves ut på en forsvarlig måte. En slik tilnærming skal innby til tillit i det enkelte prosjektet og gjøre det enklere å gjennomføre tilsvarende prosesser på andre områder. Det senker i tillegg terskelen for å ta i bruk løsninger som er testet i denne kommunen, for andre kommuner, siden de både kan ta i bruk metodikk for den praktiske gjennomføringen og ta utgangspunkt i kommunes vurderinger mht. forsvarlig forvaltning.

Avslutningsvis vil vi takke for anledningen til å delta i et forskningsbasert og samarbeidsorientert prosjekt mellom Bergen kommune og forskningssektoren ved spesielt NORCE og Universitetet i Bergen. Prosjektet har vært et godt eksempel på hvordan vi gjennom samarbeid «tetter gapet» mellom forskning og praksisfeltet. Slike forbindelseslinjer er nødvendige for å bygge kunnskapskommunen i bred forstand, og er i så måte ikke bare et konkret eksempel på videreutvikling av demokratisk innovasjon gjennom borgerpaneler. Prosjektet representerer derfor en mer generell tilnærming til å finne merverdien i skjæringspunktet mellom forskning og praksis.

\newpage

\hypertarget{andre}{%
\subsection{Erfaringer fra andre norske kommuner}\label{andre}}

\includegraphics{figs/1x/Andre_kommuner.png}
Som vi har vært inne på tidligere, så ser vi en klar definisjonsavgrensing av hva et borgerpanel er. Et borgerpanel er altså en familie av demokratiske innovasjoner, som har to kjerneelementer:

\begin{itemize}
\tightlist
\item
  tilfeldig utvelgelse av deltakere gjennom loddtrekning\\
\item
  at disse deltakerne gjennomgår en deliberativ prosess
\end{itemize}

Dersom vi bruker denne definisjonen, ser vi at det har blitt gjennomført minst 17 borgerpaneler i Norge per i dag. Disse kan ses i tabell \ref{tab:tbl-borgerpanel}.

\begin{table}[!h]

\caption{\label{tab:tbl-borgerpanel}Borgerpanel i Norge}
\centering
\resizebox{\linewidth}{!}{
\begin{tabular}[t]{lllll}
\toprule
\textbf{Tittel} & \textbf{Sted} & \textbf{Type} & \textbf{År} & \textbf{Deltakerantall}\\
\midrule
\cellcolor{gray!6}{Lekfolkskonferanse om genmodifisert mat} & \cellcolor{gray!6}{Nasjonal} & \cellcolor{gray!6}{Lekfolkskonferanse} & \cellcolor{gray!6}{1996} & \cellcolor{gray!6}{16}\\
 
Innbyggerhøring Nordland & Nordland & Deliberativ meningsmåling & 1998 & 83\\
 
\cellcolor{gray!6}{Innbyggerhøring Kongsberg} & \cellcolor{gray!6}{Kongsberg} & \cellcolor{gray!6}{Deliberativ meningsmåling} & \cellcolor{gray!6}{1998} & \cellcolor{gray!6}{Ukjent}\\
 
Lekfolkskonferanse om genmodifisert mat & Nasjonal & Lekfolkskonferanse & 2000 & 15\\
 
\cellcolor{gray!6}{Lekfolkskonferanse om stamceller og terapeutisk kloning} & \cellcolor{gray!6}{Nasjonal} & \cellcolor{gray!6}{Lekfolkskonferanse} & \cellcolor{gray!6}{2001} & \cellcolor{gray!6}{15}\\
 
Borgerpanelet for Tempe & Trondheim & Lekfolkskonferanse & 2004 & 14\\
 
\cellcolor{gray!6}{Byborgerpanelet} & \cellcolor{gray!6}{Bergen} & \cellcolor{gray!6}{Deliberativ meningsmåling} & \cellcolor{gray!6}{2018} & \cellcolor{gray!6}{76}\\
 
Borgerkraft & Trondheim & Borgerjury & 2020 & 16\\
 
\cellcolor{gray!6}{Deliberativ meningsmåling i Bergen} & \cellcolor{gray!6}{Bergen} & \cellcolor{gray!6}{Deliberativ meningsmåling} & \cellcolor{gray!6}{2021} & \cellcolor{gray!6}{90}\\
 
Borgerpanel på Grefsen-Kjelsås om overvann & Oslo & Borgerjury & 2021 & 20\\
 
\cellcolor{gray!6}{Borgerpanel på Romsås} & \cellcolor{gray!6}{Oslo} & \cellcolor{gray!6}{Borgerjury} & \cellcolor{gray!6}{2021} & \cellcolor{gray!6}{14}\\
 
Trondheimspanelet & Trondheim & Borgerjury & 2021-2022 & 50\\
 
\cellcolor{gray!6}{Ungt Borgerpanel} & \cellcolor{gray!6}{Stavanger} & \cellcolor{gray!6}{Borgerjury} & \cellcolor{gray!6}{2021-2022} & \cellcolor{gray!6}{21}\\
 
Borgerpanel om revisjon av småhusplanen & Oslo & Borgerjury & 2021-2022 & 21\\
 
\cellcolor{gray!6}{Medarbeiderpanelet i Sadexxo} & \cellcolor{gray!6}{Bedrift} & \cellcolor{gray!6}{Borgerjury} & \cellcolor{gray!6}{2022} & \cellcolor{gray!6}{15}\\
 
Deliberativ meningsmåling om kunstig intelligens & Nasjonal & Deliberativ meningsmåling & 2022 & 218\\
 
\cellcolor{gray!6}{Røros borgerpanel} & \cellcolor{gray!6}{Røros} & \cellcolor{gray!6}{Borgerjury} & \cellcolor{gray!6}{2023-2025} & \cellcolor{gray!6}{16-30}\\
\bottomrule
\end{tabular}}
\end{table}

Vi kan dele bruken av borgerpaneler inn i to perioder. Den første perioden varte fra sent på 1990-tallet til tidlig på 2000-tallet. De fleste borgerpanelene som ble gjennomført på denne tiden, var det som kalles lekfolkskonferanser og deliberative meningsmålinger. Vi kan derfor si at eksperimenteringen i den første perioden i stor grad var drevet av inspirasjon fra Danmark med det danske teknologirådet og fra USA med James Fishkins modell for deliberative meningsmålinger.

I den andre perioden, fra 2018, har det vært en økning i bruken av borgerpaneler. Økningen i Norge i den senere tid kan ses i sammenheng med økningen internasjonalt. Internasjonalt har dette blitt kalt den «deliberative bølgen» (\protect\hyperlink{ref-oecd_innovative_2020}{OECD 2020}). Hvor stor denne bølgen er, kan diskuteres, men trenden med økende bruk av borgerpaneler har vi sett også her i Norge. Det er spesielt borgerjurymodellen som har stått for denne økningen, da halvparten av alle borgerpaneler i Norge har vært av denne typen.

Vi ser også at borgerpaneler i Norge har vært mest brukt på kommunenivå, og at det bare har vært noen få på nasjonalt nivå. De nasjonale er lekfolkskonferanser som ble arrangert av det norske Teknologirådet. Blant kommunene ser man at borgerpaneler er mest blitt brukt av bykommuner i Norge. Vi ser også at det har vært gjennomført et medarbeiderpanel av Sadexxo.

I dette kapitlet skal vi se på noen av disse borgerpanelene. Disse vil være to av borgerpanelene i Trondheim, kalt \emph{Borgerkraft} og \emph{Trondheimspanelet}, ett i Oslo, borgerpanelet i Stavanger og det permanente borgerpanelet på Røros.

\hypertarget{borgerpanelene-i-trondheim-oslo-og-stavanger}{%
\subsubsection{Borgerpanelene i Trondheim, Oslo, og Stavanger}\label{borgerpanelene-i-trondheim-oslo-og-stavanger}}

\hypertarget{trondheim-borgerkraft}{%
\paragraph{Trondheim -- Borgerkraft}\label{trondheim-borgerkraft}}

I februar 2020 troppet 16 tilfeldig uttrukkede personer fra Trondheim-Sør opp i Erkebispegården i Trondheim. De ble servert mat og fikk en omvisning i selve Erkebispegården. Selv om Trondheim hadde eksperimentert med en lekfolkskonferanse på 2000-tallet, kan vi godt si at borgerpanelet i 2020 var noe nytt for Trondheim kommune. Det er derfor på sin plass å si litt om bakgrunnen for prosjektet.

I mai 2019 ble den nye strategien for medvirkning og samskaping stemt frem av bystyret i Trondheim. I den stod det blant annet at «Bystyret ønsker å øke innbyggerinvolveringen og peker spesielt på muligheten for at det systematisk settes av penger til lokale prosjekter som i stor grad er innbygger- eller brukerstyrt, eller innbyggerpanel.» (\protect\hyperlink{ref-trondheim_kommune_den_2019}{kommune 2019}). Med denne strategien ble det jobbet nøye med å bygge opp kunnskap om borgerpaneler. Kommunen hadde etablert et nært samarbeid med forskere på NTNU gjennom «Universitetskommunen», og man etablerte også et nært samarbeid med internasjonale aktører, som OECD, MASS LBP i Canada, Involve i Storbritannia og G1000 i Belgia, for å nevne noen.
Dette la grunnlaget for arbeidet med borgerpaneler i Trondheim, og de har per i dag gjennomført to. Det første, som startet rett før pandemien, var et samarbeid mellom Trondheim kommune, NTNU og SoCentral. Dette prosjektet ble kalt «Borgerkraft» og var et borgerpanel som skulle ta for seg hva bærekraft betydde lokalt i en bydel, og hvordan kommunen bør støtte gode initiativer. Prosjektet var en liten blanding av to prosesser. Den ene delen gikk ut på å mobilisere innbyggere lokalt til å komme med forskjellige initiativer om hvordan Trondheim kunne bli mer bærekraftig. Den andre delen var borgerpanelet, som skulle diskutere mer om hvordan Trondheim kommune kunne vurdere disse forskjellige initiativene, og hvordan man skulle bestemme hvem som skulle få midler. Borgerkraft hadde derfor en blanding av to demokratiske innovasjoner: borgerpanel og elementer av deltakende budsjettering.

Siden prosessen var avgrenset til de sørlige bydelene, ble den kalt \emph{Trondheim Sør}. I borgerpanelet var det 16 deltakere som var trukket ut gjennom en loddtrekning. Måten selve loddtrekningen ble gjort på var den samme som ble gjennomført i de andre borgerpanelene (utenom Stavanger, som vi skal se senere). Metoden som ble brukt, var et såkalt «borgerlotteri», som er blitt standarden for å gjennomføre gode loddtrekninger internasjonalt, der denne prosessen ofte blir kalt \emph{Civic Lottery} eller en \emph{two-stage random selection} (\protect\hyperlink{ref-mass_lbp_how_2017}{LBP 2017}; \protect\hyperlink{ref-oecd_innovative_2020}{OECD 2020}). Her ble det sendt ut 800 brev til tilfeldige personer i «Trondheim Sør». Blant de 800 svarte rundt 100 personer ja til å delta. Deretter ble det gjennomført en ny loddtrekning blant de som hadde sagt ja, men da etter kategoriene alder, kjønn og geografi, for å få panelet til å speile \emph{Trondheim Sør} så godt det lot seg gjøre. I Trondheim diskuterte man lenge om man skulle trekke ut en liste fra Folkeregisteret. Til slutt ble det bestemt at man ikke skulle gjøre dette, men heller sende ut invitasjoner til tilfeldige adresser hentet fra en database fra Posten. To elementer påvirket dette valget. For det første var det enklere, siden det å trekke ut fra Folkeregisteret kan være en lang prosess. Og for det andre, som også er den viktigste grunnen, så kunne man ende opp med å ekskludere personer som bodde i området, men som ikke har sin folkeregistrerte adresse der, dersom man brukte Folkeregisteret. Dette er spesielt relevant for en studentby som Trondheim, der mange studenter ikke er folkeregistrert i nettopp Trondheim.

I Borgerkraft var tanken at borgerpanelet skulle møtes tre ganger med mulighet for en fjerde gang. Selve prosessen fulgte noenlunde den samme oppskriften som de andre borgerpanelene ofte har fulgt. Dvs. en lærings- og kunnskapsfase, en refleksjons- og diskusjonsfase og til slutt en beslutningsfase.

I det første møtet skulle man bli orientert om prosessen, bli kjent med hverandre og få læring og kunnskap. Lærings- og kunnskapsfasen ville dreie seg om begrepet bærekraft, og tanken her var først å ta for seg disse store temaene, for så å bearbeide dem på et lokalt nivå. I det andre møtet skulle man dykke ned i tematikken og drøfte de forskjellige elementene rundt bærekraft og hva som var viktige elementer for de som deltok. Det tredje møtet skulle man fortsette, før man til slutt skulle trekke sammen alle trådene og komme med sin anbefaling, først om hva som var viktig for en bærekraftig utvikling for distriktene i «Trondheim Sør», og ut fra dette sette opp kriterier som kommunen kunne bruke for å evaluere forskjellige søknader om økonomisk støtte.

Når vi ser tilbake til definisjonen på borgerpaneler, så ser vi at det er fasiliterte gruppesamtaler. Fasiliteringskompetanse er derfor sentralt i et borgerpanel. Spørsmålet i alle borgerpanelene blir derfor hvordan man løser dette. Vi kan si at dette kan løses på tre mulige måter. Man kan løse det innad ved at man finner kompetansen innad i organisasjonen. Man kan løse det eksternt ved at man henter inn kompetanse utenfra. Eller man kan gjøre en hybrid med en blanding mellom intern og ekstern kompetanse. I Borgerkraft ble det benyttet en blanding av kommuneansatte og personer fra organisasjonen SoCentral. Kompetansen innad i kommunen ble hentet fra flere plasser i organisasjonen, og dette var ikke nødvendigvis personer som var koblet til prosessen. Trondheim hadde også erfaringer med fasilitering fra før av gjennom deres arbeid med \emph{SteinSaksPapir}-prosjektet. Denne kompetansen ble dermed brukt også i Borgerkraft.

Borgerkraft-prosjektet ble derimot ikke slik som planlagt. Det første møtet fant sted i februar 2020, og på grunn av pandemien ble det gjort helt om på prosjektet. Det ble satt på vent for en lengre periode, for å se an utviklingen, før man bestemte seg for å gå videre og fullføre panelet digitalt. Det ble derfor satt opp noen flere digitale møter, og man valgte også å dele gruppen i to. Dette ble gjort fordi man da kunne holde et digitalt møte med 8 personer i stedet for at alle 16 skulle være til stede digitalt på samme tidspunkt. Dette ble også gjort for å kunne være fleksible med tanke på datoene, siden det var vanskelig å finne datoer som passet for alle. Disse møtene ble satt opp i mai 2020, en stund etter at det siste møtet egentlig var tenkt å avholdes. Borgerkraft måtte derfor endres fra å være et fysisk borgerpanel til å bli et digitalt ett. Siden endringene ikke var planlagte, var det ikke nok tid og ressurser tilgjengelig til å legge tilstrekkelig til rette for denne endringen. Man kan også se at dette hadde innvirkning på hvor inkluderende deliberasjonen var i prosessen. Det var noen som hadde ustabilt internett, og noen hadde ikke mulighet til å gå et sted der de kunne konsentrere seg om borgerpanelet uten å bli forstyrret. Noen eldre strevde også med formatet, siden et slikt format ikke var like intuitivt for dem som for andre. I Borgerkraft kan man derfor se antydninger til at digital deliberasjon kan forsterke ulikheter i samfunnet inn i deliberasjonen. Det kan favorisere personer som har stabilt internett, de som har et eget rom de kan arbeide i, og de som er vant til å ha digitale møter.

Pandemien hadde derfor stor innvirkning på selve borgerpanelet. Det ble en lang pause mellom de forskjellige møtene siden man ikke helt visste om man kunne samles igjen våren 2020. Når man så at dette kom til å ta en stund, brukte man mye tid og ressurser på å prøve å finne ut hvordan man kunne gjennomføre det digitalt. Pausen hadde imidlertid stor innvirkning på deliberasjonen, siden det tok tid før deltakerne kom inn i prosessen igjen etter en så lang pause.

Det er også viktig å merke seg at spørsmålet og mandatet til dette borgerpanelet ikke var helt klart, og dette hadde også innvirkning på selve prosessen. Her erfarte man at rollen til borgerpanelet ikke var godt nok avklart, og dette kunne man ha vært flinkere til å formidle -- f.eks. hva vil skje med det som borgerpanelet produserer, og hvor vil man med dette? I begynnelsen var ikke de som deltok, helt sikre på hvilken rolle de hadde, og hva formålet med borgerpanelet var, noe dette sitatet fra en deltaker illustrerer:

\begin{quote}
Det tok litt tid før jeg forstod hva som var forventet av meg, og hva som var målet med møtene.
\end{quote}

Å vise tydelig helt fra starten av hvilken prosess dette er knyttet til, og hvilken rolle borgerpanelet hadde i denne prosessen, ble derfor et av de viktigste læringspunktene som kom ut fra Borgerkraft. I ettertid ser vi at andre borgerpaneler har lært av dette.
Anbefalingen til borgerpanelet ble tatt med videre inn i arbeidet, og mye av det som kom ut av borgerpanelet, blir nå innarbeidet i et nytt system for tildeling av midler til bærekraftsprosjekter.

Selv om Borgerkraft hadde utfordringer, var det samtidig et nyttig eksperiment for Trondheim kommune. Eksperimentet gjorde at kommunen ble komfortabel med denne medvirkningsmetoden, og man fikk viktig erfaring med både fasilitering og hvordan man i det hele tatt designet et borgerpanel. Erfaringen rundt Borgerkraft ble derfor viktig for det videre arbeidet med borgerpaneler, ikke bare i Trondheim, men også i Oslo og Stavanger.

\hypertarget{trondheim-trondheimpanelet}{%
\paragraph{Trondheim -- Trondheimpanelet}\label{trondheim-trondheimpanelet}}

I 2021--2022 gjennomførte Trondheim kommune et borgerpanel som skulle ta for seg den nye samfunnsdelen av kommuneplanen. Problemstillingene som ble gitt til panelet, var

\begin{quote}
Hvordan passer det gode liv i et Trondheimssamfunn inn i diskusjonen om planetens tåleevne og miljø- og klimadebatten? Hvilke smarte grep må vi gjøre fremover?
\end{quote}

50 personer ble trukket ut til å delta i prosessen, noe som gjorde den til en av de større borgerpanelene vi har hatt i Norge. Også her ble «borgerlotteri»-metodikken fulgt. Først ble det sendt ut 4000 tekstmeldinger til tilfeldige personer i Trondheim. Av disse meldte rundt 400 seg til å være med. Blant disse 400 ble det foretatt en ny loddtrekning, men denne gangen etter kategoriene kjønn, alder, geografi og utdanning for at Trondheims befolkning skulle speiles så godt som mulig.

Borgerpanelet ble gjennomført under pandemien. Selv om det på det tidspunktet var lov å samle 50 deltakere, gjorde pandemien det vanskelig for noen å delta. Av hensyn til blant annet personer i helsevesenet, studenter og de som naturlig nok var bekymret for covid-19, la man til rette for at man kunne delta digitalt i hele eller deler av prosessen om man ville. Andre former for tilrettelegging ble også gjort, da spesielt med tanke på transport og tolketjeneste.

Trondheimspanelet var linket opp til en større prosess med flere former for medvirkning og representasjon involvert. I prosessen med å lage en ny samfunnsdel av kommuneplanen gjennomførte Trondheim kommune tre medvirkningsfaser, der Trondheimspanelet var en av disse. Dette kan man se nedenfor:

\begin{figure}

{\centering \includegraphics[width=0.8\linewidth]{figs/trondheimspanelet} 

}

\caption{Medvirkningsprosess}\label{fig:unnamed-chunk-4}
\end{figure}

I første fase gjennomførte de et såkalt innsiktsarbeid. Her gjennomførte Trondheim kommune en hel rekke av ulike typer aktiviteter, hvor de benyttet biblioteker, arrangementer, workshops, spørreundersøkelser, intervjuer, ønsketrær osv. Gjennom denne prosessen fikk de til sammen inn 2000 ulike typer innspill. Dette var dermed en veldig bred medvirkningsprosess for å få så mange stemmer som mulig inn i prosessen med å lage en ny samfunnsdel.

Den andre fasen var borgerpanelet. Dette panelet ble designet rundt den samme idéen som borgerpanelene vanligvis gjør, med læring og kunnskap, refleksjon og diskusjon og til slutt en anbefaling. Innspillene fra den første medvirkningsfasen ble dokumentert og sortert, og dette ble en ressurs for Trondheimspanelet. Selve borgerpanelet var delt opp i fem samlinger. Den første samlingen tok for seg problemstillingen, og man lærte og fikk kunnskap om tematikken gjennom innspillene og gjennom de forskjellige presentasjonene. Takket være læringen fra det forrige borgerpanelet brukte man tid på å gjøre helt klart hvilket mandat Trondheimspanelet hadde, på den første samlingen. Trondheimspanelet var også linket til en planprosess i kommunen, noe som gjorde at rollen til panelet var enklere å forklare og vise til deltakerne. På den andre og tredje samlingen fortsatte deltakerne med å utveksle idéer om hvordan man vil ha det i Trondheim. Så veide man de forskjellige idéene opp mot hverandre og diskuterte hvordan man skulle oppnå det man ønsket. Den fjerde samlingen gikk ut på å sammenfatte alt dette i konkrete anbefalinger om hvilken by Trondheim skal være. Den femte samlingen var nokså unik. Her var borgerpanelet og formannskapet i kommunen samlet, og de diskuterte anbefalingene og dilemmaene fra borgerpanelet i fellesskap.

Trondheimspanelet bygde videre på arbeidet fra Borgerkraft, spesielt med tanke på fasiliteringen. I dette panelet ble all fasilitering løst innad i kommunen, og fasilitatorene som var med i Borgerkraft-prosjektet, hadde en sentral rolle i Trondheimspanelet. Slik sett har Trondheim kommune jobbet med å bygge opp en kompetanse rundt denne metodikken.

I den tredje medvirkningsfasen ble Trondheimspanelets anbefalinger tatt inn i et strategidokument, som ble sendt på en høringsrunde. Her fikk de innspill fra ulike aktører og bydelsgrupper.

Den nye samfunnsdelen av kommuneplanen har nå blitt delt inn i tre hovedområder: et grønnere samfunn, sterkere fellesskap og teknologihovedstaden. Borgerpanelets anbefalinger har satt preg på alle områdene. For eksempel har nesten alle delmålene under «et grønnere samfunn» blitt påvirket av anbefalingene fra borgerpanelet, og kanskje spesielt delmålet om at Trondheim skal gjøre det lettere å leve miljøvennlig.

Trondheimspanelet er et eksempel på hvordan man kan linke et borgerpanel opp mot en større medvirkningsprosess. Styrken med dette er at man da får både \emph{bredde} i form av andre typer medvirkning og \emph{dybde} i form av deliberasjon i borgerpanelet. Man får også andre former for representasjon inn i prosessen ved at interesseorganisasjoner og andre aktører får komme med sine innspill i høringsrunden.
Trondheimspanelet tok for seg en svært bred problemstilling. Dette er ikke nødvendigvis et problem i seg selv, siden man vil gi borgerpanelet rom til å utforske saken grundig, men samtidig bør man ikke gi borgerpanelet et for bredt spørsmål, siden det kan gjøre arbeidet vanskelig for dem. Derfor er det viktig å gjøre en avveining mellom disse to aspektene. I designet av Trondheimspanelet ble det diskutert om man skulle spisse inn problemstillingen til å ta for seg et mer konkret dilemma og mer konkrete verdispørsmål som kunne inngå i samfunnsdelen av kommuneplanen, men man valgte heller å gå for en bredere problemstilling. En slik bred problemstilling krevde mye fra personene som ledet prosessen, og det var nødvendig med nøye planlegging for å komme «i mål».

\hypertarget{stavanger-ungt-borgerpanel}{%
\paragraph{Stavanger -- ungt borgerpanel}\label{stavanger-ungt-borgerpanel}}

I 2021 bestemte Stavanger seg for å arrangere et ungt borgerpanel. Dette var et initiativ fra organisasjonen Grønn by. Prosjektet ble til et samarbeid mellom Grønn by, Smartby Stavanger, Stavanger kommune, SoCentral og NTNU.

Det var Smartby Stavanger som eide prosessen, mens andre deler av kommunen ble koblet på ved at de var med i arbeidsgruppen for borgerpanelet. Dette var et borgerpanel som hadde som mål å ta unge seriøst i diskusjonen om planprosessen for Hillevåg. Spørsmålet de skulle ta for seg, var:

\begin{quote}
Hvordan kan vi sammen skape et trygt og attraktivt nabolag?
\end{quote}

Flere elementer gjorde dette til et annerledes prosjekt enn de andre borgerpanelene. Dette var et borgerpanel for unge, og dette krevde en noe annerledes inngang til for eksempel lærings- og kunnskapsfasen og til fasilitering i borgerpanelet. Et annet element var at dette ble koblet til to skoler som har en skolekrets som dekker Hillevåg, Kristianlyst og Ullandhaug ungdomsskole. Derfor ble det også inngått et samarbeid med disse skolene, hvor man gjennomførte fysisk loddtrekning på to av dem.

Grunnen til at man gjennomførte et ungt borgerpanel i Stavanger, var i hovedsak at de ville gi de unge en stemme inn i prosessene. Det ble også påpekt at det er kanskje de unge som bruker nabolagene mest, noe som også ble påpekt av en av deltakerne i borgerpanelet: «\ldots{} viktig at de unge blir hørt, det er vi som skal vokse opp i nabolaget og bruker nabolaget masse» (\protect\hyperlink{ref-noauthor_ungt_2022}{{``Ungt {Borgerpanel}: {`{Hvordan} Kan Vi Sammen Skape Et Trygt Og Attraktivt Nabolag?'}''} 2022}).

Et av de mest interessante elementene var at man gjennomførte loddtrekningen fysisk på to skoler. Hovedinspirasjonen til dette var læringen fra «Democracy in Practice», en organisasjon som har jobbet med skoler i Bolivia, hvor det ble brukt et design med at elevrådet ble valgt ut gjennom loddtrekning. Arbeidet til organisasjonen har også inspirert Vika VGS i Oslo, som nettopp hadde begynt med å eksperimentere med å velge ut sine representanter til elevrådet gjennom en loddtrekning i stedet for at de skulle bli stemt frem. Den fysiske loddtrekningen ble gjennomført på to av skolene. Selve loddtrekningen ble koblet opp mot en introduksjon til det deliberative demokratiet og til en presentasjon av borgerpanelet, slik at alle på skolen fikk vite hva dette gikk ut på. Man skulle velge fire representanter fra hvert kull -- to gutter og to jenter. Til sammen ville man da ende opp med 24 deltakere, siden det var tre trinn på begge skolene. Siden skolene ble brukt, ble det bestemt at man ikke skulle bruke stratifisering i loddtrekningen. Ved å trekke to gutter og to jenter fra hvert klassetrinn på de to skolene oppnådde man spredning på alder, kjønn og geografi.

Det å gjennomføre en fysisk loddtrekning hadde nokså stor symbolikk. Selvsagt er det ikke alltid at en fysisk loddtrekning er å foretrekke (for eksempel ikke dersom problemstillingen hadde tatt for seg et svært kontroversielt tema), men erfaringen fra Stavanger har vært veldig positiv. Det at man også åpner opp om prosessen rundt selve borgerlotteriet, er noe man diskuterer også i fagfeltet. Det var stor entusiasme rundt selve lotteriet, og de som ble plukket ut, fikk med seg en gullbillett og et informasjonsskriv hjem. Det ble så opp til disse elevene og deres foresatte å si ja eller nei til å delta i borgerpanelet. Til slutt var det 21 elever som deltok i borgerpanelet, og disse ble betalt på samme nivå som ungdommens bystyre i Stavanger.

Det ble satt opp 5 samlinger på 3 timer hver, som skulle foregå etter skoletid. Prosessen var lik de tidligere borgerpanelene med de samme fasene. Det ble imidlertid gjort en rekke tilpasninger med tanke på at det var unge som deltok i panelet (Kristin Kverneland, personlig kommunikasjon, 20.09.2022). For det første var det fokus på å ha korte presentasjoner, på maks 15 minutter hver. Man fokuserte også på å bruke ord og uttrykk som var tilpasset ungdom i alderen 13--15 år. Det ble ikke lagt opp til at man måtte lese lange tekster, siden man kunne ha med ungdommer med lese- og skrivevansker. Og man hadde oppvarmingsaktiviteter som var tilpasset alderen, med fasilitering av frivillige som jobbet på KFUK/KFUM Forandringshuset (rettet mot ungdom). Når det gjaldt fasiliteringen, ble dette løst i en hybridform. Her ble fasiliteringen gjort av personer fra Smartby Stavanger, Grønn by, Forandringshuset og SoCentral. Det ble dermed innhentet fasiliteringskompetanse fra en rekke forskjellige aktører. Et interessant aspekt her er at de fleste av personene som var fasilitatorer, deltok på et fasiliteringskurs arrangert av MASS LBP, en Canadisk organisasjon, i forkant. Derfor ble det fokus på å bygge opp kompetanse før prosessen.

Det ble også litt endringer i prosessen siden man merket at det å holde foredrag ikke var så effektivt. Det å gå på befaring i nabolaget var en mye mer effektiv metode for læring. Dette er noe man ser også internasjonalt, og ikke bare i borgerpaneler for unge. For eksempel har borgerpanelet i Irland om biomangfold tatt i bruk befaring i stor grad i deres kunnskaps- og læringsfase (Clodagh Harris, personlig kommunikasjon, 28.09.2022).

\hypertarget{oslo-borgerpanel-puxe5-grefsen-og-kjelsuxe5s}{%
\paragraph{Oslo -- Borgerpanel på Grefsen og Kjelsås}\label{oslo-borgerpanel-puxe5-grefsen-og-kjelsuxe5s}}

I januar og februar 2021 ble borgerpanelet på Grefsen og Kjelsås gjennomført. Dette var et borgerpanel som ble gjennomført som del av forskningsprosjektet «New Water Ways», som var ledet av Norsk institutt for vannforskning i samarbeid med SoCentral og Vann- og avløpsetaten i Oslo. Selve borgerpanelprosessen var det SoCentral som hadde ansvar for. Problemet de skulle ta for seg, var

\begin{quote}
hvordan kan kommunen legge til rette for at dere som innbyggere best involveres i håndteringen av overvann på Grefsen-Kjelsås? (\protect\hyperlink{ref-skar_borgerpanel_2021}{Skar et al. 2021, 6}).
\end{quote}

Loddtrekningen fulgte den samme prosessen som ellers (borgerlotteri), men i dette borgerpanelet ble det for første gang eksperimentert med å sende ut invitasjonene via tekstmelding i stedet for brev. Det ble sendt ut 3000 tekstmeldinger til tilfeldig uttrukkede beboere på Grefsen og Kjelsås, og av disse svarte 267 personer ja til å delta. Blant de som takket ja, ble det trukket ut 20 personer ut fra alder, kjønn og geografi.

Prosessen i borgerpanelet var delt inn i fire møter, som hadde hvert sitt fokus: innsikt, refleksjon, vurdering og anbefaling. Dermed kan vi si at den samme formelen som borgerpaneler vanligvis har, ble fulgt også her.

Dette var et rent digitalt borgerpanel. Det var en rekke spørsmål knyttet til dette. Det viktigste spørsmålet var kanskje om man klarte å skape en god stemning og kultur for god deliberasjon? (\protect\hyperlink{ref-skar_borgerpanel_2021}{Skar et al. 2021, 16}). Som tidligere nevnt var en av utfordringene til Borgerkraft i Trondheim nettopp overgangen til et digitalt borgerpanel. Til sammenligning var planlagt fra starten av at dette borgerpanelet skulle være heldigitalt. En av styrkene i denne prosessen var at aktøren som designet og ledet arbeidet, SoCentral, hadde opparbeidet seg en god kompetanse på digitale møter og workshops før dette borgerpanelet kom i stand, og denne erfaringen ble definitivt innarbeidet i denne prosessen. SoCentral ringte til deltakerne en måned før borgerpanelet skulle starte, for å svare på eventuelle spørsmål deltakerne måtte ha, samt for å finne ut hvilken tilrettelegging som var nødvendig. Dette gjorde at de digitale utfordringene, som kan være ekskluderende, ble så godt som eliminert. Et annet element som bidro til dette, var en nøye planlagt kjøreplan, som rammet inn hele prosessen. Før hver samling ble deltakerne informert om hvor de var nå, og hva de skulle med dette møtet. Slik sett ble det enkelt for deltakerne å følge med på prosessen og å vite hvor man ville til slutt.

Digital deliberasjon kan med andre ord forsterke enkelte ulikheter i samfunnet dersom man ikke aktivt motvirker disse trendene. Det gjelder å jobbe aktivt med tilrettelegging før møtene skal finne sted, ved å kontakte deltakerne og sammen teste ut plattformen som skal brukes. Før man begynner, må man også ha en nøye planlagt kjøreplan, siden digitale møter ofte oppleves som mer slitsomme. Under møtene må man ha personer tilgjengelig som kan gi teknisk hjelp, og man må ha testet ut de forskjellige funksjonene på forhånd. Man kan også stille seg spørsmålet om kanskje en hybridmodell (en kombinasjon av digital og fysisk) kan være et alternativ, for å få inn noen av de sosiale elementene som det er vanskelig å skape i en digital versjon.

Spesielt viktig å merke seg er det at man gjorde mye arbeid i forkant med de som skulle presentere informasjon til borgerpanelet i kunnskaps- og læringsfasen. De la opp til korte presentasjoner med fokus på interaksjon mellom personen som holdt presentasjonen, og deltakerne i borgerpanelet. Interessant nok oppfordret man også i dette borgerpanelet deltakerne til å etterspørre mer informasjon dersom de følte at det var nødvendig. Medlemmene etterspurte da mer informasjon rundt de største utfordringene med overvann i deres område og kommunens plan for dette arbeidet (\protect\hyperlink{ref-skar_borgerpanel_2021}{Skar et al. 2021}). Derfor ble det gitt en ny presentasjon med dette temaet på neste samling. Slik sett er dette et eksempel på at man ga deltakerne i borgerpanelet en mer aktiv rolle også i designet av selve prosessen. Ved modeller som har en prosess som går over flere samlinger, som borgerjurymodellen, er dette en praksis som burde ha være mer bakt inn i designet.

I selve prosessen ble det bestemt regler for bruken av møteledere og fasilitatorer. Møtelederen hadde ansvaret for alt som skjedde i plenum, mens fasilitatorene ledet gruppene i deres diskusjoner. Det var en fasilitator per gruppe på 5--6 deltakere. Man laget også kjøreregler for gruppefasilitatorene for å sørge for at prosessen opprettholdt standardene for god deliberasjon (\protect\hyperlink{ref-skar_borgerpanel_2021}{Skar et al. 2021}). Man hadde også en egen person som var teknisk ansvarlig som kunne ta for seg eventuelle tekniske problemer.

Borgerpanelet på Grefsen og Kjelsås var en svært godt planlagt prosess med klare rolleforståelser og klare mål for hver samling. Det er imidlertid verdt å merke seg at dette var en ekstremt krevende prosess, og de personene som var med i prosessen, jobbet intenst med prosessdesign og fasiliteringene for å levere den prosessen de gjorde. De hadde god kunnskap om fasilitering fra før og hadde også tilegnet seg mer kunnskap fra kontakter internasjonalt om hvordan man designet nettopp deliberative prosesser.

\hypertarget{ruxf8ros-permanent-borgerpanel}{%
\paragraph{Røros -- Permanent borgerpanel}\label{ruxf8ros-permanent-borgerpanel}}

Borgerpanelet på Røros er per dags dato ikke blitt gjennomført enda. Dette borgerpanelet er imidlertid interessant av flere grunner, først og fremst fordi dette ikke er en stor bykommune, og fordi det er politisk vedtatt å etablere et permanent borgerpanel i kommunen.

I Røros begynte man i 2019 å se på nye måter å involvere innbyggerne på. Det var et ønske om å få flere inn i de politiske prosessene og å ha mer dialog med borgerne. Et av elementene som det ble pekt på, var det å skape et borgerpanel. I mai 2020 ble det vedtatt i kommunestyret å etablere et permanent borgerpanel.

\begin{figure}

{\centering \includegraphics[width=0.8\linewidth]{figs/røros} 

}

\caption{Mulig struktur for borgerpaneler på Røros}\label{fig:unnamed-chunk-5}
\end{figure}

Borgerpanelet skal bestå av 16--30 personer, og deltakerne skal sitte i to år om gangen. Deltakerne blir plukket ut gjennom en loddtrekning, og man skal sørge for at disse personene speiler Røros så godt det lar seg gjøre. I startfasen skal dette borgerpanelet få to saker i året som de skal deliberere over -- én sak på våren og én på høsten. Sakene blir gitt til borgerpanelet av kommunestyret/formannskapet. Etter deliberasjonsprosessen skal de komme med sin rapport, som sendes tilbake til kommunestyret for behandling. I den forstand har den likheter med modellen som ble designet for mulig institusjonalisering i Trondheim kommune (\protect\hyperlink{ref-trondheim_kommune_den_2019}{2019}).

En mulig struktur i Røros er at hvert år, under budsjettkonferansen fra september til november, blir det bestemt problemstillinger og dilemmaer som de to borgerpanelene skal ta for seg neste år. Det første borgerpanelet blir da startet opp i januar hvert år, mens det andre starter i august hvert år. Prosessene til de to borgerpanelene kan ses i figuren nedenfor.

Borgerpanelet i Røros skal ha en rådgivende rolle. Det er også bestemt at den digitale plattformen Decidim skal brukes aktivt. Den kan både brukes under selve borgerpanelet for å gjennomføre digitale møter og gjøre det mulig for resten av befolkningen å komme med innspill om hva de tenker vil være viktige spørsmål og dilemmaer. Slik sett kan vi se at borgerpanelet på Røros vil bli koblet opp mot andre medvirkningsprosesser, slik man gjorde med blant annet Trondheimspanelet.

Røros er en forholdsvis liten kommune, og derfor er det noen utfordringer knyttet til å etablere et fast borgerpanel. Et av elementene er fasiliteringskompetanse og oppbygging av en kultur rundt borgerpanelet. Som vi ser av de tidligere nevnte eksemplene ble fasiliteringskompetansen, og hvor man hentet den fra, løst på forskjellige måter i de forskjellige borgerpanelene. Et fellestrekk er imidlertid at kommunene prøvde, så langt det var mulig, å spille på de ressursene de hadde selv, og la til kompetanse fra eksterne aktører der det var mulig og nødvendig. Når kompetanseheving var nødvendig, ble det arrangert kurs for fasilitering. For en kommune som Røros vil det være sentralt at man får mulighet og tilgang til kompetanseheving for å gjennomføre gode deliberasjonsprosesser.

Internasjonalt har man begynt å se etablering av mer permanente borgerpaneler, eller en institusjonalisering, som OECD kaller det. Om dette er veien å gå når det gjelder borgerpaneler, er for tidlig å si. De fleste borgerpaneleksemplene vi har sett, har vært etablert ad hoc om enkeltsaker. Røros vil derfor være et interessant eksempel på hvordan et slikt permanent borgerpanel kan se ut, og hvordan det vil fungere.

\newpage

\hypertarget{om-forfatterne}{%
\section{Om forfatterne}\label{om-forfatterne}}

\textbf{Sveinung Arnesen} er statsviter med PhD fra Institutt for sammenlignende politikk ved Universitetet i Bergen. Han er fagleder for Demokrati og innovasjon ved forskningsinstituttet NORCE og førsteamanuensis ved Institutt for politikk og forvaltning, UiB. Arnesen er nasjonal koordinator for Den europeiske samfunnsundersøkelsen (ESS) og forsker på tematikk knyttet til valg, opinion og demokrati.

\textbf{Henrik Litleré Bentsen} er statsviter med doktorgrad fra Institutt for sammenlignende politikk ved Universitetet i Bergen. I dag arbeider han som forsker ved NORCE Helse og Samfunn. Bentsen arbeider med ulike temaer innenfor dommeradferd og offentlig opinion om domstoler, beslutningsprosedyrer og demokratisk legitimitet, virkningen av byborgerpaneler på offentlige oppfatninger om styring og demokrati, offentlige oppfatninger om hydrogen og energiomstilling samt deltakelse i arbeidslivet.

\textbf{Pål Bjørseth} jobber som rådgiver på byrådsleders avdeling i Bergen kommune, og han jobber spesielt med spørsmål knyttet til parlamentarisme, lokaldemokratiutvikling, samfunnsplanlegging og samordning på FoU-området.

\textbf{Anne Lise Fimreite} er utdannet statsviter og er professor ved Institutt for politikk og forvaltning, UiB. I sin forskning har Fimreite konsentrert seg om offentlig forvaltning og offentlig politikk. Hun er særlig opptatt av statlig styring og kommunenes handlingsrom, kommunenes økonomiske situasjon og offentlig reformvirksomhet, samt forholdet mellom forvaltningsnivåene. Hun har blant annet ledet den forskningsbaserte evalueringen av NAV-reformen (2007--2012) og prosjektet «Styringssystem i storby» (2002--2003), og i 2016--2017 ledet hun det byrådsoppnevnte lokaldemokratiutvalget i Bergen kommune som la frem rapporten «Byen og nærdemokratiet».

\textbf{Arild Ohren} er PhD-kandidat ved NTNU.
Hans forskning fokuserer på borgerpaneler og hvilken type form for politisk representasjon de representerer.
Arild er medlem av OECDs «Innovative Citizen Participation Network», samt også medlem av forskningsklyngen om demokratisk representasjon i Participedia.

\textbf{Jon Kåre Skiple} er utdannet statsviter med doktorgrad fra Institutt for sammenlignende politikk ved Universitetet i Bergen. Hans forskningsinteresser er knyttet til domstoler, politisk deltakelse og offentlig opinion.

\textbf{Jacob Aars} er professor ved Institutt for politikk og forvaltning, Universitetet i Bergen.
Aars har arbeidet med lokalstyre og lokalt demokrati.
Han har blant annet publisert flere artikler om kommuners tiltak for å styrke deltakelse i lokalpolitikken.
Han har også bistått mange kommuner i deres arbeid med å utvikle og designe borgerpaneler i Norge.

\newpage

\hypertarget{bibliography}{%
\section*{Referanser}\label{bibliography}}
\addcontentsline{toc}{section}{Referanser}

\hypertarget{refs}{}
\begin{CSLReferences}{1}{0}
\leavevmode\vadjust pre{\hypertarget{ref-aristoteles_politikk_2007}{}}%
Aristoteles. 2007. \emph{Politikk}. Translated by Tormod Eide. Oslo: Vidarforl.

\leavevmode\vadjust pre{\hypertarget{ref-arnesenloddet}{}}%
Arnesen, Sveinung, Anne Lise Fimreite, and Jacob Aars. 2021. {``Loddet Er Kastet: Om Bruken Av Innbyggerpaneler i Lokalpolitikken.''} Edited by Saglie Jo, Segaard Signe B., and Dag A. Christensen, 253--83. \url{https://doi.org/10.23865/noasp.134.ch10}.

\leavevmode\vadjust pre{\hypertarget{ref-arnesendelib2021}{}}%
Arnesen, Sveinung, Anne Lise Fimreite, and Jon Kåre Skiple. 2021. {``Deliberativ Meningsmåling i Bergen. Rapport Til Bystyret i Bergen.''} \url{url:\%20https://bookdown.org/connect/\#/apps/8cc01bc4-19c9-444e-93e5-a616021760bb/access}.

\leavevmode\vadjust pre{\hypertarget{ref-bachtiger_deliberative_2018}{}}%
Bächtiger, Andre, John S Dryzek, Jane Mansbridge, and Mark Warren. 2018. {``Deliberative {Democracy} : {An} {Introduction}.''} In \emph{The {Oxford} {Handbook} of {Deliberative} {Democracy}}, edited by Andre Bächtiger, John S. Dryzek, Jane Mansbridge, and Mark Warren, 1--35. Oxford: Oxford University Press. \url{https://doi.org/10.1093/oxfordhb/9780198747369.013.50}.

\leavevmode\vadjust pre{\hypertarget{ref-boulianne_beyond_2018}{}}%
Boulianne, Shelley. 2018. {``Beyond the {Usual} {Suspects}: {Representation} in {Deliberative} {Exercises}.''} In \emph{Public {Deliberation} on {Climate} {Change}. {Lessons} from {Alberta} {Climate} {Dialogue}}, edited by Lorelei L. Hanson, 109--33. Athabasca University: AU Press.

\leavevmode\vadjust pre{\hypertarget{ref-cook2002experimental}{}}%
Cook, Thomas D, Donald Thomas Campbell, and William Shadish. 2002. \emph{Experimental and Quasi-Experimental Designs for Generalized Causal Inference}. Houghton Mifflin Boston, MA.

\leavevmode\vadjust pre{\hypertarget{ref-courant_sortition_2019}{}}%
Courant, Dimitri. 2019. {``Sortition and {Democratic} {Principles}: {A} {Comparative} {Analysis}.''} In \emph{Legislature by {Lot} : Transformative Designs for Deliberative Governance}, edited by John Gastil and Erik Olin Wright, 229--48. London; New York: Verso.

\leavevmode\vadjust pre{\hypertarget{ref-curato_deliberative_2021}{}}%
Curato, Nicole, David M. Farrell, Brigitte Geissel, Kimmo Grönlund, Patricia Mockler, Jean-Benoit Pilet, Alan Renwick, Jonathan Rose, Maija Setälä, and Jane Suiter. 2021. \emph{Deliberative {Mini}-{Publics}: {Core} {Design} {Features}}. Bristol: Bristol University Press.

\leavevmode\vadjust pre{\hypertarget{ref-dahlberg2015democratic}{}}%
Dahlberg, Stefan, Jonas Linde, and Sören Holmberg. 2015. {``Democratic Discontent in Old and New Democracies: Assessing the Importance of Democratic Input and Governmental Output.''} \emph{Political Studies} 63: 18--37.

\leavevmode\vadjust pre{\hypertarget{ref-dalton2004democratic}{}}%
Dalton, Russell J. 2004. \emph{Democratic Challenges, Democratic Choices}. Oxford univ. press.

\leavevmode\vadjust pre{\hypertarget{ref-elstub_mini-publics_2014}{}}%
Elstub, Stephen. 2014. {``Mini-Publics: {Issues} and Cases.''} In \emph{Deliberative {Democracy}: {Issues} and {Cases}}, edited by Stephen Elstub and Peter McLaverty, 166--88. Edinburgh: Edinburgh University Press Ltd.

\leavevmode\vadjust pre{\hypertarget{ref-escobar_forms_2017}{}}%
Escobar, Oliver, and Stephen Elstub. 2017. {``Forms of {Mini}-Publics: {An} Introduction to Deliberative Innovations in Democratic Practice.''} New Democracy Foundation.

\leavevmode\vadjust pre{\hypertarget{ref-escobar_defining_2019}{}}%
---------. 2019. {``Defining and Typologising Democratic Innovations.''} In \emph{Handbook of {Democratic} {Innovation} and {Governance}}, 11--31. Cheltenham, UK; Northampton, MA, USA: Edward Elgar Publishing.

\leavevmode\vadjust pre{\hypertarget{ref-farrell_deliberative_2019}{}}%
Farrell, David M, Nicole Curato, John S Dryzek, Brigitte Geißel, Kimmo Grönlund, Sofie Marien, Simon Niemeyer, et al. 2019. {``Deliberative {Mini}-{Publics}: {Core} {Design} {Features}.''} The Centre for Deliberative Democracy \& Global Governance.

\leavevmode\vadjust pre{\hypertarget{ref-farrell_sortition_2019}{}}%
Farrell, David M., and Peter Stone. 2019. {``Sortition and {Mini}-{Publics}: {A} {Different} {Kind} of {Representation}.''} In \emph{Handbook of {Political} {Representation} in {Liberal} {Democracies}}, edited by Robert Rohrschneider and Jacques Thomassen. Oxford: Oxford University Press.

\leavevmode\vadjust pre{\hypertarget{ref-fimreitebyen}{}}%
Fimreite, Anne Lise. 2018. {``Byen Og Nærdemokratiet - Rapport Fra Lokaldemokratiutvalget.''} Bergen kommune. \url{chrome-extension://efaidnbmnnnibpcajpcglclefindmkaj/https://www.bergen.kommune.no/politikere-utvalg/api/fil/206337/Rapport-Byen-og-Naerdemokratiet}.

\leavevmode\vadjust pre{\hypertarget{ref-fishkin2018democracy}{}}%
Fishkin, James S. 2018a. \emph{Democracy When the People Are Thinking: Revitalizing Our Politics Through Public Deliberation}. Oxford University Press.

\leavevmode\vadjust pre{\hypertarget{ref-fishkin2021deliberative}{}}%
---------. 2021. {``Deliberative Public Consultation via Deliberative Polling: Criteria and Methods.''} \emph{Hastings Center Report} 51: S19--24.

\leavevmode\vadjust pre{\hypertarget{ref-fishkin_democracy_2018}{}}%
Fishkin, James S. 2018b. \emph{Democracy {When} the {People} {Are} {Thinking}: {Revitalizing} {Our} {Politics} {Through} {Public} {Deliberation}}. Oxford University Press.

\leavevmode\vadjust pre{\hypertarget{ref-fishkin2005experimenting}{}}%
Fishkin, James S, and Robert C Luskin. 2005. {``Experimenting with a Democratic Ideal: Deliberative Polling and Public Opinion.''} \emph{Acta Politica} 40 (3): 284--98.

\leavevmode\vadjust pre{\hypertarget{ref-fishkin_deliberative_2018}{}}%
Fishkin, James S., Max Senges, Eileen Donahoe, Larry Diamond, and Alice Siu. 2018. {``Deliberative Polling for Multistakeholder Internet Governance: Considered Judgments on Access for the Next Billion.''} \emph{Information Communication and Society} 21 (11): 1541--54. \url{https://doi.org/10.1080/1369118X.2017.1340497}.

\leavevmode\vadjust pre{\hypertarget{ref-fixdal_consensus_1997}{}}%
Fixdal, Jon. 1997. {``Consensus Conferences as {`Extended Peer Groups'}.''} \emph{Science and Public Policy} 24 (6): 366--76. \url{https://doi.org/10.1093/spp/24.6.366}.

\leavevmode\vadjust pre{\hypertarget{ref-flanigan_fair_2021}{}}%
Flanigan, Bailey, Paul Gölz, Anupam Gupta, Brett Hennig, and Ariel D. Procaccia. 2021. {``Fair Algorithms for Selecting Citizens' Assemblies.''} \emph{Nature} 596 (7873): 548--52. \url{https://doi.org/10.1038/s41586-021-03788-6}.

\leavevmode\vadjust pre{\hypertarget{ref-fournier_henk_van_der_kolk_when_2011}{}}%
Fournier Henk Van Der Kolk, Patrick, R Kenneth Carty, André Blais, and Jonathan Rose. 2011. {``When {Citizens} {Decide}: {Lessons} from {Citizen} {Assemblies} on {Electoral} {Reform}.''}

\leavevmode\vadjust pre{\hypertarget{ref-gastil_political_2008}{}}%
Gastil, John. 2008. \emph{Political Communication and Deliberation}. London, UK: SAGE Publications Inc. \url{https://doi.org/10.4135/9781483329208}.

\leavevmode\vadjust pre{\hypertarget{ref-1885-53297}{}}%
Hendriks, Carolyn. 2005. {``Lay Citizen Deliberations: {Consensus} Conferences and Planning Cells.''} In \emph{The Deliberative Democracy Handbook: Strategies for Effective Civic Engagement in the Twenty-First Century}, 80--110. USA: Jossey-Bass Inc.

\leavevmode\vadjust pre{\hypertarget{ref-jacquet_explaining_2017}{}}%
Jacquet, Vincent. 2017. {``Explaining Non-Participation in Deliberative Mini-Publics.''} \emph{European Journal of Political Research} 56 (3): 640--59. \url{https://doi.org/10.1111/1475-6765.12195}.

\leavevmode\vadjust pre{\hypertarget{ref-trondheim_kommune_den_2019}{}}%
kommune, Trondheim. 2019. {``Den Samskapte Kommunen.''} \url{https://innsyn.trondheim.kommune.no/motekalender/motedag/1003390374}.

\leavevmode\vadjust pre{\hypertarget{ref-lacelle-webster_citizens_2021}{}}%
Lacelle-Webster, Antonin, and Mark E. Warren. 2021. {``Citizens' {Assemblies} and {Democracy}.''} In \emph{Oxford {Research} {Encyclopedia} of {Politics}}. Oxford University Press. \url{https://doi.org/10.1093/acrefore/9780190228637.013.1975}.

\leavevmode\vadjust pre{\hypertarget{ref-mass_lbp_how_2017}{}}%
LBP, MASS. 2017. {``How to Run a {Civic} {Lottery}: {Designing} Fair Selection Mechanisms for Deliberative Public Processes {A} {Guide} and {License} {Version} 1.4.''} \url{https://www.masslbp.com/resources}.

\leavevmode\vadjust pre{\hypertarget{ref-mansbridge_should_1999}{}}%
Mansbridge, Jane. 1999. {``Should {Blacks} {Represent} {Blacks} and {Women} {Represent} {Women}? {A} {Contingent} "{Yes}".''} \emph{The Journal of Politics} 61 (3): 628--57. \url{https://doi.org/10.2307/2647821}.

\leavevmode\vadjust pre{\hypertarget{ref-mao_deciding_2013}{}}%
Mao, Yuping, and Marco Adria. 2013. {``Deciding Who Will Decide: {Assessing} Random Selection for Participants in {Edmonton}'s {Citizen} {Panel} on Budget Priorities.''} \emph{Canadian Public Administration} 56 (4): 610--37. \url{https://doi.org/10.1111/capa.12042}.

\leavevmode\vadjust pre{\hypertarget{ref-marien2019fair}{}}%
Marien, Sofie, and Hannah Werner. 2019. {``Fair Treatment, Fair Play? The Relationship Between Fair Treatment Perceptions, Political Trust and Compliant and Cooperative Attitudes Cross-Nationally.''} \emph{European Journal of Political Research} 58 (1): 72--95.

\leavevmode\vadjust pre{\hypertarget{ref-oecd_innovative_2020}{}}%
OECD. 2020. {``Innovative {Citizen} {Participation} and {New} {Democratic} {Institutions}.''} OECD. \url{https://doi.org/10.1787/339306da-en}.

\leavevmode\vadjust pre{\hypertarget{ref-ohren_representative_nodate}{}}%
Ohren, Arild. n.d. {``The {Representative} {Claim} of {Deliberative} {Mini}-{Publics}.''} PhD thesis, Department of Sociology; Political Science: Norwegian University of Science; Technology.

\leavevmode\vadjust pre{\hypertarget{ref-ryan_defining_2014}{}}%
Ryan, Matthew, and Graham Smith. 2014. {``Defining {Mini}-{Publics}.''} In \emph{Deliberative {Mini}-{Publics}: {Involving} {Citizens} in the {Democratic} {Process}}, edited by Kimmo Grönlund, André Bächtiger, and Maija Setälä, 9--26. Colchester: ECPR Press.

\leavevmode\vadjust pre{\hypertarget{ref-ryfe_participation_2012}{}}%
Ryfe, David Michael, and Brittany Stalburg. 2012. {``The {Participation} and {Recruitment} {Challenge}.''} In \emph{Challenge {Democracy} in {Motion}: {Evaluating} the {Practice} and {Impact} of {Deliberative} {Civic} {Engagement}}, edited by Tina Nabatchi, John Gastil, Matt Leighninger, and G. Michael Weiksner, 43--58. Oxford: Oxford University Press.

\leavevmode\vadjust pre{\hypertarget{ref-skar_borgerpanel_2021}{}}%
Skar, Cathrine, Marie Harbo Dahle, Håkon Iversen, Kjetil Kristensen, and Mette Øinæs Habberstad. 2021. {``Borgerpanel På {Grefsen} Og {Kjelsås}: {En} Metoderapport.''} SoCentral. \url{https://www.socentral.no/aktuelt/borgerpanel-bidrar-med-losninger/}.

\leavevmode\vadjust pre{\hypertarget{ref-smith_democratic_2009}{}}%
Smith, Graham. 2009. \emph{Democratic Innovations: {Designing} Institutions for Citizen Participation}. Cambridge: Cambridge University Press. \url{https://doi.org/10.1017/CBO9780511609848}.

\leavevmode\vadjust pre{\hypertarget{ref-smith_mini-publics_2018}{}}%
Smith, Graham, and Maija Setälä. 2018. {``Mini-{Publics} and {Deliberative} {Democracy}.''} In \emph{The {Oxford} {Handbook} of {Deliberative} {Democracy}}, edited by Andre Bächtiger, John S. Dryzek, Jane Mansbridge, and Mark Warren, 299--314. Oxford: Oxford University Press. \url{https://doi.org/10.1093/oxfordhb/9780198747369.013.27}.

\leavevmode\vadjust pre{\hypertarget{ref-noauthor_ungt_2022}{}}%
{``Ungt {Borgerpanel}: {`{Hvordan} Kan Vi Sammen Skape Et Trygt Og Attraktivt Nabolag?'}.''} 2022. Stavanger Kommune, NTNU, SoCentral, Pådriv.

\leavevmode\vadjust pre{\hypertarget{ref-warren_citizen_2008}{}}%
Warren, Mark E. 2008. {``Citizen Representatives.''} In \emph{Designing {Deliberative} {Democracy}}, edited by Mark E. Warren and Hilary Pearse, 50--69. Cambridge: Cambridge University Press. \url{https://doi.org/10.1017/CBO9780511491177.004}.

\end{CSLReferences}

\end{document}
